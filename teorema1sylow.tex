\documentclass{article}
\usepackage[utf8]{inputenc}
\usepackage[spanish]{babel}
\usepackage{graphicx}
\usepackage{geometry}
\usepackage{multicol}
\usepackage{amsmath}
\usepackage{amsthm,amsfonts,amssymb}%paquetes AMS

\title{teroema1sylow}
\author{wejerezh }
\date{September 2021}

\begin{document}

%\maketitle
\section{Preliminares}
\begin{enumerate}
    \item Si N es un subgrupo normal de un grupo G, entonces las clases laterales de N en G forman un grupo G/N con la operación (aN)(bN)=abN. Este grupo se llama cociente de G por N.
    \item Sea H un subgrupo de un grupo G. El número de clases laterales izquierdas de H en G es el índice $(G: H)$ de H en G.
    \item Sea H un subgrupo normal de G. Defina el homorfismo natural o homomorfismo canónico
    \begin{center}
        $\phi = G \longrightarrow G/H$
    \end{center}
    por :
    \begin{center}
        $\phi(g)= gH$
    \end{center}
    \item Un grupo G es un  p-grupo si todo elemento en G tiene orden potencia de p, donde p en un número primo. Un subgrupo de un grupo G es un p-subgrupo si es un p-grupo.
    \item El conjunto:
    \begin{center}
        $N(H)=\{g \in G:gHg^{-1}=H\}$
    \end{center}
    es un subgrupo de G llamado normalizador de H en G. Notemos que H es un subgrupo normal de N(H). De hecho, N(H) es el mayor subgrupo de G en el que H es normal
    \item Sea N un subgrupo normal de un grupo G. Las clases laterales de N en G forman un grupo G/N de orden [G:N].
    \item (Teorema de Cauchy) Sea G un grupo tal que p divide el orden de G. Entonces G tiene un elemento de orden p y por lo tanto un subgrupo de orden p.
    \item lema(17.2) Sea H un p-subgrupo de un grupo finito G. Entonces
        \begin{center}
            (N[H] : H) $\equiv$ (G : H) (mod p).
        \end{center}
\end{enumerate}
\newpage
\section{Primer Teorema de Sylow.}
Sea G un grupo finito y sea $|G|=p^{n}m $ donde $n\geq 1$ y donde p no divide m. Entonces
\begin{enumerate}
    \item G contiene un subgrupo de orden $p^{i}$ para cada i en $1 \leq i \leq n$
    \item Todo subgrupo H de G de orden $p^{i}$ es un subgrupo normal de un subgrupo de orden $p^{i+1}$ para  $1 \leq i \leq n.$
\end{enumerate}

\textbf{demostración}\\
\textbf{1}. Sabemos que G contiene un subgrupo de orden p por el teorema de Cauchy. Usamos un argumento de inducción y mostramos que la existencia de un subgrupo de orden $p^{i}$ para $i \leq n$ implica la existencia de un subgrupo de orden $p^{i+1}$. \\
Sea H un subgrupo de orden $p^{i}$ para $i \leq n $.nosotros vemos que p divide a $(G:H)$ por el lema 17.2, nosotros entonces  sabemos que p divide $(N [H]: H).$para H subgrupo normal de N [H].\\
Nosotros podemos formar $N [H] / H$, y vemos que p divide $|N [H] / H|.$ Según el teorema de Cauchy, el grupo de factores $N [H] / H$ tiene un subgrupo K, que es de orden p. \\
Si $\gamma: N [H] \longrightarrow N [H] / H$ es el homomorfismo canónico, entonces $\gamma^{-1} [K] = \{x \in N [H] | \gamma (x) \in K\} $es un subgrupo de $N [H]$ y por lo tanto de G. Este subgrupo contiene a H y es de orden $p^{i+1}$\\

\textbf{2}. Repetimos la construcción de la parte 1 y notamos que $ H \leq \gamma ^ {- 1} \leq N [H] $ donde $ \gamma ^ {- 1} [K] = p ^ {i + 1} $. Dado que es normal en N [H], por supuesto es normal en el grupo posiblemente más pequeño $ \gamma ^ {- 1} [K] $

\end{document}