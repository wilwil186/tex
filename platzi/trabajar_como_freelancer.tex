\documentclass{article}
\usepackage[utf8]{inputenc}
\usepackage{hyperref}
\title{Curso para Trabajar como Freelance}
\author{wejerezh }
\date{October 2021}

\begin{document}

\pagestyle{empty}
%\afterpage{\blankpage}
\begin{center}
\begin{figure}[h]
\centering


\end{figure}
\Large
\hrule
\vspace{4mm}
\textbf{Notas del Curso para Trabajar como Freelance}\\

\vspace{4mm}
\hrule
\large
\vfill
Autor\\

Wilson Eduardo Jerez Hernández \\
\vfill
Profesora\\
Anahí Salgado Díaz de la Vega
\vfill
Platzi\
Escuela de Habilidades Blandas\\
Curso para Trabajar como Freelance\\
\end{center}
\newpage
\newpage
\addtocontents{toc}{\hfill \textbf{Página} \par}
\tableofcontents
\newpage

\section{Qué aprenderás sobre el mundo del freelance}
    En este curso aprenderás todo lo necesario para empezar a trabajar como freelancer sin importar si eres diseñador, programador, escritor, traductor. El curso es para cualquier profesión.
    
    Objetivos del curso:
    \begin{enumerate}
        \item Entender las habilidades que debe tener un Freelance
        \item Crear tu propio portafolio e impulsar tu marca personal para atraer clientes
        \item Analizar cómo establecer el precio de un proyecto
        \item Determinar cómo establecer un contrato
        \item Decidir cómo organizar el espacio de trabajo de un freelance
    \end{enumerate}
\section{Ser o no ser un Freelancer} 
    El término freelancer, como dato curioso, viene de las personas que se conocían como mercenarios, personas dueñas de su tiempo, de su agenda. Eran personas que eran contratadas para asesinar o secuestrar a otras personas.
    
    Un freelancer es un trabajador independiente que ofrece sus servicios específicos para empresas o personas. Trabaja de forma autónoma gestionando su tiempo y trabajo.
    
    No tiene horarios fijos, ni usa uniforme, tampoco sufre por el tráfico de la ciudad o el ajetreo del transporte público.
    
    Ser freelancer es un estilo de vida, es una actitud ante la profesión. Una responsabilidad que debe tener, es ser responsable de si mismo.
    
    Tu carrera profesional dependerá solo de ti y por eso es importante tener un portafolio, estar actualizado sobre tecnologías, tener proyectos claves, clientes claves, puntos interesantes para trabajar.

\section{Mi mejor momento para ser Freelancer}
    ¿Cuál es el mejor momento para ser freelancer?
    Cuando ya tienes experiencia trabajando en un ambiente laboral y en equipo, esto te va a servir de práctica para cuando estés en situaciones donde trabajes remotamente, sin dejar a un lado tu objetivo: lograr resultados.
    
    Cuando ya tienes disciplina en tu vida personal, y sabes comunicarte con las personas. Además de saber tratar a tus clientes como clientes y sepas vivir de manera organizada utilizando herramientas que te ayuden.
\section{Mi portafolio y mi marca personal}

    Debes tener un perfil de Linkedin no importa tu carrera o profesión, esta es la red profesional para conseguir empleo.
    \begin{enumerate}
        \item Incluye tus mejores proyectos, es tu oportunidad para mostrar la calidad de tu trabajo.
    
        \item Describe los skills que usaste para la construcción de tus proyectos. Colocándolas harás que más personas se interesen en tu perfil.
    
        \item Si tienes certificaciones muéstralas también.
    \end{enumerate}

\section{Dónde encontrar trabajo como Freelancer}
    \url{https://www.freelancer.es/}\\
    Es un sitio muy versátil donde encuentras proyectos de muchos temas y para muchas profesiones. Este es uno de los sitios con más experiencia y mejor reputación en el mercado Freelancer. Aquí encontrarás muchos proyectos de España.\\
    \url{https://www.workana.com/}\\
    Este es un sitio más enfocado a las áreas de diseño y programación, reúne una comunidad muy grande de este tipo de profesionales, encontrarás algunas startups y negocios que están comenzando solicitando tus servicios. Aunque siempre hay de todo.\\
    \url{https://www.fiverr.com/}\\
    Este es un sitio con un mercado más Estadounidense, si dominas inglés y te gustaría trabajar con este tipo de empresas comienza aquí.\\
    \url{https://www.soyfreelancer.com/}\\
    En este sitio abundan proyectos para LATAM si buscas algo más local este sitio es una gran opción. La diversidad de proyectos es muy amplia y puedes encontrar desde unos muy profesionales y te exijan un gran esfuerzo hasta otros más fáciles de ejecutar.

\section{Cómo cobrar como Freelancer}
tips:
\begin{enumerate}
    \item Averigua cuánto cobrar por proyecto.
    \item No hagas tratos con la familia porque puede complicar la relación familiar.
    \item Evalúa los costos relacionados
    \item Estima el esfuerzo que requiere cada proyecto.
    \item hacer presupuesto mensual
    \item Estima cuántos proyectos podrías hacer al mes. No te atiborres con proyectos, se realista en cuántos proyectos puedes hacer al mismo tiempo.
\end{enumerate}
\url{https://www.soyfreelancer.com/cuanto-cobrar-por-tu-trabajo}

\section{contratos}
    En este tema debemos ser serios y responsables. Debemos tener claro el pago y las condiciones que estamos solicitando. También debemos ser empáticos con el cliente que espera recibir garantías.
        \begin{enumerate}
            \item realiza un plan de trabajo 
            \item cronogramas 
            \item Documentación 
            \item Alcances
        \end{enumerate}
    se muy claro con lo que haces.
    \textbf{HelloSign}
    \url{https://www.hellosign.com/}
    es una plataforma que provee contratos(contractos digitales)
    \textbf{NUNCA}ejecutes un proyecto sin contrato.
    \textbf{cobros}
    50\% al arrancar
    50\% diferido en meses.
    
\section{Recibir pagos trabajando como Freelance}
siempre es una negociación. ten una cuenta bancaria a tu nombre. \\
transferencia electrónicas\\
Debes tener una cuenta en el banco a tu nombre, en algunas empresas no te van a realizar el pago si la cuenta no está a tu nombre. Esto funciona muy bien cuando están en el mismo país, si manejas negocios con bancos nacionales.
\\
Payoneer\\Esto funciona muy bien cuando tu cliente se encuentra en otro país. Payoneer es la forma de manejar transferencias electrónicas de manera internacional y te proporciona una tarjeta de débito, puedes manejar tu moneda en dolares y euros\\Paypal\\Es la opción amada por muchos. Es el método de pago más popular, por la facilidad para crear una cuenta. Es como una capa que te permite generar transferencias internacionales y que al final se depositará en tu cuenta bancaria nacional.

Paypal tiene una comisión al momento de hacer esta transferencia a tu cuenta bancaria nacional.\\Skrill\\Es un método de pago que también tiene bastante tiempo en el mercado.\\MercadoPago\\Con esta opción puedes generar un código QR y enviárselo a tu cliente y a partir de eso puede generar una transferencia a tu cuenta. Puede que tenga alguna restricción dependiendo de tu país. Funciona muy bien en México, Colombia y Argentina.

\section{Facturas e impuestos}
depende del pais. \textbf{Nunca evadir impuestos}
hay varias calculadoras en linea para calcular cuanto debemos pagar de impuestos, y calcular cuanto pagariamos para que nos demos una idea de cuanto cobrar.si estas en colombia no olvides registrarte como persona natural 

\section{Organizar mi espacio}

no uses tanto pijama por que te duermes, el comedor no es un buen lugar va a ver mucho ruido, tampoco en tu cama te duermes xd, no tener condiciones adecuadas de trabajo puede afectar tu productividad, si es imposible busca coworking 

\section{Organizar mi jornada laboral}

realiza tu horario de trabajo, se muy diciplinado la autonomia del tiempo puede ser un arma de doble filo, realiza pausas activas

\section{Soy Freelancer, soy productivo}
\url{https://www.atlassian.com/software/jira/guides} es de pago
\url{https://trello.com/}
usa google task para las pequeñas tareas.
google calendar.
\url{https://monday.com/lang/es/} tambien sirve de pago y free
tecnica pomodoro \url{https://francescocirillo.com/pages/pomodoro-technique} acomodola si quieres, la profe usa la aplicación be focus, comicate con tu equipo utiliza aplicaiones con slack, meet, zoom , skype

\section{Cuidarme laboralmente}

Al ahorrar para tu retiro debes dedicar un porcentaje de tus ingresos y hay una fórmula muy fácil para tener un fondo suficiente, es la siguiente: debes ahorrar un porcentaje igual a tu edad menos 20.
Cuidar mi trabajo, cuidar mis ingresos, cuidarme a mi.

Esto debería ser un mantra en nuestra vida diaria, debe ser un balance en nuestra vida que debemos tener y procurar
\begin{enumerate}
    \item Contrata un seguro médico. Al ser freelancer no tendremos este beneficio que nos otorga una compañía al ser obligado por la ley. Depende del freelancer contratarlo.
    \item Prepárate para el futuro. Contratar una Afore o un Fondo de pensiones también va a correr por tu cuenta al no estar trabajando en una compañía
    \item También puedes ir ahorrando una cantidad de dinero para tu casa.
    \item Busca más fuentes de ingresos que sean recurrentes, mensuales y puedas estar seguro algunos meses para no pasar momentos difíciles
\end{enumerate}

\section{Busca más fuentes de ingresos que sean recurrentes, mensuales y puedas estar seguro algunos meses para no pasar momentos difíciles}

Una de las artes que desarrollas con la experiencia es la de ser administrado. Administrar nuestras finanzas es uno de los retos más grandes y es aún más grande cuando tus ingresos no son regulares, como cuando eres Freelancer.

Si no puedes medir algo no puedes controlarlo, es por eso que voy a presentarte algunas app y sitios que te van a ayudar a mantener registro de tus gastos.

iHomeMoney
Uno de mis sitios favoritos es iHomeMoney me agrada mucho su interfaz es muy fácil de usar además que tiene gráficos muy interesantes que me permiten ver muy fácil en qué estoy gastando mi dinero.

YNAB
YNAB es una aplicación muy atractiva pues partes de un presupuesto para definir el curso de tus finanzas. Tener presupuestos para todo es una de las mejores prácticas para cuidar tu dinero.

Mint
Mint es un sitio y app que reúne todo lo anterior. Maneja gráficos sencillos de ver, presupuestos y un sistema de alertas (¡muy bueno!). Algo que la hace más atractiva es que funciona con Paypal y algunos bancos.

\section{Conclusiones}
Cuidar mi trabajo, cuidar mis ingresos y cuidarme a mi.
\end{document}
