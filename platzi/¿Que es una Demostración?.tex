\documentclass{article}
\usepackage[utf8]{inputenc}
\usepackage{hyperref}
\usepackage{graphicx}
\usepackage{geometry}
\usepackage{multicol}
\usepackage{amsmath}
\usepackage{amsthm,amsfonts,amssymb}%paquetes AMS

%comienza por los titulos
%Entrada o introducción al tema de forma enganchante 
     %cuerpo del blog fácil de escanear
%Citas con hipervínvulos y referencias para aumentar tu credibilidad 
%conclusión
%CTA... 

\begin{document}

\begin{center}
    \textbf{¿Como nacen las formulas en las matemáticas?}
\end{center}
    En el colegio e incluso en la universidad, nos enseñan a memorizar ciertas formulas de las matemáticas
    y algoritmos tales como:

    \begin{equation*}
        \frac{a}{b}+\frac{c}{d}=\frac{a \cdot d + b \cdot c}{b \cdot d}
    \end{equation*}

    Pero, sabes el ¿por que? ¿por que esto es verdad? Bueno en matemáticas a las verdades
    absolutas es decir algo que siempre es verdad pasé lo que pase lo llamamos teoremas, y concluimos
    que son verdaderas siempre que exista una Demostración. Una Demostración es una serie de pasos
    que nos permite concluir que cierta proposición es verdadera.
\\
\\ 
Ahora bien \textbf{¿Como hacemos para Demostrar el Teorema aquí expuesto?} bien en matemática
existe algo llamado \textbf{Axiomas de campo}, calma por definición un axioma es una proposición o
Enunciado tan evidente que se considera que no requiere Demostración. Y un cuerpo es simplemente
un sistema algebraico en donde existe la multiplicación y la adicción. Un ejemplo seria los naturales
 $\{0,1,2,3,4...\}$ pues sus elementos se pueden sumar y multiplicar entre si.
\\
\\ 
Explicado lo anterior vamos a enunciar los axiomas de campo:
\begin{enumerate}
    \item Para todo a,b que pertenecen a $\mathbb R$ (los Números Reales de toda la vida) se tiene: 
    $a + b = b + a$, y $a \cdot b = b \cdot a$.
    \item para todo a,b,c en $\mathbb R$ se tiene que:  $(a+b)+c = c+(a+b)$ y $a \cdot (b \cdot c)= (a \cdot b) \cdot c$
    \item Para todo a,b,c en  $\mathbb R$ se tiene que: $(a + b) \cdot c = a \cdot c + b \cdot c $
    \item Existen elementos 0 y 1 en $\mathbb{R}$ tales que $0 \not = 1$ y para cualquier a en $\mathbb R$ se cumple que 
    $x + 0 = 0 + x = x$ y $x \cdot 1 = 1 \cdot x = x.$
    \item Para cualquier a en $\mathbb R$ existe un -a, talque $a+(-a)=0$
    \item Para cualquier a en $\mathbb R$ existe un $\frac{1}{a}$ talque $a \cdot \frac{1}{a} = 1$ con $ a \not = 0$ 
     recordemos que a no puede ser igual a 0, porque si a = 0 entonces $\frac{1}{a}$ daría indeterminado, es decir su resultado es desconocido.
\end{enumerate}
     Ahora enunciemos un \textbf{Lema}, un lema es un teorema pero anteriormente se hizo la demostración.
\\ \\ 
\textbf{Lema} $\frac{a}{b}\cdot b \cdot d = a \cdot d $  y $\frac{c}{d} \cdot b \cdot d = c \cdot d$ para cualesquiera a,b,c,d en $\mathbb R$
con $b \not = 0$ y $d \not = 0$. (un consejo para demostrar este lema sera usando los axiomas 2,1, 6 y 4) 
\\ \\ 
     Bueno sin más vueltas al asunto enunciemos formalmente y demostremos el teorema expuesto al principio de este blog \\ \\
\textbf{Teorema}: si a,b,c,d están en $\mathbb R$, con $b \not = 0$ y $d \not = 0$ entonces $\frac{a}{b}+\frac{c}{d}=\frac{a \cdot d + b \cdot c}{b \cdot d}$
\\ \\
\textbf{Demostración} es claro que $b \cdot d \not = 0$ puesto que $b \not = 0$ y $d \not = 0$.
\\ \\ 
Vamos a comenzar del lado izquierdo de la ecuación y trataremos de llegar al lado derecho es decir de $\frac{a}{b}+\frac{c}{d}$
llegaremos a $\frac{a \cdot d + b \cdot c}{b \cdot d}$
\\ \\
Tenemos que $\frac{a}{b}+\frac{c}{d}= \frac{a}{b}+\frac{c}{d} \cdot 1$ esto por el axioma 4.
\\ \\
     Y tenemos que $1 = b \cdot d \cdot \frac{1}{b \cdot d}$ esto por axioma 6. Por ende reemplazando el 1 en el paso anterior tenemos
$\frac{a}{b}+\frac{c}{d}= \frac{a}{b}+\frac{c}{d} \cdot (b \cdot d \cdot \frac{1}{b \cdot d}) $ 
\\ \\ 
Ahora por el axioma 2 podemos hacer $(\frac{a}{b}+\frac{c}{d} \cdot b \cdot d )\cdot \frac{1}{b \cdot d}$
\\ \\ 
Por el axioma 3 podemos $(\frac{a}{b}\cdot b \cdot d + \frac{c}{d} \cdot b \cdot d) \cdot \frac{1}{b \cdot d}$
\\ \\ Por lo tanto tenemos que $(a \cdot d + c \cdot b) \cdot \frac{1}{b \cdot d}$ y listo simplemente organizando tenemos que $\frac{(a \cdot d + c \cdot b)}{b \cdot d} $
     y concluimos que $\frac{a}{b}+\frac{c}{d}=\frac{a \cdot d + b \cdot c}{b \cdot d}$ que es lo que queríamos demostrar.
\\ \\ 
\begin{center}
    \textbf{Conclusiónes}
\end{center}
     \begin{enumerate}
     \item Todo lo que sea verdad en matemáticas o bien es un axioma o se demuestra a partir de axiomas.
     Podríamos decir que un axioma es como una semilla de un árbol, es algo pequeño pero a partir de ella nace la teoría que conocemos. 
     \item Si piensas que las matemáticas no son para ti o que no te gustan o que simplemente no se te dan, tal vez sea la forma como te enseñaron
     las matemáticas. Las matemáticas pueden ser mucho más didácticas y divertidas de lo que pueden aparentar a simple vista, solo basta con mirarlas diferente
 \end{enumerate}
Por último quiero dejarte dos retos el primero es que demuestres el lema utilizado en la Demostración y el otro es que nunca pares de aprender. 
Si tienes dudas escríbeme un comentario en este blog o escríbeme a mí Twitter @WilsonJerez estaré feliz de resolver tus dudas.

\end{document}