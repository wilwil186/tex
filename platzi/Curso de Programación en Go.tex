\documentclass{article}
\usepackage[utf8]{inputenc}
\usepackage[spanish]{babel}
\usepackage{graphicx}
\usepackage{geometry}
\usepackage{multicol}
\usepackage{amsmath}
\usepackage{amsthm,amsfonts,amssymb}%paquetes AMS
\usepackage{hyperref}

\begin{document}
    \section{Introducción al Curso de Golang}
    aprender  termianr y linea de comandos y git y github antes de comenzar
    \section{¿Qué es, por qué y quienes utilizan Go?}
    \begin{enumerate}
        \item compilado y estáticamnete tipado(indicar el tipo de varibale o el tipo de constante para guardar un valor en el). 
        \item Creado en Google por Robert
        Griesemer, Rob Pike y Ken Thompson.
        \item Anunciado en Noviembre 2009.
        \item Primera versión en Marzo 2012.
        \item Go/Golang.
        \item Programadores: Gophers.
    \end{enumerate}
    con la potencia de c pero un lenguaje amiga como la de python
    GO gopher es nuestra mascota pero no es la oficial.
    GO ícono(2018). 
    \textbf{¿por que Go?}
    \begin{enumerate}
        \item Gran velocidad de compilación.
        \item Alto rendimiento para tareas pesadas.
        \item Soporte nativo para concurrencia.
        \item Top 5 más amados y mejores pagados
        en encuesta Stackoverflow 2020 (\$ 74 k)
        \item Obliga a implementar buenas prácticas.
        \item Comunidad muy receptiva.
    \end{enumerate}
    \textbf{¿Qui'enes usan Go?}
    \begin{enumerate}
        \item MercadoLibre: 70.000 con 20 MB RAM
        \item Twitch: Usuarios concurrentemente
        \item Twitter: Procesar analítica de la App con Kafka
        \item Uber: Posición conductores y pasajeros
        \item Docker y Kubernetes: Para despliegue de Apps
    \end{enumerate}
    \section{Instalar Go en Linux}
    ir a \url{https://golang.org/dl/}. personalmente recomiendo este video.
    \url{https://www.youtube.com/watch?v=E1yiCT2Rdj8}
    \section{Nuestras primeras líneas de código en Go}
    ecuerden que pueden revisar todo el código del curso en el repositorio de 
    \url{https://github.com/osmandi/curso_golang_platzi}
    

\end{document}