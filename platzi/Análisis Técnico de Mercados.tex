\documentclass{article}
\usepackage[utf8]{inputenc}
\usepackage{hyperref}
\usepackage{graphicx}
\usepackage{geometry}
\usepackage{multicol}
\usepackage{amsmath}
\usepackage{amsthm,amsfonts,amssymb}%paquetes AMS
\title{Curso de Introducción al Análisis Técnico de Mercados}
\author{wejerezh }
\date{October 2021}

\begin{document}

%\maketitle
\pagestyle{empty}
%\afterpage{\blankpage}
\begin{center}
\begin{figure}[h]
\centering


\end{figure}
\Large
\hrule
\vspace{4mm}
\textbf{Notas del Curso de Introducción al Análisis Técnico de Mercados}\\

\vspace{4mm}
\hrule
\large
\vfill
Autor\\

Wilson Eduardo Jerez Hernández \\
\vfill
Profesora\\
Angela Ocando
\vfill
Platzi\\
Escuela de Blockchain y Criptomonedas\\
Curso de Introducción al Análisis Técnico de Mercados\\
\end{center}
\newpage


\newpage
\addtocontents{toc}{\hfill \textbf{Página} \par}
\tableofcontents
\newpage


\section{Qué es el análisis técnico y su filosofía}
    \textbf{¿Qué es el análisis técnico?}
Es el estudio de los movimientos de precio de un activo financiero con el objetivo de poder hacer proyecciones a partir de gráficos y, de esa forma, poder tomar mejores decisiones de inversión.

Con movimientos de precio me refiero a la historia o recorrido de un activo financiero como: una divisa, el oro, la plata, algún stock, acción de empresa e incluso bitcoin o criptomonedas.

A partir de estos movimientos evaluar y encontrar patrones que nos permitan hacer una proyección del precio en el futuro.

El análisis técnico, vale decir, no es una herramienta infalible. Es decir: esto no te permitirá predecir el 100% de los movimientos del precio. Nadie puede con precisión adelantarse a esto, es imposible.

Sin embargo, sí que el conjunto de herramientas tanto técnicas como aquellas que serán parte de tu estrategia pueden permitirte ser rentable en el tiempo y llevar mayores proyecciones a favor que las negativas. Es mejor aclarar que la forma de alcanzar resultados positivos son la consecuencia de estudio, esfuerzo, paciencia, prueba, error y un conjunto de actividades que te ayudarán a seguir el camino correcto para alcanzar tus metas.



\end{document}
