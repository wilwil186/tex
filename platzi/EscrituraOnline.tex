\documentclass{article}
\usepackage[utf8]{inputenc}
\usepackage{hyperref}
\usepackage{graphicx}
\usepackage{geometry}
\usepackage{multicol}
\usepackage{amsmath}
\usepackage{amsthm,amsfonts,amssymb}%paquetes AMS

\begin{document}
    \section{Escribir no es un arte, es práctica}
    No escribas mucho, Sé constante.
    Reto: En 30 días, No pares de escribir #EScrituraPlatzi. 
    \section{¿De qué debería hablar?}
    ¿Que te apasiona? Piensa qué quieres lograr. 
    \section{Cambio de dirección: piensa primero en el lector}
    Empieza con empatía. Continúa con la utilidad. Mejora con el análisis. Optimiza con amor.
    Escribe sólo para una persona y actúa como si la quisiera muchísimo. \\ 
    \textbf{Cambia de lugar con el lector}
    \begin{itemize}
    \item ¿Qué está sintiendo mi lector?
    \item ¿Qué preguntas le quedan después de leer el texto?
    \item ¿Estoy haciendo trabajar demasiado a mi lector para que entienda lo que quiero decir?
    \end{itemize}
    Remplaza el "yo" y "nosotros" por el "tú" y "ustedes"
    \section{Primero las bases: la palabra}
    Las palabras son imágenes. 
\end{document}