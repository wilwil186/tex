\documentclass{article}
\usepackage[utf8]{inputenc}
\usepackage[spanish]{babel}
\usepackage{graphicx}
\usepackage{geometry}
\usepackage{multicol}
\usepackage{amsmath}
\usepackage{amsthm,amsfonts,amssymb}%paquetes AMS
\usepackage{hyperref}


\begin{document}
    \section{Clase 1}
    \subsection{Objetivos}
    \begin{enumerate}
        \item Identificar el lugar óptimo para comenzar a practicar tus habilidades de ciencia de 
        datos. 
        \item Aprenderás a utilizar diferentes tipos de Jupyter Notebooks. 
        \item Instalar herramientas elemntales para programar para ciencia de datos. 
        \item Dominarás el manejo de ambientes virtuales con CONDA
    \end{enumerate}
    \subsection{¿En que lugar programar?}
    \begin{enumerate}
        \item Navegador
        \item local 
        \item servidor de tu empresa
        \item celular
    \end{enumerate}
    puedes trabajar en cualquier SO,no obstante a la people le gusta un SO basado en linux o unix, pero todo bien 
    si usas windows puedes usar wsl. \\
    hay muchas herramientas, ventajas y desaventajas
    \subsection{Notebooks}
    anteriormente era común crear un archivo fuente ejecutar y tales, pero en el 2001 el fisico colombiano fernando perez 
    creo Ipython una herramienta que mejoraba el Relp(lectura evaluacion impresion bucle) de python, y eso dio origuen a Jupyter Notebooks
    \subsection{Notebooks vs Script}
    \begin{enumerate}
        \item ambos son utiles ventajas y desventajas. 
        \item Organización. 
        \item Experimento y prototipado. 
        \item objetivo. 
    \end{enumerate}
    \section{Clase 2}
    \subsection{Google colab:primeros pasos}
    Sistema basado en Notebooks. 
    \subsection{Notebooks en la nube vs. locales}
    \begin{enumerate}
        \item Ambos son utiles
        \item Configuración de entorno. 
        \item Tiempo de ejecución. 
        \item Escalabilidad. 
    \end{enumerate}
    \\ \textbf{¿Qué es Google Colab?} \\ 
    \begin{enumerate}
        \item Servicio en la nube 
        \item Basado en Jupyter Notebooks
        \item No requiere configuración. 
        \item Trabajo a nivel de archivo. 
        \item uso gratuito de GPUs y TPUs. 
    \end{enumerate}
    
\end{document}