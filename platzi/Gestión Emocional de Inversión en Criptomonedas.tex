\documentclass{article}
\usepackage[utf8]{inputenc}
\usepackage[spanish]{babel}
\usepackage{graphicx}
\usepackage{geometry}
\usepackage{multicol}
\usepackage{amsmath}
\usepackage{amsthm,amsfonts,amssymb}%paquetes AMS
\usepackage{hyperref}

\begin{document}
    \pagestyle{empty}
%\afterpage{\blankpage}
\begin{center}
\begin{figure}[h]
\centering


\end{figure}
\Large
\hrule
\vspace{4mm}
\textbf{Notas del Audiocurso de Gestión Emocional de Inversión en Criptomonedas }\\

\vspace{4mm}
\hrule
\large
\vfill
Autor\\

Wilson Eduardo Jerez Hernández \\
\vfill
Profesora\\


Angela Ocando
\vfill
Platzi\\
Escuela de Blockchain y Criptomonedas\\
Audiocurso de Gestión Emocional de Inversión en Criptomonedas\\
\end{center}
\newpage


\addtocontents{toc}{\hfill \textbf{Página} \par}
\tableofcontents
\newpage
    \section{El secreto de invertir en criptomonedas}
    da miedo no? XD. las emociones tienen que ver mucho en tus emociones.
    \section{Pilar fundamental de las inversiones}
    el 90\% pierde el 90\% de su dinero en los primeros 90 dias de aprendizajes.
    los mercados financieros son movidos por las personas.
    la economia conductual. es la disiplina que estudia e interviene las conductas relacionadas con el dinero.
    para ser inversores necesitamos \begin{enumerate}
        \item tiempo de estudio
        \item metodologia
        \item gestion financiera
        \item gestión Emocional
    \end{enumerate}
    \section{El principal problema de las inversiones}
    las emociones son muy impotantes.\begin{enumerate}
        \item miedo. miedo a perder. en este ecosistema hay que controlar este sentimiento. una forma de controlarlo es tener una metodologia.
        \item avaricia. el interes por generar beneficios inmediatos. esta mentalidad es equivocada.
        \item esperezanta. intentar cambiar los resultados. no hay forma de predecir el 100\% de los movimientos.
        \item desconocimineto. aprender una o dos herramientas no seran suficientes. sin embargo es un gran comienzo,aprender de cada fallo.siempre porteger nuestro capital.
        
    \end{enumerate}

    \section{El impacto de las creencias limitantes en las inversiones}
    hay solo una unica diferencia que tiene exito en la vida y la gente que fracasa, y esta en su forma de pensar.
    analiza que hace parte de tus creencias que te ayudaran a crecer y continuar avanzando y elemina aquellas que te limiten.
    4 creencias \begin{enumerate}
        \item el dinero es malo.
        \item el dinero no crece en los arboles. 
        \item se necesita dinero para ganar dinero.
        \item el dinero es importante. tener dinero es fundamental siempre y cuando se sepa que el dinero es una herramienta.
    \end{enumerate}
    ahorrar y invertir son dos cosas muy diferentes, si en el presente solo nos concentramos ahorrar. 
    en crisis tienes dos opciones llorar o vender pañuelos.
    video recomendado: \url{https://www.youtube.com/watch?v=csnhsE_jwBs&amp;t}
    (\begin{center}
        todos nos morimos, y luego no pasa nada, ten metalmente cuanto de vida que tienes, el tiempo es el unico recurso no renovable.
        el dinero realmente no existe, tu tiempo lo puedes conbverir en dinero.
        ahorrar dinero no es mas importante que ahorrar tiempo, "lo que es de uno es de uno", tu deberias ser responsable de ti mismo.
        "busque seguridad mijo" "la inversión en vivienda siempre funciona" lo mas importante que nosotros tenemos es el tiempo.
        en ocociones si ustedes pagan mas por algo pero a cambio obtienen mas tiempo ganan. ROI (return of investment).
        deberia comprar este lugo o comprar esta herramienta.hay una forma de aprender mejor que platzi es leer.
        viajar se trata de aprender.no hay nada de malo en inverit en tu diversión pero ten mucho cuidado con la \textbf{gratificación inmediata}
    \end{center})
    lectura:
    \url{https://www.misfinanzasparainvertir.com/4-creencias-que-limitan-su-exito-financiero/}
    \section{Errores psicológicos en ínversiones}
    el ser humano no esta preparado para perder. pero tampoco para ganar.
    \textbf{lowertrading} operar en exceso. para controlar nuestras emociones debemos obligarnos a ser disiplinados.
    no saber perder. liquidar una mala posición con daños menores.
    \section{Desarrolla tu Inteligencia Emocional}
    la inteligencia Emocional es la capicidad de reconocer, interpretar gestionar y aplicar sus emociones y sentimientos, para optimizar su racionamineto
    y solucion de problemas.
    pasos esenciales 
    \begin{enumerate}
        \item cuestionate. 
        \item acepta y regula. las emociones no pueden desactivarse. 
        \item crea una estragia de inversion. 
    \end{enumerate}
    \section{Organiza tu mente}
    el capital de inversion es la herramienta principal de un inversor. 
    por eso es importante tener un plan y gestión del riesgo. preservar y cuidar nuestro capital.
    piensa siempre en resultado a largo plazo. algunas recomendaciones.
    \begin{enumerate}
        \item no arriesges mas de \% 1 de tu capital de inversion por transación
        \item no expongas tu capital de inversión a mas de \% 3 de riesgo diario.
        \item siempre un stoploss.
        \item siempre busca una proyeción de superior a 3 veces la perdida de tu operación.
        \item diario de trading.
        \item opera de forma objetiva.
        \item nadie es inmune a la perdidas.
        \item siempre sigue tu plan e instrucciones.
        \item no permistas la avaria y el exceso de confianza.
    \end{enumerate}
    \section{Lleva un diario de trading}
    aprender de tus errores y maximizar tus aciertos. 
    registrar cada una de nuestras operaciones, obejetivos,estrategias, emociones. entre otro.
    un diario de trading es facil pero muy efectivo.
    \section{lo deberias registrar en este diario son:}
    \begin{enumerate}
        \item la fecha de entra
        \item la fecha de salida
        \item simbolo
        \item precio de entrada
        \item tamaño de la posición
        \item stop loss
        \item take profit
        \item precio de salida
        \item ganacias o perdidas
        \item porcentanje de perdida o ganancia respecto a tu capital de entrada
        \item que viste en el precio para entrar
        \item grafico 
        \item motivo de salida 
        \item emociones
        \item notas 
    \end{enumerate}
    organizar por periodos es importante. medir con precisión tus exitos y tus fracasos.
    \section{}
    bitcoin se convirtió en una reserva de valor, para inversionistas pacientes, dispuestos a enfrentar grandes cambios en la cotización del precio.

*no inviertas mas de lo que estas dispuesto a perder*

recuerda que los retrocesos son parte de los ciclos de mercado.por ende no entres en pánico.
sigue estos consejos:
1. entender como funciona el ecosistema, nunca invierta en un negocio que no pueda entender
2. define tu limite de perdida, imagina en peor escenario.
3. no te dejes llevar por emociones. en caídas abruptas  hay 3 opciones:
	a)  mantener tu posición
	b) vender el pánico y perder dinero
	c) comprar.
sea cual sea siempre sigue tu estrategia  y tu propio estudio 
4. diversifica tu portafolio. divide y vencerás
5. herramientas de análisis de cripto. como https://es.tradingview.com/
https://coinstats.app/
6. no subestimes el poder de largo plazo pero no dejes nada a la suerte.
nunca pares de aprender XD aprende de tus errores.
    \section{}
    sigue el camino correcto.
no por controlar nuestras emociones tenemos que dejar de aprender de este ecosistema. aprende, aprende y sigue aprendiendo. nunca pares de aprender. sin prisa pero sin pausa.
    

\end{document}