\documentclass{article}
\usepackage[utf8]{inputenc}
\usepackage[spanish]{babel}
\usepackage{graphicx}
\usepackage{geometry}
\usepackage{multicol}
\usepackage{amsmath}
\usepackage{amsthm,amsfonts,amssymb}%paquetes AMS
\usepackage{hyperref}


\begin{document}


    \pagestyle{empty}
%\afterpage{\blankpage}
\begin{center}
\begin{figure}[h]
\centering


\end{figure}
\Large
\hrule
\vspace{4mm}
\textbf{Notas del Clases del Curso Gratis de Marca Personal}\\

\vspace{4mm}
\hrule
\large
\vfill
Autor\\

Wilson Eduardo Jerez Hernández \\
\vfill
Profesor\\

Freddy Vega
\vfill
Platzi\\
Escuela de Marketing Digital\\
Curso Gratis de Marca Personal\\
\end{center}
\newpage

    \newpage
    \addtocontents{toc}{\hfill \textbf{Página} \par}
    \tableofcontents
    \newpage

    \section{Cómo escribir contenido que genere valor}   
    Leer y escribir son dos actividades que te van a ayudar a seguir creciendo.

    El primer párrafo tiene que ser un gancho poderoso, pues es el segmento que va a aparecer cuando compartas tu contenido en redes sociales. Intenta también empezando con una idea fuerte, por ejemplo eliminar una noción o un mito. Esto de entrada hace interesante lo que vas a decir a continuación.

    No le tengas miedo al punto seguido. Este hace más fácil de leer y separar las ideas. Haz párrafos cortos, resalta algunas frases y utiliza imágenes para tener una mejor estructura.

    Los buenos escritores siempre quieren que el lector haga algo, así solo sea cambiar una perspectiva. Intenta incluir un CTA (call to action) al final de tus artículos.

    \section{Cómo comprar un dominio}
        Tener un dominio es importante para fortalecer tu marca personal. Ahí puedes alojar tu blog, portafolio o cualquier cosa con la que decidas empezar. Intenta buscar la extensión .com sobre todas las demás para evitar limitar tu audiencia.

        \textbf{Otras recomendaciones:}

        \begin{enumerate}
            \item Elige un nombre que sea fácil de escribir y de recordar
            \item Utiliza un servicio como namecheap.com para adquirir tu dominio
            \item Compra tu dominio por más de un año para garantizar un mejor posicionamiento en los resultados de búsqueda de Google
        \end{enumerate}
    
    \section{Uso de Google Suites y mail corporativo}
        Google tiene un servicio llamado Google Suite con el que puedes configurar tu dominio y obtener cuentas de correo, calendario y documentos compartidos.
        \textbf{las cosas buenas hay que pagarlas} como platzi.
    
    \section{Tips para escribir bien}

    Para escribir un buen artículo en internet necesitas cumplir con 5 condiciones:
    \begin{enumerate}
        \item Responder una pregunta.
        \item Tener un buen título.
        \item Facilitar que te encuentren.
        \item Evitar que se vayan.
        \item Tener buena ortografía y redacción.
    \end{enumerate}
    Te voy a dar una serie de tips para escribir un buen texto, según cada uno de los cinco requerimientos de la lista.
    \subsection{Responder una pregunta y Tener un buen título}
    No es necesario tener el título definitivo antes de empezar a escribir. Esto puedes pulirlo más adelante con los tips de la tercera condición “Facilita que te encuentren”. Lo que si tienes al principio es una pregunta o una cuestión que quieres resolver ¿verdad? Quizás vas a explicar algo, vas a enseñar un paso a paso, vas a contar algo que sucedió o incluso a dar tu opinión. Todas estas opciones tienen una pregunta implícita que debería estar planteada de algún modo en el título de tu artículo.

    \textbf{algunos tips:}
    \begin{enumerate}
        \item Responde la pregunta rápido. Ya habrá espacio para profundizar más adelante.
        \item Engancha a tu lector con algo llamativo. Las imágenes funcionan muy bien.
    \end{enumerate}
    \subsection{Malos títulos}
    “Delicia mexicana para acompañar tus platos”
    ¿Qué pregunta responde esto? Además, empieza a pensar en palabras clave e imagina qué otras cosas puedes encontrar en Google si buscas “delicia mexicana”.
    “Guacamole”
    Esto tampoco responde a una pregunta. No es necesariamente un mal título, pero si aprendes sobre volúmenes de búsqueda de keywords verás que si eres más específico puedes ser más acertado.
    \url{https://platzi.com/blog/keyword-research/}
    \subsection{Buenos títulos}
    “Receta de guacamole”
    Este probablemente es el mejor, pues puede plantear preguntas comunes como ¿cómo se prepara? ¿cuáles son los ingredientes? ¿qué es el guacamole?
    “Cómo preparar guacamole”
    Este también funciona, pues es muy probable que alguien busque exactamente con esas palabras. Lee aquí sobre las answer boxes de Google \url{https://platzi.com/blog/answer-boxes/} y por qué esto es importante.
    “Receta fácil y rápida de guacamole”, “El guacamole perfecto”, “Guacamole autentico mexicano”
    Puedes ser un poco más creativo y apelar a la emoción con adjetivos. Lo importante es que consideres si realmente aportan.
    Un buen título y responder la pregunta del usuario rápidamente son las dos condiciones que te ayudarán a tener un porcentaje de rebote bajo. Si quieres profundizar en este tema te recomiendo aprender sobre Google Analytics.
    \url{https://platzi.com/cursos/google-analytics/}
    \subsection{Facilita que te encuentren y Evita que se vayan: Guía de SEO express para escritores}
    Crea un increíble primer párrafo. Esto no solo facilita que te encuentren sino que evita que se vayan, pero ya llegaremos a ese punto.
    \subsection{Cómo mejorar la ortografía y la redacción}
    “La mala ortografía es como el mal aliento”. Es por eso que esta debería ser la primera condición, pero la dejé al final para llenarte de tips para mejorar:
    \begin{enumerate}
        \item No escribas como hablas, la mayoría de palabras sobran. No tengas miedo a hacer afirmaciones propias. Un ejemplo común es cuando empezamos frases así: “Todos sabemos que”. Lo que sigue después de esa frase es oro, y es lo que realmente quieres decir.
        \item No le temas al punto seguido. Escribe párrafos cortos y separa bien tus ideas. Si no tienes un punto seguido en un párrafo lo más probable es que debas reemplazar una de esas comas a las que tanto recurres.
        \item Lee en voz alta lo que escribiste. Si hiciste una pausa y no hay una coma o un punto algo anda mal.
        \item Dale estructura a tu texto. Utiliza subtítulos (que respondan otras preguntas relacionadas), negritas para resaltar cosas importantes y listas para mostrar un paso a paso o aspectos relevantes.
        \item Pide ayuda. Dile a un amigo o familiar que lea lo que escribiste antes de publicar. Seguro conoces a alguien que te gusta cómo escribe y que te puede señalar mejoras.
        \item Lee mucho. La mejor forma de mejorar tu forma de escribir es leyendo.
    \end{enumerate}
    \section{Por qué tener tu marca personal en YouTube}
    La mayoría del contenido que se consume en internet son videos, por eso es importante incluirlos dentro de tu marca personal. No hace falta tener demasiados recursos para producir video, en principio todo lo que necesitas está en tu smartphone. Si quieres hacer videos increíbles te recomiendo el Curso de Edición de Video con Premiere PRO, el de After Effects y el de Producción de Video para YouTube.
    Enseñar a crear un canal de YouTube, paso a paso.
    \section{Crea tu canal de YouTube}
    crear una cuenta de google.
    \url{https://platzi.com/blog/cuantos-videos-hay-que-hacer-antes-de-ganar-dinero-en-youtube/}
    primer objetivo:\\
    \begin{enumerate}
        \item mas de 4000 horas de visualición oubliza y 1000 sub
        \item 1000 sub
        \item cuenta adsence
    \end{enumerate}

    \section{Tu portafolio es tu mejor currículum}
    \begin{enumerate}
        \item Define tu objetivo.
        \item Define el medio.
        \item Haz un esquema o prototipo. (bajaer a lapiz y papel lo que piensas)
        \item sobre tí. puedes grabarte
        \begin{enumerate}
            \item Trasmite quién eres. 
            \item Define tus mejores aptitudes profesionales.
            \item en algún espacio di algo de tu hobby(solo uno)
            \item cuanta cuáles son tus anhelos de crecimiento ¿en quién te quieres convertir?
        \end{enumerate}
        \item tus trabajos (no solo muestra tus trabajos), si tienes mucha experiensa, resumen con los mejores. numero ideal 4.
        si no tienes experiencia, sube los proyectos que tiene cada curso de platzi(documenta todo el proceso)
        \item Mantente actualizado. manten tu portafolio actualizado
    \end{enumerate}
    \section{Cómo llamar la atención de reclutadores}
    no tienes trabajo por que no saber pedir no envies impresiones a lo pendejo 
    \begin{enumerate}
        \item marca personal poderosa
        \item proyectos profesionales
        \item unir marca personal a tus proyectos
        \item investiga donde quieres trabajar
    \end{enumerate}
    ignora a los reclutadores, algunos valen la pena, funcionan como reclutadores, intenta conectar directamente con la empresa
    no tengas miedo en mostrar una propuesta, se ambisioso. pero humilde
    \section{Tips para entrevistas de trabajo}
    antes de la entrevista busca que es la empresa. y que realmente sea la que te cautiva.
    revisa siempre tu cv. \textbf{se honesto} aporvecha cada momento para motrar que eres el ideal.
    la actitud es importante. contacto visula permante pero no axficiante. grabarte antes de una entrevista.
     después. analiza. Sé crítico.
\end{document}
