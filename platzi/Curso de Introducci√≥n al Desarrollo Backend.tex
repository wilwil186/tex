\documentclass{article}
\usepackage[utf8]{inputenc}
\usepackage[spanish]{babel}
\usepackage{graphicx}
\usepackage{geometry}
\usepackage{multicol}
\usepackage{amsmath}
\usepackage{amsthm,amsfonts,amssymb}%paquetes AMS
\usepackage{hyperref}

\begin{document}

    \pagestyle{empty}
%\afterpage{\blankpage}
\begin{center}
\begin{figure}[h]
\centering


\end{figure}
\Large
\hrule
\vspace{4mm}
\textbf{Notas del Curso de Introducción al Desarrollo Backend}\\

\vspace{4mm}
\hrule
\large
\vfill
Autor\\

Wilson Eduardo Jerez Hernández \\
\vfill
Profesor\\

Facundo García Martoni
\vfill
Platzi\\
Desarrollo Backend con Go\\
Curso de Introducción al Desarrollo Backend\\
\end{center}
\newpage


\addtocontents{toc}{\hfill \textbf{Página} \par}
\tableofcontents
\newpage

    \section{Te doy la bienvenida al desarrollo backend}
    el mundo esta lleno de aplicaciones web. ellas nacen en la web. 
    \section{Yin y Yang de una aplicación: frontend y backend}
    la corroceria es el acara visible del auto, y otro componente impotarte es el motor.
    ahora piensa en aplicaciones. igual que los carros tienen lo mismos componente "relativamente".
    en tecnologia la correceria: es el frontend y el motor es el backend.
    python es un lenguaje para onstruir backend. las tecnologias principales del frontend son: 
    \begin{enumerate}
        \item html
        \item css
        \item JavaScript
    \end{enumerate}
    esos son los basicos. css estilos. JS interacción. html una estrcutura.
    cada lenguaje tiene sus derivados piensa como si fuera codigo estrito por demas personas.
    en backend se pueden utilizar lenguajes de progamación como:
    \begin{enumerate}
        \item JavaScript
        \item php 
        \item Java
        \item Go
        \item Rust
        \item Ruby
        \item python
    \end{enumerate}
    y seguramente hay muchos mas. y cada uno de estos lenguajes tienen sus derivados.
    \section{Framework vs. librería}
    ¿como se construye un auto? comenzamos por el motor, tenemos que empezar por un lado.
    seguramente hay una guia para hacer el motor, a esto en progamación se le llama libreria.
    las reglas para contruiir el motor, se le llama Framework, en español un marco de trabajo.
    el Framework usualmente es un conjunto de librerias de paosos y recetas para contruir un motor.
    \section{Cómo se conecta el frontend con el backend: API y JSON}
    API Y JASON.
    una API es solo una sección de motor que permite que el frontend se comunique con el backend.
    dos grandes estadares son \textbf{SOAP} o simple obeject access protocol, y \textbf{REST} o representational state transfer.
    este primer estandar mueve la informacion con \textbf{XML} extensible markup languague parecido a html.
    soap a quedado un poco en el olvido :'(. 
    por culpa de JASON por ser superior. 
    un JASON es como un diccionario de python.
    los diccionario en python son iguales a los objetos en JS.
    \section{El lenguaje que habla Internet: HTTP}

    \url{https://developer.mozilla.org/es/docs/Web/HTTP/Status}

    \textbf{cliente}: dispositivos. \textbf{servidor}: una computadora del dia encendida 24/7
    protocolo de tranferencia de hipertexto. el cliente hace la peticion en idioma HTTP.
    200 es ok. dentro del rango de los 400 es que algun recurso no es encontrado. 
    si esta en el ranfo de los 500 hay un error en el codigo.
    \textbf{tarea investicar los estatus code}
    \section{¿Cómo es el flujo de desarrollo de una aplicación web?}
    el lugar en el que comienzas a trabajar es tu editor de texto. git sistema de control de versiones.
    browser (navegador). 
    en el servidor se coloca una aplicación para que este disponible para todos.
    esto se denomida deploy(pasar de el entorno de trabajo a un resposorio). normalmente se hace un push a github antes.
    desde github se hace \textbf{CI/CD}. este porceso lo que hace es testarlo. y si todo esta ok se envia a el servidor.
    el servidor guarda el codigo es un servidor. normalmente son de paga pero se pueden conseguir algunas de manera gratuita.
    la nube es una forma popular de llamar a muchos servidores.
    \section{El hogar de tu código: el servidor}
    los servidores estan en datacenters.al hecho de guardar una aplicacion
    en un servidor se le denomina hosting. hay distintos tipos de hosting. entendamos 3 conceptos
    \begin{enumerate}
        \item iass (infrastutura como servivio). es una opción cuando quiereas tener control como la cpu la ram y la ssd, esta \begin{enumerate}
            \item Aws (amazon web services) 
            \item microsoft azure
            \item digital ocean (economico)
        \end{enumerate}
        vas a encontrar dos tipos de iass en internet 
        \begin{enumerate}
            \item VPS(virtual private server). para rendimiento superior.
            \item shared hosting. alojamiento compartido. es mas barato.
        \end{enumerate}
        \item pass (plataforma como servicio).se encarga de actualizar, tu rela. 
        just deploy. opciones: \begin{enumerate}
            \item google aee engine. servicio de google cload
            \item firebase. un poco mas complejas.
            \item heroku. 
        \end{enumerate}
        \item sass (software como servico). no reinventar la rueda. no code. 
        es una aplicacion que un provedor te presta para que puedas hacer funcionar tu servicio.
        \begin{enumerate}
            \item google docs.
            \item slack.
            \item wordpress.
        \end{enumerate}
        
    \end{enumerate}    
    hay que elegir una de ellas cuando se publica tu aplicacion

    \section{Proyecto: diseño y bosquejo de una API}
    
    cosas fundamentales twitter, por ejemplo un twit, si fuéramos a crear un clon de twiter, tengo que ser capaz de crear usuarios de publicar twits de actualizar usuarios y actualizar twits, de borrar un usuario y de borrar un twits ademas ver a usuarios y ver twits 
    \textbf{API} aplicacion, program, interface. 
    una api se construye de varias maneras 
    por ejemplo en python \begin{enumerate}
        \item fastapi
        \item Django
        \item flask
    \end{enumerate}
    \textbf{endpoint/route/path} es simplemente una section de la url de tu Proyecto.
    normalmente seria http://twiteer.com/api/... lo que esta despues de dominio seria el endpoint

\end{document}