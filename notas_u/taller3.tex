\documentclass[12pt]{book}
\usepackage[utf8]{inputenc}
\usepackage[spanish]{babel}
\usepackage{hyperref}
\usepackage{graphicx}
\usepackage{geometry}
\usepackage{multicol}
\usepackage{amsmath}
\usepackage{multicol}
\usepackage{color}
\usepackage{bbding}

\begin{document}
    \begin{enumerate}
        \item Realice un código que imprima las siguientes fórmulas
            \begin{equation*}
                (a+b)^{n} = \sum_{k = 0}^{n}\binom{n}{k}a^{k}n^{n-k}
            \end{equation*}
            \begin{equation*}
                P(A \cap B \cap C ) = P(A)+P(B)+P(C)-P(A \cup B)-P(B \cup C )+P(A\cup B\cup C)
            \end{equation*}
            \begin{equation*}
                \oint_C f \cdot dr = \iint_{D}(\frac{\partial Q}{\partial x}-\frac{\partial P}{\partial y})dA
            \end{equation*}
            \begin{equation*}
                1+\cfrac{1}{\sqrt{2}+\cfrac{1}{\sqrt{2}+\cfrac{1}{2 + \cdots}}}
            \end{equation*}
            \begin{equation*}
                \sum_{n = 0}^{\infty}(\prod_{k = 0}^{n}\frac{2k+n-1}{3k+n-2})\frac{(x-3)^{n}}{n!}
            \end{equation*}
            \begin{equation*}
                \sum_{-\infty}^{\infty}(a_{n}\sin x +b_{n} \cos x)
            \end{equation*}
            \begin{equation*}
                \ln(\prod_{k=1}^{n}\arccos (\frac{k}{pi})) = \sum_{k=1}^{n}\ln(\arccos(\frac{k}{\pi}))
            \end{equation*}
            \begin{equation*}
                X_{t} = X_{0} \exp(\mu t - \frac{\sigma^{2}}{2}W_{t})
            \end{equation*}
            \begin{equation*}
                V^{*} \approx \hom(V,F) \times \hom(F,V)
            \end{equation*}
            \begin{equation*}
                (\sum_{i=1}^{n}x_{i})^{2} = \sum_{k=1}^{n}x_{i}^{2}+2\sum_{i \not = j \\ i < j} x_{i}x_{j}
            \end{equation*}
        \item Diseñe un código que imprima cada una de las siguientes afirmaciones:
            \begin{itemize}
                \item Si $f(x)$ es una función continua en el intervalo $I=[a,,b]$ y $P={a = x_{0},x_{1},\cdots,x_{n-1},x_{n}}$ es una 
                partición de $I$, entonces
                $\int_{a}^{b} f(x)dx := \sum_{k=0}^{n-1}\int_{x_{i}-1}^{x_{i}}f(x)dx$
                \item Si ${A_{1},A_{2},\cdots}$ es uan familia enumerable infinita de conjuntos medibles disyuntos dos a dos, es decir, $A_{i} \cap A_{n} = \emptyset$ para cada $i \not j$, entonces 
                \begin{equation*}
                    P(\bigcup_{k=1}^{\infty}A_{k})=\sum_{k=1}^{\infty}P(A_{k})
                \end{equation*}
                \item Si $T$ es una trasformación lineal de un espacio vectorial V a un espacio vectorial $W$ de dimensión finita, $\dim_{F}(W) = n < \infty$, entonces 
                \begin{equation*}
                    P(T)+v(T)=n
                \end{equation*}
                \item  Si S es un sólido ene l espacio limitado por una superficie orientable $\sum$, sxi \textbf{n} es la normal unitaria exterior a $\sum$ y si \textbf{F} es un campo vectorial definido en V, entonces tenemos
                \begin{equation*}
                    \iiint_{V}(\mathrm{div} F) dV = \iint_{\sum} \textbf{F} \cdot \textbf{n} d\sum
                \end{equation*}
                \item Si $(x_{1},y_{1}),(x_{2},y_{2}),\cdots,(x_{n},y_{n})$ son $n$ puntos en el plano tal que $x_{1} \not = x_{j}$ para cada $i \not = j$, entonces existe un único polinomio de grado $n$
                \begin{equation*}
                    p(x) = a_{0}+a_{1}x+a_{2}x^{2}+\cdots+a_{n-1}x^{n-1}+a_{n}x^{n}
                \end{equation*}
                tal que $p(x_{i})=y_{i}$ para cada $i=1,2,3,\cdots,n$.
            \end{itemize}
    \end{enumerate}
\end{document}