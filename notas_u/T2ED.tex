\documentclass{book}
\usepackage[utf8]{inputenc}
\usepackage[spanish]{babel}
\usepackage{hyperref}
\usepackage{graphicx}
\usepackage{geometry}
\usepackage{multicol}
\usepackage{amsmath}
\usepackage{amsthm,amsfonts,amssymb}%paquetes AMS

\begin{document}
    \begin{enumerate}
        \item Suponga que desea modelar una población con 
        una ecuación diferencial de la forma $\frac{dP}{dt}=f(P)$,
        donde $P(t)$ es la población en eel tiempo $t$.\\ 
        Los experimentos han sido realizadas sobre la población 
        que dan la sigueinte información:
        \begin{itemize}
            \item Los únicos puntos de equilibrio son P = 0,
            P = 10 y P = 50. 
            \item Si la población es 100, la población disminuye. 
            \item Si la población es de 25, la población aumenta. 
        \end{itemize}
        \begin{itemize}
            \item[(a)] Dibuje las líneas de fase posibles para este sistema
            para $P>0$ (hay dos)
            \item[(b)] Dé un bosquejo aproximado de las funciones correspondientes
            $f(p)$ para cada uno de sus lineas de fase.
            \item[(c)] Dé una fórmula para las funciones $f(p)$ cuya gráfica 
            concuerde(cualitativamente) con los bosquejos aproximados
            de la parte (b) para cada una de sus líneas de fase.  
        \end{itemize}
        \item Un tanque de 400 galones inicialmente 
        contiene 200 galones de agua que contienen 2 
        partes por mil millones en peso de dioxina, 
        un carcinógeno extremadamente potente. 
        Supongamos que el agua contiene
        5 partes por mil millones de dioxina fluyen 
        hacia la parte superior del tanque a una 
        velocidad de 4 galones por minuto. 
        El agua del tanque se mantiene bien 
        mezclada y se extraen 2 galones por minuto del 
        fondo del tanque. ¿Cuánta dioxina hay en el 
        tanque cuando está lleno?
        \item Sean $x_{1}=x_{1}(t)$ y $x_{2}=x_{2}(t)$ 
        las soluciones de la ecuación de Bessel
        \begin{center}
            $2t^{2}x^{''}+tx^{'}+(t^{2}-n^{2})x=0$, $n>0$ constante,
        \end{center}
        definidas sobre el intervalo $0<t<\infty$, y que satisfacen las condiciones 
        $x_{1}(1)$, $x_{1}^{'}(1)=0$ , $x_{2}(1)=0$, $x_{2}^{'}=1$, Demuestre 
        que $x_{1}(t)$ y $x_{2}(t)$ forman un conjunto fundamental de soluciones en 
        $(o, \infty)$.
    \end{enumerate}
\end{document}