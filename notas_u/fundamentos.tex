\documentclass{book}

\usepackage[reqno]{amsmath} % Paquete para el manejo de expresiones matemáticas [reqno], [leqno] y [fleqn]
\usepackage{amsthm}
\usepackage{amssymb,amsmath,latexsym} %Paquete para llamar símbolos matemáticos
\usepackage{tabularx,booktabs,ragged2e}
\usepackage{amsfonts}
\usepackage[mathscr]{euscript}
\usepackage{graphicx} %paquete para el manejo de transformaciones geométricas de imagénes
\usepackage{color} %paquete para el manejo de color en textos.
%\usepackage[utf8]{inputenc} %paquete para el manejo de caracteres acentuados
\usepackage[french,spanish]{babel} %paquete que genera documentos en diferentes idiomas
\usepackage{enumerate}
\usepackage{multicol} % Paquete para modificar el número de columnas
\usepackage{layout} % Paquete  para revisar los valores de 
\usepackage{verbatim}
\usepackage[all]{xy} 
\usepackage{pgfplots}
%graficos
\usepackage{pgfplots}
\pgfplotsset{compat=1.15}
\usepackage{mathrsfs}
\usetikzlibrary{arrows}
\pagestyle{empty}
%graficos
\pagestyle{myheadings}  % Estilo de página
%\pagenumbering{arabic} % Estilo de numéración
\hoffset1cm
\newcounter{Teorema}
\newcommand{\Teorema}{\stepcounter{Teorema}{\bf Teorema \theTeorema.} }
\DeclareMathOperator{\arcsec}{arcsec} % Creación de nuevos comandos en latex
\DeclareMathOperator{\Var}{Var}
\DeclareMathOperator*{\Hom}{Hom}
\setcounter{MaxMatrixCols}{15}
\allowdisplaybreaks % Control de cambios de página en alineaciones
\decimalpoint
%\renewcommand{\theequation}{\thesection.\arabic{equation}}
\numberwithin{equation}{section}
%\renewcommand{\theequation}{\theparentequation\arabic{equation}}
%% Nuevos teoremas
\theoremstyle{plain}  % Requiere el paquete amsthm
\newtheorem{thm}{Teorema}[section]
\newtheorem*{Dem}{Demostración}
\newtheorem{Corol}[thm]{Colorario}
\newtheorem{Prop}[thm]{Proposición}
\newtheorem{axiom[thm]}{Axioma}
\newtheorem{conj}{Conjetura}
\newtheorem{Def}{Definición}[section]
\newtheorem{Ej}{Ejemplo}[section]
\newtheorem{notacion}[Def]{Notación}
\newtheorem{nota}[Def]{Nota}
\renewcommand{\qedsymbol}{$\heartsuit$}
\providecommand{\abs}[1]{\lvert#1\rvert} %valor absoluto
\providecommand{\norm}[1]{\lVert#1\rVert} %norma
\usepackage{fancyhdr}
\pagestyle{empty}
\usepackage{pgfplots}
\usepackage{mathrsfs}
\usepackage{tcolorbox}
\usepackage{amsmath, amsthm, amssymb}
\usepackage{mathrsfs}
\usepackage{graphicx}
\usepackage{geometry}
\usepackage{amsmath}
\spanishdecimal{.}
\usepackage{color}
\usepackage{multicol}
\usepackage{bbding}
\usepackage{stackrel}
\usepackage{hyperref}
\usepackage{geometry}
\usepackage{multicol}
\usepackage{amsfonts}
\usepackage{multicol}
\usepackage{bbding}
%\usepackage[manejador]{color,graphicx}

%colores 
\definecolor{dorado}{cmyk}{0,0.10,0.84,0}
\definecolor{melon}{cmyk}{0,0.29,0.84,0}
\definecolor{naranja}{cmyk}{0,0.42,1,0}
\definecolor{durazno}{cmyk}{0,0.46,0.50,0}
\definecolor{fresa}{cmyk}{0,1,0.50,0}
\definecolor{ladrillo}{cmyk}{0,0.77,0.87,0}
\definecolor{violeta}{cmyk}{0.07,0.90,0,0.34}
\definecolor{purpura}{cmyk}{0.45,0.86,0,0}
\definecolor{aguamarina}{cmyk}{0.85,0,0.33,0}
\definecolor{esmeralda}{cmyk}{0.91,0,0.88,0.12}
\definecolor{pino}{cmyk}{0.92,0,0.59,0.25}
\definecolor{oliva}{cmyk}{0.64,0,0.95,0.40}
\definecolor{canela}{cmyk}{0.14,0.42,0.56,0}
\definecolor{cafe}{cmyk}{0,0.81,1,0.60}
\definecolor{marron}{cmyk}{0,0.72,1,0.45}
\definecolor{gris-claro}{cmyk}{0,0,0,0.30}
\definecolor{gris-oscuro}{cmyk}{0,0,0,0.50}

\begin{document}
\chapter{conjuntos}
    \begin{enumerate}
        \item Relación
        \item Relación funcional
        \item Función
        \item conjunto de llegada
        \item Dominio
        \item conjunto de salida
        \item Codominio 
        \item Rango
        \item Función inversa
        \item Función inyectiva 
        \item Función sobreyectiva 
        \item Función biyectiva 
        \item conjunto infinito 
        \item conjunto finito 
        \item Relacion de orden 
        \item Relacion de orden total 
        \item Relación de equivalencia 
    \end{enumerate}
\chapter{Logica}
\section{Inferncia lógica}
\subsection{Modus ponendo Ponens} 
\textbf{Premisa 1} Si él está en el partido de fútbol, entonces él esta en el estadio. \\ 
\textbf{Premisa 2} Él está en el partido de fútbol \\
\textbf{Conclusión} Él está en el estadio. \\
Simbólicamente, el primer ejemplo se extresa así: \\ 
Sea: \\ P:="Él está en el partido de fútbol" \\ 
Q:="él está en el estadio" 
entonces:  \\ 
$\frac{%
\begin{array}{l}
p\rightarrow q \\ 
p%
\end{array}
}{%
\begin{array}{l}
q%
\end{array}
}$
\subsection{La regla de doble negación}
La regla de doble negación es una regla simple que permite pasar de una premsisa única a la concluión. Un 
ejemplo simple es el de una negación, que brevemente se denomina "doble negación". Sea la proposición: \\ 
No ocurre que Ana no es estudiante\\ 
Evidentemente se puede decir: \\ 
Ana es un estudiante. \\ 
Así la regla de doble negación tiene dos formas simbólicas. 

\end{document}