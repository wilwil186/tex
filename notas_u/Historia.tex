\documentclass{book}
\usepackage[utf8]{inputenc}
\usepackage{hyperref}
\usepackage{graphicx}
\usepackage{geometry}
\usepackage{multicol}
\usepackage{amsmath}
\usepackage{amsthm,amsfonts,amssymb}%paquetes AMS
 
\begin{document}
\chapter{Infinito en las matemáticas griegas}
    quizás la caracteristica más interesante y más moderna de las matemáticas griegas radica en su tratamiento del infinito. LOs giegos temían el infinito y trataron de evitarlo, pero 
    al hacerlo sentaron las bases para un tratamineto rigurosos de procesos infinitos en el cálculo del siglo XIX.
    Las contribuciones más originales a la teoría del infinito en la antigüedad.
    eran la teoría de las proporciones y el método del agotamiento. 
    Ambos fueron ideado por Eudoxo y expuesto en el Libro V de los elementos de Euclides.
    La teoría de proporciones desarrolla la idea de que una "cantidad" $\lambda$ (Lo que ahora llamaríamos un número real)
    puede ser conocido por su posición entre los numeros racionales.Es decir , se conoce $\lambda$ si conocemos los números racionales menores que $\lambda$ 
    y los números racionales mayores que $\lambda$. el método de agotamiento generaliza esta idea de "cantidades" a regiones del plano o el espacio.
    Una región se vuelve "conocida" (en área o vulumen) cuando se conoce su posición entre áreas de lso polígonos dentro de él y las áreas de los poígonos fuera de él;
    conocemos el voulmen de una piámide cuando conocemos los volúmenes de pilas de prismas dentro y fuera de ella.
    Usando este método, Euclides encontró que el volumen de un tetraedro es igual a un 1/3 del área de su base pro su altura, y Arquímedes encontró el área de un segmento parabólico. Ambos se basaron en un proceso infinito que es fundamental para muchos cálculos de área y volumen: 
    la suma de una serie geométrica infinita. 

    \section{Miedo al infinito}
    el rechazo de los griegos a los números irracionales era solo parte de un rechazo general de los procesos infinitos.
    De hecho, hasta finales del siglo XIX, la mayoría de los matemáticos se mostraban reacios a aceptar el infinito como algo más que "potencial".
    La infinitud de un proceso, colección o magnitud se entendía como la posibilidad de su continuación indefinida, y nada más, ciertamente no como la posibilidad de una eventual culminación. Por ejemplo, los números naturales 1, 2, 3,. . ., se puede aceptar como un infinito potencial, generado a partir de 1 mediante el proceso de sumar 1, sin aceptar que hay una totalidad completa.
    Según la tradición, se asustaron ante las paradojas de Zenón, alrededor del 450 a. C.
    Conocemos los argumentos de Zenón solo a través de Aristóteles, quien los cita
    en su Física para refutarlos, y no está claro qué deseaba lograr el propio Zenón. ¿Había, por ejemplo, una tendencia a la especulación sobre el infinito que desaprobaba? Sus argumentos son tan extremos que casi podrían ser parodias de argumentos vagos sobre el infinito que escuchó entre
    sus contemporáneos.
    \textbf{zenón fallo!} pero concluyo que todo numero finito puede dividirse un número infinito de veces.
    este concepto se llama "series infinitas" y es usado en finanzas para calcular los pagos hipotecarios. por eso toma tanto tiempo pagarlos.
    
    \section{Teoría de las proporciones de Eudoxo}
    La teoría de las proporciones se atribuye a Eudoxo (alrededor de 400-350 a. C.)
    y se expone en el Libro V de los Elementos de Euclides. El propósito de
    La teoría es permitir que las longitudes (y otras cantidades geométricas) sean tratadas
    tan precisamente como los números, admitiendo sólo el uso de números racionales.
    recordemos que los griegos no podían aceptar números irracionales, pero aceptaban cantidades geométricas irracionales tales
    como la diagonal del cuadrado unitario. Para simplificar la exposición de la teoría,
    llamemos racionales a las longitudes si son múltiplos racionales de una longitud fija.
    La idea de Eudoxo era decir que una longitud $\lambda$ está determinada por aquellos
    longitudes racionales menores y mayores. Para ser precisos, dice
    $\lambda_{1}$ = $\lambda_{2}$ si cualquier longitud racional $<\lambda_{1}$  también es $<\lambda_{2}$, y viceversa. igualmente
    $<\lambda_{1}<\lambda_{2}$ si hay una longitud racional$>\lambda_{1}$ pero $<\lambda_{2}$. Esta definición utiliza
    los racionales para dar una noción infinitamente aguda de longitud evitando
    cualquier uso manifiesto del infinito. Por supuesto, el conjunto infinito de longitudes racionales $<\lambda_{1}$
    está presente en espíritu, pero Eudoxo evita mencionarlo hablando de un
    longitud racional arbitraria $<\lambda$.

    \section{}
    
\end{document}