\documentclass{book}
\usepackage[utf8]{inputenc}
\usepackage{hyperref}
\usepackage{graphicx}
\usepackage{geometry}
\usepackage{multicol}
\usepackage{amsmath}
\usepackage{amsthm,amsfonts,amssymb}%paquetes AMS
 
\begin{document}
\chapter{Infinito en las matemáticas griegas}
    quizás la caracteristica más interesante y más moderna de las matemáticas griegas radica en su tratamiento del infinito. LOs giegos temían el infinito y trataron de evitarlo, pero 
    al hacerlo sentaron las bases para un tratamineto rigurosos de procesos infinitos en el cálculo del siglo XIX.
    Las contribuciones más originales a la teoría del infinito en la antigüedad.
    eran la teoría de las proporciones y el método del agotamiento. 
    Ambos fueron ideado por Eudoxo y expuesto en el Libro V de los elementos de Euclides.
    La teoría de proporciones desarrolla la idea de que una "cantidad" $\lambda$ (Lo que ahora llamaríamos un número real)
    puede ser conocido por su posición entre los numeros racionales.Es decir , se conoce $\lambda$ si conocemos los números racionales menores que $\lambda$ 
    y los números racionales mayores que $\lambda$. el método de agotamiento generaliza esta idea de "cantidades" a regiones del plano o el espacio.
    Una región se vuelve "conocida" (en área o vulumen) cuando se conoce su posición entre áreas de lso polígonos dentro de él y las áreas de los poígonos fuera de él;
    conocemos el voulmen de una piámide cuando conocemos los volúmenes de pilas de prismas dentro y fuera de ella.
    Usando este método, Euclides encontró que el volumen de un tetraedro es igual a un 1/3 del área de su base pro su altura, y Arquímedes encontró el área de un segmento parabólico. Ambos se basaron en un proceso infinito que es fundamental para muchos cálculos de área y volumen: 
    la suma de una serie geométrica infinita. 

    \section{Miedo al infinito}
    el rechazo de los griegos a los números irracionales era solo parte de un rechazo general de los procesos infinitos.
    De hecho, hasta finales del siglo XIX, la mayoría de los matemáticos se mostraban reacios a aceptar el infinito como algo más que "potencial".
    La infinitud de un proceso, colección o magnitud se entendía como la posibilidad de su continuación indefinida, y nada más, ciertamente no como la posibilidad de una eventual culminación. Por ejemplo, los números naturales 1, 2, 3,. . ., se puede aceptar como un infinito potencial, generado a partir de 1 mediante el proceso de sumar 1, sin aceptar que hay una totalidad completa.
    Según la tradición, se asustaron ante las paradojas de Zenón, alrededor del 450 a. C.
    Conocemos los argumentos de Zenón solo a través de Aristóteles, quien los cita
    en su Física para refutarlos, y no está claro qué deseaba lograr el propio Zenón. ¿Había, por ejemplo, una tendencia a la especulación sobre el infinito que desaprobaba? Sus argumentos son tan extremos que casi podrían ser parodias de argumentos vagos sobre el infinito que escuchó entre
    sus contemporáneos.
    \textbf{zenón fallo!} pero concluyo que todo numero finito puede dividirse un número infinito de veces.
    este concepto se llama "series infinitas" y es usado en finanzas para calcular los pagos hipotecarios. por eso toma tanto tiempo pagarlos.
    \subsection{arquimedes y la tortuga}
    Aquiles y la tortuga es, quizás, la más conocidas de las paradojas de Zenón. 
    El filósofo argumentaba que, en una hipotética carrera entre Aquiles (el guerrero que mató a Héctor) 
    y una tortuga,  si esta tenía última una ventaja inicial, el humano siempre perdería. 
    Zenón “demostraba” que, a pesar de que el guerrero corre mucho más rápido que la tortuga, 
    nunca podría alcanzarla.

    Imaginemos que la distancia a cubrir en la carrera son cien metros, 
    y que la tortuga tiene cincuenta metros de ventaja. Al darse la orden de salida, 
    Aquiles recorre en poco tiempo la distancia (cincuenta metros) que los separaba inicialmente. 
    Pero, al llegar allí, descubre que la tortuga ya no está, sino que ha avanzado, mucho más 
    lentamente, diez o veinte centímetros. Lejos de desanimarse, el guerrero sigue corriendo. 
    Pero, al llegar de nuevo donde estaba la tortuga, ésta ha avanzado un poco más. 
    Zenón sostiene que esta situación se repite indefinidamente, y que Aquiles jamás 
    logrará alcanzar a la tortuga, que finalmente ganará la carrera.
    
    Es bastante obvio que esto no es así, y es muy fácil comprobar en la práctica que dicho 
    razonamiento es erróneo. Sin embargo, no es tan fácil encontrar donde está el fallo, 
    y hubo que esperar hasta mediados del siglo XVII para que el matemático escocés James Gregory 
    demostrara matemáticamente que una suma de infinitos términos puede tener un resultado finito. 
    Los tiempos en los que Aquiles recorre la distancia que lo separa del punto anterior en el que 
    se encontraba la tortuga son infinitos, pero cada vez más y más pequeños. La suma de todos estos 
    tiempos, a pesar de su infinito número, da como resultado un lapso de tiempo finito, 
    que es el momento en que Aquiles alcanzará a la tortuga.
    

    \section{Teoría de las proporciones de Eudoxo}
    La teoría de las proporciones se atribuye a Eudoxo (alrededor de 400-350 a. C.)
    y se expone en el Libro V de los Elementos de Euclides. El propósito de
    La teoría es permitir que las longitudes (y otras cantidades geométricas) sean tratadas
    tan precisamente como los números, admitiendo sólo el uso de números racionales.
    recordemos que los griegos no podían aceptar números irracionales, pero aceptaban cantidades 
    geométricas irracionales tales
    como la diagonal del cuadrado unitario. Para simplificar la exposición de la teoría,
    llamemos racionales a las longitudes si son múltiplos racionales de una longitud fija.
    La idea de Eudoxo era decir que una longitud $\lambda$ está determinada por aquellos
    longitudes racionales menores y mayores. Para ser precisos, dice
    $\lambda_{1}$ = $\lambda_{2}$ si cualquier longitud racional 
    $<\lambda_{1}$  también es $<\lambda_{2}$, y viceversa. igualmente
    $<\lambda_{1}<\lambda_{2}$ si hay una longitud racional$>\lambda_{1}$ pero $<\lambda_{2}$. 
    Esta definición utiliza
    los racionales para dar una noción infinitamente aguda de longitud evitando
    cualquier uso manifiesto del infinito. Por supuesto, el conjunto infinito de longitudes 
    racionales $<\lambda_{1}$
    está presente en espíritu, pero Eudoxo evita mencionarlo hablando de un
    longitud racional arbitraria $<\lambda$. \\ \\ 
    La teoría de las proporciones tuvo tanto éxito que retrasó el desarrollo de una teoría de los 
    números reales durante 2000 años. Esto fue irónico, porque la teoría de las proporciones puede 
    usarse para definir números irracionales tan bien como longitudes. Sin embargo, era comprensible, 
    porque las longitudes irracionales comunes, como la diagonal del cuadrado unitario, surgen de 
    construcciones que son intuitivamente claras y finitas desde el punto de vista geométrico. 
    Cualquier enfoque aritmético de $\sqrt{2}$, ya sea por secuencias, decimales o fracciones continuas, 
    es infinito y, por lo tanto, menos intuitivo. Hasta el siglo XIX, esto parecía una buena razón 
    para considerar que la geometría era una base mejor para las matemáticas que la aritmética.
    Luego, los problemas de la geometría llegaron a un punto crítico y
    los matemáticos comenzaron a temer a la intuición geométrica tanto como antes temían al infinito.
    Hubo una purga del razonamiento geométrico de los libros de texto y una laboriosa reconstrucción 
    de las matemáticas sobre la base de números y conjuntos de números. La teoría de conjuntos 
    se analiza con más detalle en el capítulo 24. Basta decir, por el momento, que la teoría de 
    conjuntos depende de la aceptación de infinitos completos.
    \\ \\ 
    La belleza de la teoría de la proporción radicaba en su adaptabilidad a este nuevo clima. 
    En lugar de longitudes racionales, tome números racionales. En lugar de comparar longitudes 
    irracionales existentes por medio de longitudes racionales,
    ¡construya números irracionales desde cero usando conjuntos de racionales!
    La longitud de $\sqrt{2}$ es determinada por los dos conjuntos de racionales positivos. 
    $L_{\sqrt{2}} = \{r: r^{2} < 2\}$ y $U_{\sqrt{2}} = \{r: r^{2} > 2\}$
    Dedekind (1872) decidió dejar que $\sqrt{2}$ fueran este par de conjuntos. En general, 
    sea cualquier partición 
    de los racionales positivos en conjuntos L, U tal que cualquier miembro de L sea menor que 
    cualquier miembro de U sea un número real positivo. Esta idea, ahora conocida como un corte de 
    Dedekind, es más que un giro de Eudoxus; da una construcción completa y uniforme de todos los 
    números reales, o puntos en la línea, usando solo los racionales. En resumen, es una explicación 
    de la continua
    en términos de lo discreto, resolviendo finalmente el conflicto fundamental en la matemática 
    griega. Dedekind estaba comprensiblemente satisfecho con su logro.
    El escribio.
    La afirmación se hace con tanta frecuencia que el cálculo diferencial se ocupa de la magnitud 
    continua y, sin embargo, en ninguna parte se da una explicación de esta continuidad. . . . 
    Entonces sólo quedaba descubrir su verdadero origen en los elementos de la aritmética y así, al
    mismo tiempo, asegurar una definición real de la esencia de la continuidad. 
    Lo logré el 24 de noviembre de 1858
    \section{Notas biográficas: Arquímedes}
    Arquímedes es uno de los pocos matemáticos antiguos cuya vida se conoce en detalle, gracias a 
    la atención que recibió de autores clásicos como Plutarco, Livio y Cicerón y su participación 
    en el sitio históricamente significativo de Siracusa en el 212 a. C. Nació en Siracusa 
    (una ciudad griega en lo que hoy es Sicilia) alrededor del 287 a. C. e hizo la mayor parte 
    de su trabajo importante allí, aunque es posible que haya estudiado durante un tiempo en 
    Alejandría. Parece haber estado relacionado con el gobernante de Siracusa, el rey Hierón 
    II, o al menos en buenos términos con él. Hay muchas historias de mecanismos inventados 
    por Arquímedes en beneficio de Hierón: poleas compuestas para barcos en movimiento, dispositivos 
    balísticos para la defensa de Siracusa y un planetario modelo. La historia más famosa 
    sobre Arquímedes es la que cuenta Vitruvio (De architectura, Libro IX, Cap. 3), que tiene a 
    Arquímedes saltando de su baño con un grito de "¡Eureka!" cuando se dio cuenta de que pesar 
    una corona sumergida en agua le permitiría comprobar si era oro puro. Los historiadores dudan 
    de la autenticidad de esta historia, pero al menos reconoce la comprensión de Arquímedes de la 
    hidrostática. En la antigüedad, la reputación de Arquímedes se basaba en sus inventos mecánicos, 
    que sin duda eran más comprensibles para la mayoría de la gente que
    sus matemáticas puras. Sin embargo, también se puede argumentar que su mecánica teórica (
    incluida la ley de la palanca, los centros de masa, el equilibrio y la presión hidrostática) 
    fue su contribución más original a la ciencia. Antes de Arquímedes no existía en absoluto una 
    teoría matemática de la mecánica, solo la mecánica completamente incorrecta de Aristóteles. 
    En matemática pura, Arquímedes no hizo ningún ajuste conceptual comparable.
    vances, excepto quizás en su Método, que utiliza sus ideas de la estática como un medio para 
    descubrir resultados en áreas y volúmenes. Los conceptos que Arquímedes necesitaba para las 
    pruebas en geometría —la teoría de las proporciones y el método de agotamiento— ya los había 
    proporcionado Eudoxo, y
    Fue la fenomenal perspicacia y la técnica de Arquímedes lo que lo elevó por encima de sus 
    contemporáneos.
    La historia de la muerte de Arquímedes se ha contado a menudo, aunque con distintos detalles. 
    Fue asesinado por un soldado romano cuando Siracusa cayó ante los romanos bajo el mando de Marcelo 
    en el 212 a. C. Probablemente estaba haciendo matemáticas en el momento de su muerte, 
    pero si enfureció a un soldado al decir "¡Apártate de mi diagrama!" es una conjetura. 
    Esta historia nos ha llegado de Tzetzes (Chiliad, Libro II). Otra versión de la muerte de 
    Arquímedes se da en Vidas de Plutarco, en el capítulo sobre Marcelo. Plutarco también nos 
    dice que Arquímedes pidió que en su lápida se inscribiera una figura y descripción de su 
    resultado favorito, la relación entre los volúmenes de la esfera y el cilindro. (Demostró que 
    el volumen de la esfera es dos tercios del del cilindro envolvente. Ver Heath (1897), p. 43, y 
    Ejercicio 9.2.5.) Un siglo y medio después, Cicer (Tusculan Disputations, Book V ) informó haber 
    encontrado la lápida cuando era cuestor en Sicilia en el 75 a. C. La tumba había sido descuidada, 
    pero la figura de la esfera y el cilindro aún era reconocible.
\chapter{la geometría proyectiva}
    \section{Resumen}
    Aproximadamente al mismo tiempo que la revolución algebraica en la geometría clásica, 
    también salió a la luz un nuevo tipo de geometría: la geometría proyectiva. Partiendo de la idea de 
    proyectar una figura de un plano a otro, la geometría proyectiva fue inicialmente la preocupación de los artistas. 
    En el siglo $XVII$, solo un puñado de matemáticos estaban interesados en él, y sus descubrimientos no se 
    consideraron importantes hasta el siglo $XIX$. Las cantidades fundamentales de la geometría clásica, como la 
    longitud y el ángulo, no se conservan mediante la proyección, por lo que no tienen significado en la geometría 
    proyectiva. La geometría proyectiva solo puede discutir cosas que se conservan mediante proyección, 
    como puntos y líneas. Sorprendentemente, existen teoremas no triviales sobre puntos y líneas. Uno de ellos 
    fue descubierto por el geómetra griego Pappus alrededor del año 300 $d.C.$, y otro por el matemático francés 
    Desargues alrededor de 1640. Aún más sorprendente, hay una cantidad numérica preservada por proyección. 
    Es una "proporción de proporciones" de longitudes llamada proporción cruzada. En geometría proyectiva, 
    la relación cruzada juega un papel similar al que juega la longitud en la geometría clásica.
    Una de las virtudes de la geometría proyectiva es que simplifica la clasificación de curvas. 
    Todas las secciones cónicas, por ejemplo, son \textquotedblleft proyectivamente iguales\textquotedblleft y solo hay cinco tipos de curvas cúbicas. 
    El punto de vista proyectivo también elimina algunas aparentes excepciones al teorema de Bézout. Por ejemplo, 
    una línea (curva de grado 1) siempre se encuentra con otra línea exactamente en un punto, porque en la geometría 
    proyectiva incluso las líneas paralelas se encuentran.

\end{document}