\section{Identidad de Lagrange}

\begin{frame}{Identidad de Lagrange}
Para números reales $x_{1},...,x_{n}$ y $y_{1},...,y_{n}$
\begin{equation*}
    (\sum_{k=1}^{n}x_{k}y_{k})^{2} = (\sum_{k=1}^{n} x^{2}_{k})(\sum_{k=1}^{n}y^{2}_{k})-\sum_{1\leq i<j \leq n}(x_{i}y_{j}-x_{j}y_{i})^{2}
\end{equation*}
    
\end{frame}
%%%%%%%%%%%%%%%%%%%%%%%%%%%%%%%%%%%%%%%%%%%%%%%%%%%%%%%%%%%%%%%%%%%%%%%%%%%%%%%%%%%%%%%%%%%%%%%%%%%%%%%%%%%%%%%%%%%%%%%%%%%%%%%%%%%%%%
\begin{frame}{Demostración}
Por teorema anterior 
\begin{equation}
    (\sum_{k=1}^{n}x_{k}y_{k})^{2} =  \sum_{k=1}^{n} x^{2}_{k}y^{2}_{k} +\underline{ \sum_{1 \leq i < j \leq n} 2 x_{i}y_{j}x_{j}y_{i}}
\end{equation}

Observamos
\begin{equation}
    (x_{i}y_{j} - x_{j}y_{i})^{2} = x^{2}_{i}y^{2}_{j}+x^{2}_{j}y^{2}_{i}-\underline{2x_{i}y_{j}x_{j}y_{i}}
\end{equation}
  
aplicando axiomas de los reales 
\begin{equation}
    2x_{i}y_{j}x_{j}y_{i} = x^{2}_{i}y^{2}_{j}+x^{2}_{j}y^{2}_{i} - (x_{i}y_{j} - x_{j}y_{i})^{2} \to 
\end{equation}

\begin{equation}
    (\sum_{k=1}^{n}x_{k}y_{k})^{2} = \sum_{k=1}^{n} x^{2}_{k}y^{2}_{k} + \sum_{1 \leq i < j \leq n}  x^{2}_{i}y^{2}_{j}+x^{2}_{j}y^{2}_{i} - (x_{i}y_{j} - x_{j}y_{i})^{2} 
\end{equation}

\end{frame}
\begin{frame}
    
reescribiendo 
\begin{equation}
    (\sum_{k=1}^{n}x_{k}y_{k})^{2} = \underline{ \sum_{k=1}^{n} x^{2}_{k}y^{2}_{k} +  \sum_{1 \leq i < j \leq n} x^{2}_{i}y^{2}_{j}} -  \sum_{1 \leq i < j \leq n} (x_{i}y_{j} - x_{j}y_{i})^{2} 
\end{equation}

utilizamos el Lema 2 
\begin{equation}
    (\sum_{k=1}^{n}x_{k}y_{k})^{2} = (\sum_{k=1}^{n} x^{2}_{k})(\sum_{k=1}^{n} y^{2}_{k}) -  \sum_{1 \leq i < j \leq n} (x_{i}y_{j} - x_{j}y_{i})^{2}
\end{equation}

\textbf{Q.E.D.}
\end{frame}