\section{Desigualdad de Cauchy-schwarz}

\begin{frame}{Desigualdad de Cauchy-schwarz en $R^{n}$}
Sean $X$ y $Y$ vectores en $R^{n}$, entonces: 
\begin{equation*}
    |X \cdot Y| \leq ||X|| \cdot ||Y||
\end{equation*}

\end{frame} 


\begin{frame}{Por componentes}
De manera equivalente podemos escribir 
\begin{equation*}
    |X \cdot Y|^{2} \leq ||X||^{2} \cdot ||Y||^{2}
\end{equation*}


Notese que para $X=(x_{1},x_{2}, \cdots , x_{n})$ y $Y=(y_{1},y_{2}, \cdots , y_{n})$\\ 
$||X||^{2}=(x^{2}_{1}+x^{2}_{2}+ \cdots x^{2}_{n})=\sum_{k=1}^{n}x^{2}_{k}$ y $||Y||^{2}=(y^{2}_{1}+y^{2}_{2}+ \cdots y^{2}_{n})=\sum_{k=1}^{n}y^{2}_{k}$ y \\
$|X \cdot Y|^{2}=(x_{1}y_{1}+x_{2}y_{2}+ \cdots + x_{n}y_{n})^{2}=(\sum_{k=1}^{n}x_{k}y_{k})^{2}$ 
\\ 
Por tanto la desigualdad de Cauchy-schwarz en $R^{n}$ escrita por componentes quedaria 
\begin{equation*}
    (\sum_{k=1}^{n} x^{2}_{k})(\sum_{i=1}^{n}y^{2}_{k}) \geq (\sum_{k=1}^{n}x_{k}y_{k})^{2}
\end{equation*}
\end{frame}

\begin{frame}{Demostración}
Utilizando la identidad de lagrange: 
$(\sum_{k=1}^{n}x_{k}y_{k})^{2} = (\sum_{k=1}^{n} x^{2}_{k})(\sum_{i=1}^{n}y^{2}_{k})- \underline{ \sum_{1\leq i<j \leq n}(x_{i}y_{j}-x_{j}y_{i})^{2}} \to (\sum_{k=1}^{n}x_{k}y_{k})^{2} + \sum_{1\leq i<j \leq n}(x_{i}y_{j}-x_{j}y_{i})^{2} = (\sum_{k=1}^{n} x^{2}_{k})(\sum_{i=1}^{n}y^{2}_{k})$ de donde $(\sum_{k=1}^{n} x^{2}_{k})(\sum_{i=1}^{n}y^{2}_{k}) \geq (\sum_{k=1}^{n}x_{k}y_{k})^{2} $\\Lo cual demuestra la desigualdad de Cauchy-schwarz en $R^{n}$. \\ 
\textbf{Q.E.D.}
\end{frame}
