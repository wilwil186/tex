\documentclass{book}
\usepackage[utf8]{inputenc}
\usepackage{hyperref}
\usepackage{graphicx}
\usepackage{geometry}
\usepackage{multicol}
\usepackage{amsmath}
\usepackage{amsthm,amsfonts,amssymb}%paquetes AMS

\begin{document}
    \chapter{Conceptos Básicos}
    \section{Experimento aleatorio}
    Un experimento se dice \textbf{aleatorio} si su resultado no puede ser determinado de antemano.
    \\ 
    \textbf{Ejemplos} \\ 
    Son ejemplos de experimentos aleatorios. 
    \begin{itemize}
        \item El lanzamineto de un dado.
        \item El lanzamineto de una moneda.
        \item Observar el precio del dolar con respecto al precio del peso colombiano en el momento en que cierra la bolsa de valores de 
        New York.
        \item Observar el número de clientes que son atendidos en una hora en una sucursal bancaría.
    \end{itemize}
    \section{Espacio Muestral}
    El conjunto $\Omega$ de todos los posibles resultados de un experimento aleatorio se llama \textbf{espacio muestral}.
    Los elementos $w \in \Omega$  son llamados \textbf{puntos muestrales}. \\ 
    \textbf{Ejemplos} \\ 
    \begin{itemize}
        \item En el experimento del lanzamiento de un dado corriente el espacio muestral es 
            \begin{equation*}
                \omega = \{ 1, 2 ,3 ,4 ,5 ,6\}
            \end{equation*}
        \item En el experimento del lanzamiento de una moneda corrinete. Los posibles resultados, para este 
        experimento son : 'cara'=c y 'sello'=s, esto es 
            \begin{equation*}
                \omega = \{ c,s\}
            \end{equation*}
    \end{itemize}
    \textbf{El espacio muestral $\Omega$ se llama discreto si es finito o enumerable}.
    \section{$\sigma$-álgebra}
    Sea $\Omega \not = \emptyset $.Una colección $\zeta$ de subconjuntos de $\Omega$ se llama $\sigma$-álgebra, si y solo si.
    \begin{itemize}
        \item $\Omega \in \zeta$
        \item Si $A \in \zeta$, entonces $\displaystyle\bigcup_{i \in \mathbb{N}}{A_{i} \in \zeta }$
    \end{itemize}
    Los elementos de $\zeta$ se llaman \textbf{eventos}.
    \section{Espacio medible}
    Sean $\Omega \not = \emptyset $ y  $\zeta$ una $\sigma$-álgebra.La pareja 
    $( \Omega, \zeta )$ se llama \textbf{Espacio medible}.
    \section{Eventos mutuamente excluyentes}
    Dos eeventos A y B se dicen mutuamente exluyentes si $A \cap B = \emptyset$ 
\end{document}