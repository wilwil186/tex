\documentclass[12pt]{article}
\usepackage[utf8]{inputenc}
\usepackage{hyperref}
\usepackage{graphicx}
\usepackage{geometry}
\usepackage{multicol}
\usepackage{amsmath}

%%cuerpo del documento %%%%%%%%%%%%%%%%%%%%%%%%%%%%%%%%%%%%%%%%%%%%%%%%%%%%%%%%%%%%%%%%%%%%%%%%%%%%%%%%%%%%%%%%%%%%%%%%%%
\begin{document}


%%%%%%%%%%%%%%%%%%%%%%%presentación %%%%%%%%%%%%%%%%%%%%%%%%%%%%%%%%%%%%%%%%%%%%%%%%%%%%%%%%%%%%%%%%%%%%%%%%%%%%%%%%

\pagestyle{empty}
%\afterpage{\blankpage}
\begin{center}
\begin{figure}[h]
\centering


\end{figure}
\Large
\hrule
\vspace{4mm}
\textbf{Notas del Curso de Introducción al \LaTeX}\\

\vspace{4mm}
\hrule
\large
\vfill
Autor\\

Wilson Eduardo Jerez Hernández \\
20181167034\\ 
\vfill
Profesor\\

Jhonatan Steven Mora Rodríguez
\vfill
Universidad Distrital Francisco José de Caldas\\
Facultad de Ciencias y educación\\
Introducción al \LaTeX\\
\end{center}
\newpage


%%%%%%%%%%%%%%%%%%%%tabla de contenidos%%%%%%%%%%%%%%%%%%%%%%%%%%%%%%%%%%%%%%%%%%%%%%%%%%%%%%%%%%%%%%%%%%%%%%%%%%%%

\addtocontents{toc}{\hfill \textbf{Página} \par}
\tableofcontents
\newpage
%%%%%%%%%%%%%%%%%%%%%%%%%%%%%%%%%%%%%%%%%%%%%%%%%%%%%%%%%%%%%%%%%%%%%%%%%%%%%%%%%%%%%%%%%%%%%%%%%%%%%%%%%%%%%%%%%%%
\begin{abstract}
    hola
\end{abstract}
\newpage
%%%%%%%%%%% primera sección
\section{Evaluación}
        \begin{enumerate}
            \item 1er corte 35\% la nota total se tomará del promedio total de los talleres durante el corte.(cada taller se entrega el mismo dia en el cual fue asignado)
            \item 2er corte 35\% la nota total se tomará del promedio total de los talleres durante el corte.
            \item Final 30\% construir un documento en LaTeX, con las siguientes caraterísticas: 
                \begin{itemize}
                    \item Minimo debe tener 20 páginas
                    \item El doumento debe contener: ecuaciones, tablas, incluir gráficas, bibliografía, contenido, etc.
                    \item Deben entregar el archivo.tex
                \end{itemize}
    \end{enumerate}
%%%%%%%%%%%%%%2 sección 
\section{Corte 1}
%%%%%%%%%%% 
    \subsection{Introducción}
        utilizamos \% para comentar . En el preambulo colocamos los paquetes que vamos a utilizar.
        Hay algunos simbolos que no se pueden utilizar libremente pues \LaTeX lo tiene guardados para algunas cosas especificas.
\\
%una tabla 
\begin{tabular}{c|c}\hline
Introducción & resultado \\ \hline \hline
\textbackslash\% & \% \\ 
\textbackslash\$ & \$ \\
\textbackslash\& & \&
\end{tabular}
\\
Para centrar un texto podemos utilzar el comando \textbackslash centerline\{COMA\}, 
Hay comando simples y complejos 
los comando simples son como el que acabamos de ver, los complejos son por ejemplo: 
\textbackslash includegraphics[ options ]\{name\}. 
otro concepto a tener en cuenta son la declaraciones goblales y locales. 
local seria por ejemplo el  \textbackslash textit\{hola mundo\} pues escribiria en cursiva solo el hola mundo 
a cambio una declaración global hace que todo el documento este en cursiva. loes entorno simpre hay que cerrarlos.
para capitulos \textbackslash chapter\{Corte 1\}
para  secciónes \textbackslash section\{Introducción\}
para subsecciónes \textbackslash subsection\{Introducción\}
\\ 
se puede escribir a dos columnas modificando las opciones en el : \textbackslash documentclass[options]\{book\}

\end{document}