\documentclass{article}
\usepackage[utf8]{inputenc}
\usepackage[spanish]{babel}
\usepackage{hyperref}
\usepackage{graphicx}
\usepackage{geometry}
\usepackage{multicol}
\usepackage{amsmath}
\usepackage{stmaryrd}

\begin{document}

\begin{enumerate}
    \item modelo de población logístico pag. 8 
    \item ejercicio 18 de la seccion 1.1. Libro Ecuaciones Diferenciales (Blanchard, Devaney, Hall)
\end{enumerate}
\textbf{modelo de población logístico} \\
Recursos limitados y modelo logístico de población\\
Para ajustar el modelo de crecimiento exponencial de la población para tener en cuenta un entorno y recursos limitados, agregamos los supuestos:
\begin{itemize}
    \item Si la población es pequeña, la tasa de crecimiento de la población es proporcional a su tamaño.
    \item Si la población es demasiado grande para ser sustentada por su medio ambiente y recursos, 
    la población disminuirá. Es decir, la tasa de crecimiento es negativa.
\end{itemize}
Para este modelo, usamos nuevamente
\begin{center}
    t = tiempo (variable independiente),\\
    P = población (variable dependiente),\\
    k = coeficiente de tasa de crecimiento para pequeños\\
    poblaciones (parámetro).
\end{center}
Sin embargo, nuestra suposición sobre recursos limitados introduce otra cantidad, 
el tamaño de la población que corresponde a ser "demasiado grande". Esta cantidad es un segundo 
parámetro, denotado por N, que llamamos capacidad de carga del medio ambiente. En términos de capacidad 
de carga, 
asumimos que $P (t)$ aumenta si $P(t) <N$. Sin embargo, si $P (t)> N$, asumimos que $P (t)$ 
es decreciente. \\ 
Usando esta notación, podemos restaurar nuestras suposiciones como:
\begin{enumerate}
    \item $\frac{dP}{dt} \approx KP $si  P es pequeño.
    \item si $P > N$. $\frac{dP}{dt}<0$.
\end{enumerate}
También queremos que el modelo sea "algebraicamente simple", o al menos tan simple como sea posible,
por lo que tratamos de modificar el modelo exponencial lo menos posible. Por ejemplo, podríamos buscar 
una expresión de la forma
\begin{equation*}
    \frac{dP}{dt} = k
\end{equation*}
Queremos que el factor "algo" esté cerca de 1 si P es pequeño, pero si $P> N$
queremos que "algo" sea negativo. La expresión más simple que tiene estas propiedades es la
función
\begin{center}
    (algo)=$(1-\frac{P}{N})$
\end{center}
Note that this expression equals 1 if P = 0, and it is negative if $P > N$ . Thus our model is
\begin{center}
    $\frac{dP}{dt}=k(1-\frac{P}{N})P$
\end{center}
Esto se denomina modelo logístico de población con tasa de crecimiento k y capacidad de carga N. 
Es otra ecuación diferencial de primer orden. Se dice que esta 
ecuación no es lineal porque su lado derecho no es una función lineal de P como lo era en el modelo 
de crecimiento exponencial.
\\ \textbf{Análisis cualtitativo} \\ 
Aunque la ecuación diferencial logística es un poco más complicada que el modelo de crecimiento exponencial, no hay forma de que podamos adivinar soluciones. El método de separación de variables discutido en la siguiente sección produce una fórmula para la solución de esta ecuación diferencial particular. Pero por ahora, nos basamos únicamente en métodos cualitativos para ver qué predice este modelo a largo plazo.
\\ 
%%%%%%%%%%%%%%%%%%%%%%%%%%%%%%%%%%%%%%%%%%%%%%%%%%%%%%%%%%%%%%%%%%%%%%%%%%%%%%%%%%%%%%%%%%%%%%%%%%%%%%%%%%%%%%%5
\textbf{Desarrollo ejercicio 18}\\ 
\\ 
Preliminares \\ 
Ejercicio 17: \\ 
Suponga que una especie de pez en un lago en particular tiene una población modelada por el modelo logístico 
de población con tasa de crecimiento k, capacidad de carga N y tiempo t medido en años. 
Ajuste el modelo para tener en cuenta cada una de las siguientes situaciones.
\begin{enumerate}
    \item Cada año se recolectan cien peces.
    \item Un tercio de la población de peces se recolectan anualmente.
    \item El número de peces que se recolectan cada año es proporcional a la raíz 
    cuadrada del número de peces en el lago.   
\end{enumerate}

Ejercicio 18: \\ 
Suponga que el parámetro de la tasa de crecimiento k = 0.3 y la capacidad de carga 
N = 2500 en el modelo 
logístico de población del ejercicio 17. Suponga que P (0) = 2500.
\begin{enumerate}
    \item  Si se recolectan 100 peces cada año, ¿qué predice el modelo para el 
    comportamiento a largo plazo de la población de peces? En otras palabras, 
    ¿qué arroja un análisis cualitativo del modelo?
    \item Si se recolectan un tercio de los peces cada año, ¿qué predice el modelo para 
    el comportamiento a largo plazo de la población de peces?
\end{enumerate}

\end{document}