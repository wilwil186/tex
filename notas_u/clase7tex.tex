\documentclass[12pt]{book}

\usepackage[leqno]{amsmath} % Paquete para el manejo de expresiones matemáticas
\usepackage{amssymb,amsmath,latexsym} %Paquete para llamar símbolos matemáticos
\usepackage[mathscr]{euscript}
\usepackage{graphicx} %paquete para el manejo de transformaciones geométricas de imagénes
\usepackage{color} %paquete para el manejo de color en textos.
%\usepackage[utf8]{inputenc} %paquete para el manejo de caracteres acentuados
\usepackage[french,spanish]{babel} %paquete que genera documentos en diferentes idiomas
\usepackage{enumerate}
\usepackage{multicol} % Paquete para modificar el número de columnas
\usepackage{layout} % Paquete  para revisar los valores de 
\usepackage{verbatim}
\pagestyle{myheadings}  % Estilo de página
%\pagenumbering{arabic} % Estilo de numéración
\hoffset1cm
\newcounter{Teorema}
\newcommand{\Teorema}{\stepcounter{Teorema}{\bf Teorema \theTeorema.} }


\DeclareMathOperator{\arcsec}{arcsec} % Creación de nuevos comandos en latex
\DeclareMathOperator{\Var}{Var}

\DeclareMathOperator*{\Hom}{Hom}

\newcommand{\tto}{\longrightarrow}
\newcommand{\N}{\ensuremath{\mathbb{N}}}
\newcommand{\parcial}[2]{\frac{\partial#1}{\partial#2}}
\newcommand{\Norma}[1]{\Vert#1\Vert}
\newcommand{\upla}[2]{(#1_1,#1_2,$\ldots$,#1_{#2})}
	\newcommand{\uplamatrix}[3]{
	\begin{pmatrix}  
		#1_{11} &  #1_{12}  & \cdots  &  #1_{1 #3} \\   
		#1_{21} &  #1_{22} &  \cdots  & #1_{2 #3}  \\ 
		\vdots     &  \vdots      & \vdots  & \vdots \\
	#1_{#2 1} & #1_{#2 2} &  \cdots &  #1_{#2 #3}
\end{pmatrix}}
	\newcommand{\kupla}[3][k]{
	(#2_{#3},$\ldots$ #2_{#1})	
}


\setcounter{MaxMatrixCols}{15}

\decimalpoint
\begin{document}
\makeatletter	
\def\@roman#1{\romannumeral#1}
\makeatother
%\layout	% Muestra las medidas de las márgenes usadas en el documento.
\renewcommand{\baselinestretch}{1.5}% Controlar la distancia de espaciamiento entre renglones
\renewcommand{\thefootnote}{\arabic{footnote}} % Para hacer que Latex use diferentes simbolos para enumerar los píes de página.
\renewcommand{\refname}{Referencias bibliográficas}




\noindent\textbf{Definición de nuevos comandos}

\begin{verbatim}
	\newcommand{\nombre}{definición}
\end{verbatim}
\begin{verbatim}
	\newcommand{\tto}{\longrightarrow}
\end{verbatim}
\begin{verbatim}
\newcommand{\N}{\ensuremath{\mathbb{N}}}
\end{verbatim}


$f:A\tto B$\\

\N\\


\vspace{1.5cm}
\noindent\textbf{Comandos con argumentos obligatorios}

\begin{verbatim}
	\newcommand{\nombre}[n]{definición}
\end{verbatim}

$\frac{\partial f}{\partial x}$

\begin{verbatim}
	\newcommand{\parcial}[2]{\frac{\partial#1}{\partial#2}}
\end{verbatim}

$\parcial{f}{x}$\\

$\parcial{z}{x_i}$

\begin{verbatim}
	\newcommand{\Norma}[1]{\Vert#1\Vert}
\end{verbatim}
 
$\Norma{u}$\\

$\Norma{u+v}$

\begin{verbatim}
	\newcommand{\upla}[2]{(#1_1,#1_2,$\ldots$,#1_{#2})}
\end{verbatim}

$\upla{a}{n}$ \\

$\upla{c}{m}$


\begin{verbatim}
	\newcommand{\uplamatrix}[3]{
	\begin{pmatrix}  
		#1_{11} &  #1_{12}  & \cdots  &  #1_{1 #3} \\   
		#1_{21} &  #1_{22} &  \cdots  & #1_{2 #3}  \\ 
		\vdots     &  \vdots      & \vdots  & \vdots \\
		#1_{#2 1} & #1_{#2 2} &  \cdots &  #1_{#2 #3}
\end{pmatrix}}
\end{verbatim}


$\uplamatrix{a}{m}{n}$\\

$\uplamatrix{c}{k}{p}$\\

\vspace{1.5cm}

\noindent\textbf{Comandos con argumentos opcionales}

\begin{verbatim}
	\newcommand{\nombre}[n][defecto]{definición}
\end{verbatim}


\begin{verbatim}
	\newcommand{\kupla}[3][k]{
(#2_{#3},$\ldots$ #2_{#1})	
}
\end{verbatim}


$\kupla{a}{m}$

$\kupla{a}{1}$

$\kupla[p]{a}{1}$


$\kupla[10]{a}{1}$

\vspace{1.5cm}

\noindent\textbf{Alineación y enumeración de fórmulas}


\begin{equation}
	f^{\prime}(x):=\lim\limits_{h\to 0}\frac{f(x+h)-f(x)}{x}
\end{equation}

\vspace{1.5cm}

\noindent\textbf{División de fórmulas con} \verb*|multline|

\begin{multline}
	\parcial {f}{u}(a,y)=\lim\limits_{h\tto 0}\frac{f(x+h a,y+hb)-f(x,y)}{h}=\\
	\lim\limits_{h\tto 0}\frac{f(x,y)-f(x,y+bh)}{h}+\frac{f(x+ah,y)-f(x,y)}{h}\\
	=a\parcial{f}{x}(x,y)+b\parcial{f}{y}(x,y)=\nabla f(x,y)\cdot (a,b)
\end{multline}

\begin{multline*}
	\parcial {f}{u}(a,y)=\lim\limits_{h\tto 0}\frac{f(x+h a,y+hb)-f(x,y)}{h}=\\
	\lim\limits_{h\tto 0}\frac{f(x,y)-f(x,y+bh)}{h}+\frac{f(x+ah,y)-f(x,y)}{h}\\
	=a\parcial{f}{x}(x,y)+b\parcial{f}{y}(x,y)=\nabla f(x,y)\cdot (a,b)
\end{multline*}



\noindent\textbf{Alineación con } \verb*|gather|

\begin{gather}
	A\cup B:=\{x\mid x\in A \text{ o } x\in B\}\\
	A\cap B:=\{x\mid x\in A \text{ y } x\in B\} \\
		A\setminus B:=\{x\mid x\in A \text{ y } x\notin B\}\\
	A^{c}:=\{x\mid x\notin A \} \\
		A\bigtriangleup  B:=A\setminus B\cup B\setminus A
\end{gather}




\begin{gather*}
	A\cup B:=\{x\mid x\in A \text{ o } x\in B\}\\
	A\cap B:=\{x\mid x\in A \text{ y } x\in B\} \\
	A\setminus B:=\{x\mid x\in A \text{ y } x\notin B\}\\
	A^{c}:=\{x\mid x\notin A \} \\
	A\bigtriangleup  B:=A\setminus B\cup B\setminus A
\end{gather*}


\begin{gather}
	A\cup B:=\{x\mid x\in A \text{ o } x\in B\}\\
	A\cap B:=\{x\mid x\in A \text{ y } x\in B\} \\
\notag	A\setminus B:=\{x\mid x\in A \text{ y } x\notin B\}\\ 
\notag A^{c}:=\{x\mid x\notin A \} \\
\notag	A\bigtriangleup  B:=A\setminus B\cup B\setminus A
\end{gather}


\noindent\textbf{Alineación con} \verb*|align|

\begin{verbatim}
    
\begin{align}
		A\cup B:=&\{x\mid x\in A \text{ o } x\in B\}\\
	A\cap B:=&\{x\mid x\in A \text{ y } x\in B\} \\
	A\setminus B:=&\{x\mid x\in A \text{ y } x\notin B\}\\
	A^{c}:=&\{x\mid x\notin A \} \\
	A\bigtriangleup  B:=&A\setminus B\cup B\setminus A
\end{align}

\end{verbatim}
\begin{align}
		A\cup B:=&\{x\mid x\in A \text{ o } x\in B\}\\
	A\cap B:=&\{x\mid x\in A \text{ y } x\in B\} \\
	A\setminus B:=&\{x\mid x\in A \text{ y } x\notin B\}\\
	A^{c}:=&\{x\mid x\notin A \} \\
	A\bigtriangleup  B:=&A\setminus B\cup B\setminus A
\end{align}


\begin{align*}
	\Norma{u+v}^{2}&=(u+v)\cdot(u+v)\\ 
	                                      &=u\cdot u+u\cdot v+v\cdot u+v\cdot v \\
	                                      &=\Norma{u}^{2}+2 u\cdot v+\Norma{v}^{2}\\
	                                      &\leq \Norma{u}^{2}+2 \Norma{u}\Norma{v}+\Norma{v}^{2}\\
	                                      &=\left(\Norma{u}+\Norma{v}\right)^{2}
\end{align*}



\begin{align*}
x &= ax+b & X&= uX+v & A&= aA+B\\
x' &= ax'+b & X' &= uX'+v & A' &= aA'+B'\\
y&= (1-a)y & Y&= (1-u)Y & B &= (1-a)B\\
y' &= (1-b)y' & Y' &= (1-v)Y' & B' &= (1-b)B'
\end{align*}


\begin{align*}
	a^{\prime}& =a^{\prime}*e    && (\text{Por la ley modulativa}) \\
	&= a^{\prime}*(a*a^{-1})   & & (\text{Por la ley cancelativa })\\
	&=(a^{\prime}*a)*a^{-1} & & (\text{Por la ley asociativa}) \\
	&=e*a^{-1}  && (\text{Por la ley cancelativa})\\
	&=a^{-1}
\end{align*}

\newpage

\noindent Puesto que la igualdad
\begin{align*}
	(fg)^{\prime}&=f^{\prime}g+fg^{\prime}\\
	\intertext{se puede escribir como}
	fg^{\prime}&=(fg)^{\prime}-f^{\prime}g\\
	\intertext{se concluye}
	\int fg^{\prime}&=\int (fg)^{\prime}-\int f^{\prime}g
\end{align*}





\noindent\textbf{Alineación con } \verb*|split|


\begin{equation*}
	\begin{split}
				A\cup B:=&\{x\mid x\in A \text{ o } x\in B\}\\
		A\cap B:=&\{x\mid x\in A \text{ y } x\in B\} \\
		A\setminus B:=&\{x\mid x\in A \text{ y } x\notin B\}\\
		A^{c}:=&\{x\mid x\notin A \} \\
		A\bigtriangleup  B:=&A\setminus B\cup B\setminus A
	\end{split}
\end{equation*}


\begin{equation*}
\begin{split}
&\frac{f(a+h)-f(a)}h-\frac{\partial f}{\partial x}(a)=\\
&\phantom{f(a+h)}\frac{u(a+h)-u(a)-d_au(h)}h +i\frac{v(a+h)-v(a)-
d_av(h)}h
\end{split}
\end{equation*}


\end{document}