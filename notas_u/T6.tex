\documentclass[11pt]{book}

\usepackage[reqno]{amsmath} % Paquete para el manejo de expresiones matemáticas [reqno], [leqno] y [fleqn]
\usepackage{amsthm}
\usepackage{amssymb,amsmath,latexsym} %Paquete para llamar símbolos matemáticos
\usepackage{amsfonts}
\usepackage[mathscr]{euscript}
\usepackage{graphicx} %paquete para el manejo de transformaciones geométricas de imagénes
\usepackage{color} %paquete para el manejo de color en textos.
%\usepackage[utf8]{inputenc} %paquete para el manejo de caracteres acentuados
\usepackage[french,spanish]{babel} %paquete que genera documentos en diferentes idiomas
\usepackage{enumerate}
\usepackage{multicol} % Paquete para modificar el número de columnas
\usepackage{layout} % Paquete  para revisar los valores de 
\usepackage{verbatim}
\pagestyle{myheadings}  % Estilo de página
%\pagenumbering{arabic} % Estilo de numéración
\hoffset1cm
\newcounter{Teorema}
\newcommand{\Teorema}{\stepcounter{Teorema}{\bf Teorema \theTeorema.} }
\usepackage{ tipa }



\DeclareMathOperator{\arcsec}{arcsec} % Creación de nuevos comandos en latex
\DeclareMathOperator{\Var}{Var}

\DeclareMathOperator*{\Hom}{Hom}

\setcounter{MaxMatrixCols}{15}

\allowdisplaybreaks % Control de cambios de página en alineaciones

\decimalpoint

%\renewcommand{\theequation}{\thesection.\arabic{equation}}
\numberwithin{equation}{section}
%\renewcommand{\theequation}{\theparentequation\arabic{equation}}


%% Nuevos teoremas

\theoremstyle{plain}  % Requiere el paquete amsthm

\newtheorem{thm}{Teorema}[section]
\newtheorem{Corol}[thm]{Colorario}
\newtheorem{Prop}[thm]{Proposición}
\newtheorem{axiom[thm]}{Axioma}
\newtheorem{conj}{Conjetura}
\newtheorem{Def}{Definición}[chapter]
\newtheorem{Ej}{Ejemplo}[chapter]
\newtheorem{notacion}[Def]{Notación}
\newtheorem{nota}[Def]{Nota}

\renewcommand{\qedsymbol}{$\heartsuit$}
\providecommand{\abs}[1]{\lvert#1\rvert} %valor absoluto
\providecommand{\norm}[1]{\lVert#1\rVert} %norma
\pagestyle{empty}
\begin{document}

\chapter{}

\noindent \textbf{{\huge CAMPOS VECTORIALES E\\
INTEGRALES DE LÍNEA}}

\section{Campos vectoriales}

\begin{Def}
{\it
Sea $D$ una región abierta en $\mathbb{R}^{n}$. Un \textbf{Campo vectorial} en $D$ es una apliclación F que a cada punto $p \in D$ le asigna un vector $F(p) \in \mathbb{R}^{n}$, con $m>1$. Si denotamos por $\vec{x}$ el vector posición de p, entonces podemos describir el campo vectorial por la función vectorial. Las funciones $f_{i}: D \to \mathbb{R}$ se llaman \textbf{componentes} del campo F. Si las componente $f_{i}$ son derivables decimos que el campo vectorial F es derivable. 

}
\end{Def}
{\it
\begin{Ej}

Sea $D \subseteq \mathbb{R}^{n}$ una región abierta y $f: D \to \mathbb{R}$ una función derivable. Entonces el campo  vectorial 


\begin{align*}
	F(\vec{x})&=\bigtriangledown f(x_{1},\cdots,x_{n}\\
	&=(\frac{\partial f}{\partial x_{1}}(x_{1},\cdots,x_{n}),\cdots,\frac{\partial f}{\partial x_{n}})
\end{align*}

\end{Ej}
\noindent se llama \textbf{Campo vectorial gradiente}. Los vectores del campo gradiente son ortogonales a las superficies de nivel de la función f.
}
\vfill
En muchos casos para entender un campo vectorial necesitamos dibujarlo, y esto no resulta una tarea no muy corta. Podemos usar para esta tarea ciertas líneas de campo, un concepto muy importante
\newpage


{\it
\begin{Def}
Una \textbf{línea de campo} de un campo vectorial $F(\vec{x})$ es una curva $\vec{r}(t)$, tal que
\begin{equation*}
    \frac{d\vec{r}}{dt} = F(\vec{r}(t)).
\end{equation*}
\end{Def}
Geométricamente significa que el campo vectroial F es tangente a sus líneas de campo en cada punto. 
Analicamente, las línea de campo de un campo vectorial $F(x_{1}, \cdots, x_{n}$ con componentes $f_{1},f_{2} \cdots f_{n}$ son las soluciones del sistema de ecuaciones diferenciales 
}
\begin{equation*}
\begin{cases}
\frac{dx_{1}}{dt}(t)=f_{1}(x_{1}(t), \cdots , x_{n}(t)) \\
\frac{dx_{2}}{dt}(t)=f_{2}(x_{1}(t), \cdots , x_{n}(t)) \\ 
\vdots \\ 
\frac{dx_{n}}{dt}(t) = f_{n}(x_{1}(t),\cdots , x_{n}(t)
\end{cases}
\end{equation*}

\section{Intregales de Línea}
{\it 
\begin{Def}
(Intregal de línea sobre un campo escalar). Sea $f: D \to \mathbb{R}$ una función continua, donde $D \subseteq \mathbb{R}^{n}$ es una región abierta. Y sea $\gamma$ una curva suave en $D \subseteq \mathbb{R}^{n}$ con una ecuación dada por una función vectorial $\vec{r}:[a,b] \to D$, $\vec{r} = \vec{r}(s)$, donde s es el parámetro de longitud de arco y b-a es la longitud de la curva $\gamma$. Entonces $f(\vec{r}(s))$ es una función real continua sobre el dominio $[a,b]$
\end{Def}
La \textbf{intregal de línea} de la función f a lo largo de la curva $\gamma$, donde $\gamma$ esta parámetrizada en términos del parámetro natura s (longitud de arco), $s\in [a,b]$, es 
}
\begin{equation*}
    \int_{\gamma}fds=\int_{a}^{b}f(\vec{r(s)}ds
\end{equation*}
{\it
Si tenemos una ecuación $\vec{r} = \vec{p}(t). t \in [c,d]$ de la curva dada $\gamma$, respecto a un parámetro arbitrario t al parámetro natural s aplicando la fórmula }
\begin{equation*}
    s = \int_{c}^{t} \norm{p'(u)}du \Rightarrow ds = \norm{p'(t)}dt
\end{equation*}
\newpage
\noindent entonces la intregal de línea a lo largo de $\gamma$, de cualquier parametrización $\vec{p}(t)$ de $\gamma$, es
\begin{equation*}
    \int_{\gamma}fds = \int_{c}^{d} f(\vec{p}(t)) \norm{\vec{p}'(t)}dt
\end{equation*}
\begin{Ej}
La intregral de línea de la función f(x,y)=xy a lo largo de la circunferencia con centro en el origen y radio $r>0$ es 
\end{Ej}

\begin{align*}
    \int_{\gamma} f ds &= \int_{0}^{2\pi} (r \cos{t})(r \sin{t})\norm{(-r \sin{t},r \cos{t})}dt \\
    &= \int_{0}^{2\pi} r^{3} \cos{t} \sin{t} dt \\
    &= 0
\end{align*}
\begin{Def}
(Intregal de línea sobre un campo vectorial). Sea $F: D \to \mathbb{R}^{n}$ un campo vectorial continuo, donde $D \subseteq \mathbb{R}^{n}$ una región abierta. Sea $\gamma$ una curva suave en $D$ con una ecuación dada por una ecuación vectorial $\vec{r}:[a,b] \to D$, $\vec{r}=\vec{r}(t)$. Entonces $F(\vec{r}(t))$ es una función vectorial continua sobre el dominio $[a,b]$.
\end{Def}
\noindent La \textbf{Intregal de línea} del campo vectorial F a lo largo de la curva $\gamma$ es 
\begin{equation*}
    \int_{\gamma} F \cdot d\vec{r} = \int_{a}^{b} F(\vec{r}(t)) \cdot \vec{r}'(t)dt
\end{equation*}

\begin{Ej}
la intregal de línea del campo vectorial $F(x,y)=(x+y,y)$ a lo largo de la curva con parametrización $\vec{r}(t)=(\cos{t},\sin{t}), t \in [0,2\pi]$ es 
\end{Ej}
\begin{align*}
\int_{\gamma} F \cdot d\vec{r} &= \int_{0}^{2\pi} (\cos{t} +\sin{t},\sin{t}) \cdot (- \sin{t}, \cos{t})dt \\
&= \int_{0}^{2 \pi} - \sin^{2} dt \\ 
&= \frac{1}{2} \int_{0}^{2 \pi} \cos{2t}-1dt \\ 
&= -\pi
\end{align*}
\begin{notacion}
la notación
\end{notacion}
\begin{equation*}
    \int_{C} P(x,y) dx + Q(x,y)dy,
\end{equation*}
\newpage
\noindent donde C es una curva del plano, representa la integral de lnea sobre C del campo vectorial en el plano
\begin{equation*}
	F(x,y) = (P(x; y);Q(x; y)).
\end{equation*}
\noindent Igualmente la notacion
\begin{equation*}
	\int_{C} P(x,y,z) dx + Q(x,y,z)dy + Q(x,y,z)dz,
\end{equation*}
\noindent donde C es una curva en el espacio, representa la integral de linea sobre C
del campo vectorial en el espacio
\begin{equation*}
	F(x, y, z) = (P(x, y, z),Q(x, y, z),R(x, y, z))
\end{equation*}
	\section{El Teorema de Green}
{\it 
	\begin{Def}
		(Campo vectorial conservativo). Un campo vectorial Sea $F:D \,\subseteq \mathbb{R}^{n} \to \mathbb{R}^{n}$, es un \textbf{campo vectorial conservativo} si existe un campo escalar $f:D \, \subseteq \mathbb{R}^{n} \to \mathbb{R}$ tal que el gradiente su gradiente sea el campo vectorial dado, es decir
	\end{Def}
\begin{equation*}
		\bigtriangledown f = F
\end{equation*}
\\
	\noindent a la funcion f la llamaremos el \textbf{potencial escalar.} 
\begin{thm} (Teorema fundamental para integrales de lnea). Sea C una curva suave por partes, definida por $\vec{r}(t)$	 para $a \leq b\leq \vec{r}(t)\in D$. para cada $t \in [a; b]$  y $ D \subseteq F: \mathbb{R}^{n} \to \mathbb{R}$ es una funcon derivable con gradiente continuo sobre C, entonces
		\end{thm}
	\begin{equation*}
			\int_{C} \bigtriangledown f \,.\,d\vec{r} (t) = f(\vec{r}(b)) - f(\vec{r}(a))
	\end{equation*}
	\newpage
	\noindent Prueba.
		\begin{align*}
		\int_{C} \bigtriangledown f.d\vec{r} (t) &= \int_{a}^{b} 
		\begin{pmatrix}					
			\frac{\partial f}{\partial x_{}} (x(t), 	\frac{\partial f}{\partial y_{}} (x(t) \end{pmatrix}.\, (x'(t); y'(t)) dt \\		
		&=  \int_{a}^{b} 				
		\frac{\partial f}{\partial x_{}} (x(t); y(t)) 	x'(t) +	\frac{\partial f}{\partial y_{}}y'((x(t), y(t))dt \\
		&= 	\int_{a}^{b}\frac{\partial f \circ r}{\partial x_{}}dt \\
			&= f(r(t)) \textpipe_{a}^{b}\\
		&= f(r(b))- f(r(a))  \\ 
	    \end{align*}
	\begin{flushright}
			$\Box$
	\end{flushright}
	
\begin{Def}
	Sea $F:D \subseteq F: \mathbb{R}^{n} \to \mathbb{R}^{n}$ un campo vectorial y C1 y C2 dos curvas suaves contenidas en D definidas por $\vec{r_1}$   y  $ \vec{r_2}$  curvas suaves contenidas en D definidas por  $\int_{C} F \,.\,\,d\vec{r}$ es \textbf{ independiente de la trayectoria}  si y solo si
\end{Def}
	\begin{equation*}
	\int_{C_1} F\,.\,d\vec{r_1} = \int_{C_2} F\,.\,d\vec{r_2} 
\end{equation*}
\\
\begin{Def}
	La integral 
	$\int_{C} F \,.\,\,d\vec{r}$  es independiente de la trayectoria $\oint_{c} F \,.\,\,d\vec{r} = 0$ si y solo si para toda curva cerrada C contenida en el dominio	de F.
\end{Def}
\begin{Def}
	Una region R se llama \textbf{conexa} si para cualquier par	de puntos $a,b \in \mathbb{R}$  existe una trayectoria continua $\vec{r} :[0,1] \rightarrow \mathbb{R}$ tal que  $r(0) = a$ y $r(1) = b$
\end{Def}
\begin{Def}
	Sean F un campo vectorial continuo sobre una region abierta y conexa D y C cualquier curva contenida en D. Si $\int_{C} F \,.\,\,d\vec{r}$ es independiente de la trayectoria si y solo si F es un campo conservativo. 
\end{Def}
\begin{Def}
	Un conjunto R se llama \textbf{ simplemente conexo} si no
tiene huecos.
\end{Def}
}
\newpage
\begin{thm}
Contenidos $\cdots$ Sea $F: D \subseteq \mathbb{R}^{n} \to \mathbb{R}^{n}$, $F=(f_{1},f_{2},\cdots,f_{n})$ un campo continuo abierto simplemente conexo $D$. entonces $F$ es un campo vectorial conservativo si y solo si la matriz de derivadas parciales de $F$
\begin{equation*}
    DF=[\frac{\partial f_{i}}{\partial x_{j}}]_{n\times n}
\end{equation*}
es simétrica, es decir, 
\begin{equation*}
    \frac{\partial f_{i}}{\partial x_{j}} = \frac{\partial f_{j}}{\partial x_{i}}, \text{ para cada } 1\leq i, j \leq n 
\end{equation*}
\end{thm}
\begin{thm}
(Teorema de Green). Sea $\mathbb{C}^{+}$ una curva cerrada simple, orientada positivamente, suave por partes y $D$ una región del plano $xy$ acotada por $\mathbb{C}$. Sea $F(x,y) = P(x,y)\text{\textbf{i}}+Q(x,y)\text{\textbf{j}}$ un campo vectorial derivable en un abierto $\Omega$, donde $\Omega$ contiene tanto a $D$ como a su frontera $\mathbb{C}$. Entonces
\begin{equation*}
    \oint_{\mathbb{C}} Pdx+Qdy = \iint_{D} \frac{\partial Q}{\partial x} -\frac{\partial P}{\partial y} dA
\end{equation*}
\end{thm}
	\begin{thm}[Teorema de Green para Regiones Múltiplemente Conexas.]
	Sean $C_1,C_2,\dotsc,C_n$ curvas de Jordan suaves a trozos que tienen las siguientes propiedades:
	\begin{itemize}
		\item Dos cuales quiera de esas curvas no se cortan.
		\item Todas las curvas $C_2,C_3,\dotsc,C_n$ están en el interior de $C_1$.
		\item La curva $C_i$ no está en el interior de la curva $C_j$ para cada $i\neq j$, $i>1$, $j>1$. Designemos por $R$ la región que consiste en la reunión de $C_1$ con la porción del interior de $C_1$ que no está dentro de cualquiera de las curvas $C_2,C_3,\dotsc,C_n$. Sean $P$ y $Q$ derivables con continuidad en un conjunto $S$ que contiene a $R$. Tenemos entonces la siguiente identidad:
		$$\iint\limits_{R}\frac{\partial Q}{\partial x}-\frac{\partial P}{\partial y}dxdy=\oint_{C_1}(Pdx+Qdy)-\sum_{i=2}^{n}\oint_{C_i}(Pdx+Qdy)$$
	\end{itemize}
\end{thm}
\begin{Def}[Rotación de un campo vectorial.] Sea $F(x,y,z)=P(x,y,z){\bf{i}}+Q(x,y,z){\bf j}+R(x,y,z){\bf k}$ un campo vectorial sobre $\mathbb{R}^{3}$, tal que todas las derivadas parciales de $P$, $Q$ y $R$ existen. Entonces el \textbf{rotacional} de $F$, que se denota por $\mathrm{Rot}\ F$ o $\nabla\times F$, es un campo vectorial sobre $\mathbb{R}^{3}$ definido por 
	\begin{align*}
		\nabla\times F&=\left|\begin{matrix}
			{\bf i}&{\bf j}&{\bf k}\\
			\frac{\partial}{\partial x}&\frac{\partial}{\partial y}&\frac{\partial}{\partial z}\\
			P&Q&R
		\end{matrix}\right|\\
	&=\Bigl(\frac{\partial R}{\partial y}-\frac{\partial Q}{\partial z}\Bigl){\bf i}+\Bigl(\frac{\partial P}{\partial z}-\frac{\partial R}{\partial x}\Bigl){\bf j}+\Bigl(\frac{\partial Q}{\partial x}-\frac{\partial P}{\partial y}\Bigl){\bf k}
	\end{align*}
Para un campo vectorial $F$ definido en el plano $xy$, o en una región de este, con componentes $F(x,y)=P(x,y){\bf i}+Q(x,y){\bf j}$, el rotacional es,
$$\nabla\times F=\Bigl(\frac{\partial Q}{\partial x}-\frac{\partial P}{\partial y}\Bigl){\bf k}$$
\end{Def}
\begin{thm}[Primera Forma Vectorial Del teorema de Green] Sea $C^{+}$ una curva cerrada simple (Jordan), orientada positivamente, suave por partes y $D$ una región del plano $xy$ acotada por $C$. Sea $F(x,y)=P(x,y){\bf i}+Q(x,y){\bf j}$ un campo vectorial derivable en $\Omega$, donde $\Omega$ es una región del plano que contiene a $D$ y su frontera $C$, entonces
	$$\oint_CF\cdot d\overrightarrow{r}=\iint\limits_D\nabla\times F\cdot{\bf k}dA$$
\end{thm}
\begin{Def}[Divergencia de un campo vectorial] 
	Sea $F(x,y,z)=P(x,y,z){\bf i}+Q(x,y,z){\bf j}+R(x,y,z){\bf k}$ un campo vectorial tal que todas las derivadas parciales de $P$, $Q$ y $R$ existan. Entonces la \textbf{divergencia} de $F$, que se denota por $\mathrm{div}\ F$ o $\nabla\cdot F$ es un campo escalar sobre $\mathbb{R}^{3}$ definido por
	$$\nabla\cdot F=\frac{\partial P}{\partial x}+\frac{\partial Q}{\partial y}+\frac{\partial R}{\partial z}$$
\end{Def}
\begin{thm}[Segunda Forma Vectorial del Teorema de Green] Sea $C$ una curva cerrada simple, suave por partes orientada positivamente y $D$ una región en el plano $xy$ acotada por $C$. Sea $F(x,y)$ un campo vectorial sobre $\mathbb{R}^{2}$ derivable en $\Omega$, donde $\Omega$ es una región del plano que contiene a $D$ y su frontera $C$. Entonces
	$$\oint_CF\cdot Nds=\iint\limits_D\nabla\cdot FdA$$
	donde $N$ es el vector normal unitario de $C$ en la dirección hacia afuera de la región acotada por $C$.
\end{thm}
\begin{thm}
Sea $F(x,y,z)$ un campo vectorial derivable definido en una
región abierta, entonces
$$\nabla\cdot(\nabla\times F)=\mathrm{div}(\mathrm{Rot}(F))=0$$
\end{thm}
\begin{proof}[Prueba]
\begin{align*}
\nabla\cdot(\nabla\times F)&=\mathrm{div}(\mathrm{Rot}(F))\\
&=\mathrm{div}\left|\begin{matrix}
	{\bf i}&{\bf j}&{\bf k}\\
	\frac{\partial}{\partial x}&\frac{\partial}{\partial y}&\frac{\partial}{\partial z}\\
	f_1&f_2&f_3
\end{matrix}\right|\\
&=\mathrm{div}[(\frac{\partial f_3}{\partial y}-\frac{\partial f_2}{\partial z}){\bf i}+(\frac{\partial f_1}{\partial z}-\frac{\partial f_3}{\partial x}){\bf j}+(\frac{\partial f_2}{\partial x}-\frac{\partial f_1}{\partial y}){\bf k}]\\
&=\frac{\partial^{2} f_3}{\partial x\partial y}-\frac{\partial^{2} f_2}{\partial x\partial z}+\frac{\partial^{2} f_1}{\partial y\partial z}-\frac{\partial^{2} f_3}{\partial y\partial x}+\frac{\partial^{2} f_2}{\partial z\partial x}-\frac{\partial^{2} f_1}{\partial z\partial y}\\
&=\frac{\partial^{2} f_1}{\partial y\partial z}-\frac{\partial^{2} f_1}{\partial z\partial y}+\frac{\partial^{2} f_2}{\partial z\partial x}-\frac{\partial^{2} f_2}{\partial x\partial z}+\frac{\partial^{2} f_3}{\partial x\partial y}-\frac{\partial^{2} f_3}{\partial y\partial x}\\
&=0
\end{align*}
\end{proof}
\begin{thm}[Área de una región plana] El área de una región $D$ en el plano $xy$ acotada por la curva simple $C$ es 
	$$A(D)=\frac{1}{2}\oint_{C^{+}}xdy-ydx$$
\end{thm}
\begin{proof}[Prueba] Por el teorema de Green la integral de líne del enunciado del teorema se puede expresar:
\begin{align*}
\frac{1}{2}\oint_{C^{+}}xdy-ydx&=\frac{1}{2}\iint_D\left(\frac{\partial x}{\partial x}-\frac{\partial(-y)}{\partial y}\right)dA\\
&=\frac{1}{2}\iint_D2dA\\
&=\iint_DdA\\
&=A(D).
\end{align*}

\end{proof}
\end{document}