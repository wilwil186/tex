\documentclass{article}
\usepackage[utf8]{inputenc}
\usepackage{hyperref}
\usepackage{graphicx}
\usepackage{geometry}
\usepackage{multicol}
\usepackage{amsmath}
\usepackage{amsthm,amsfonts,amssymb}%paquetes AMS
 
\begin{document}
    \section{Introducción a las ecuaciones diferenciales}
    \subsection{Importancia de las ecuaciones diferenciales}
    La importantancia de las ecuaciones diferenciales son la expresión 
    matemática de aquéllas leyes fundamentales 
    de la naturaleza que son formuladas en términos de razones de cambio
    de cantidades variables.
    Estas leyes surgen en diversos campos de aplicaclión por ejemplo:
    \begin{enumerate}
        \item movimiento newtonianos y no newtonianos.
        \item difusión de calor
        \item elasticidad
        \item estudio de fluidos
        \item crecimiento de poblaciones 
        \item economia
        \item ecologia
        \item ciencias sociales 
    \end{enumerate}
    y muchos otros.
    \subsection{Estudio de las soluciones de una ecuación diferencial}
    el análisis de las soluciones tanto teóricas como numéricas de las ecuaciones
    diferenciales es fundamental para la investigación y compresión de los fenómenos naturales
    o teóricos que representan; tanto el estudio de las propiedades generales de las ecuaciones, que nos 
    proporciona información valiosa que se puede aplicar en cada cado particular: como su complemento, la resolución numérica,
    que permite contruir soluciones de una ecuación, paso a paos en la vecindad de un punto y con cierto margen de error.
    \subsection{Ecuación diferencial}
    En términos generales, una ecuación diferencial es una ecuación que 
    contiene derivadas de las funciones que aparecen en la ecuación.
    \\
    \textbf{Ejemplo:}
    \\
    
\end{document}