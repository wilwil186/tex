\documentclass[12pt]{book}
\usepackage[utf8]{inputenc}
\usepackage[spanish]{babel}
\usepackage{hyperref}
\usepackage{graphicx}
\usepackage{geometry}
\usepackage{multicol}
\usepackage{amsmath}
\usepackage{amsfonts}
\usepackage{amssymb}
\usepackage{multicol}
\usepackage{color}
\usepackage{bbding}
\usepackage{stackrel}
\providecommand{\abs}[1]{\lvert#1\rvert}
\providecommand{\norm}[1]{\lVert#1\rVert}
\pagestyle{empty}

\begin{document}

\begin{enumerate}
    \item Defina los siguientes comandos:
    \begin{itemize}
        \item Defina el comando \texttt{\textbackslash limit\{f\}\{a\}}, con dos argumentos, que impprima:
            \begin{equation*}
                f'(a) = \lim_{h \to 0}\frac{f(a+h)-f(a)}{h}
            \end{equation*}
        \item Defina el comando \texttt{\textbackslash prod[k]\{a\}\{n\}}, con tres argumentos uno de ellos opcional, que imprima 
        \begin{equation*}
            \prod_{i=k}^{n}a_{i}=a_{k}a_{k+1}\cdots a_{n} 
        \end{equation*}
        \item Defina el comando \texttt{\textbackslash sumar[k]\{a\}\{n\}}, con tres argumentos, uno de ellos opcional, que imprima
        \begin{equation*}
            \sum_{i=k}^{n}=a_{k}+a_{k+1}+\cdots+a_{n}
        \end{equation*}
        \item Defina el comando \texttt{\textbackslash D2\{F\}\{n\}\{a\}}, con tres argumentos que imprima
        \begin{equation*}
        \begin{pmatrix}
            \frac{\partial f^{2}}{ax_{1}x_{1}} (a) & \frac{\partial f^{2}}{ax_{1}x_{1}} (a) &  \cdots & \frac{\partial f^{2}}{ax_{1}x_{1}} (a)  \\
            \frac{\partial f^{2}}{ax_{1}x_{1}} (a) & \frac{\partial f^{2}}{ax_{1}x_{1}} (a)& \cdots & \frac{\partial f^{2}}{ax_{1}x_{1}} (a) \\ 
            \vdots & \vdots & \ddots & \vdots \\ 
            \frac{\partial f^{2}}{ax_{1}x_{1}} (a) & \frac{\partial f^{2}}{ax_{1}x_{1}} (a) & 
            \cdots & \frac{\partial f^{2}}{ax_{1}x_{1}} (a)
        \end{pmatrix}
    \end{equation*}
    \item Realize un código en \LaTeX que imprima: \\
    
     Una \textbf{partición} P de un intervalo [a,b] es un conjunto 
    \begin{equation}
    P = \{x_{0},x_{1},x_{2} \cdots, x_{n}\}
    \end{equation}
    tal que

    \newpage
    \item $x_{i}<x_{i+1}$ para cada $i=0,1, \cdots, n-1$,
    \item $x_{0}=a, y x_{n}=b.$
    \end{itemize}
\end{enumerate}
Para una partición $P=\{x_{0}<x_{1}< \cdots < x_{n}\}$ de un intervalo cerrado [a,b] y una función $f$ definida y acotada en [a,b], llamanos a la suma 
\begin{equation}
    \sum_{i=1}^{n}f(x_{i}^{*})\triangle x
\end{equation}
tal que
\begin{itemize}
    \item $x_{i}^{*} \in [x_{i-1},x_{i}]$ para cada $i = 1, \cdots, n$ y 
    \item $\triangle x = \stackbin[1 \leq i \leq n]{}{\min}\abs{x_{i}-x_{i-1}}$
\end{itemize}
una \textbf{suma de Riemann} de la funciń $f$ sobre el intervalo [a,b] y bajo la partición P. Además , decimos que una función $f$ es \textbf{Riemann integrable} en el intervalo [a,b], si 
\begin{equation}
    \lim_{\triangle x \to 0}\sum_{i=1}^{n}f(x_{i}^{*})\triangle x
\end{equation}
existe es un número real, es decir, el valor al cual tiende este limite no depende de la forma en que se seleccionen las particiones P sobre [a, b] y ni como se eligen los valores $x_{i}^{*}$.
 \\
 En el caso en el que la función $f$ sea integrable en el intervalo [a,b] escribimos
 \begin{equation}
     \int_{a}^{b} f(x) dx = \lim_{\triangle x to 0} \sum_{i}^{n}f(x_{i=1}^{*})\triangle x
\end{equation}
donde el lado derecho se lee: `` la integral de $f (x)$ de a a b con respecto
a x''.
\\
Notemos que no todo función real definida y acotada en un intervalo es Riemann intregable, como ocurre con la función 
\begin{equation*}
    f(x) = 
    \begin{cases}
    1 \text{ si } x \in \mathbb{Q} \\
    0  \text{ en otro caso}
    \end{cases}
\end{equation*}

\newpage
En este caso si restringimos las particiones $P=\{x_0<x_1<\cdots<x_n\}$ tal que $\Delta x=\frac{b-a}{n}$, y seleccionamos siempre $x_i^{*}\in\mathbb{Q}$ tenemos
\begin{align*}
    \lim_{\Delta x\to 0}\sum_{i=1}^{n}f(x_i^{})\Delta x&=\lim_{\Delta x\to 0}\sum_{i=1}^{n}f(x_i^{})\frac{b-a}{n}\\
    &=\lim_{\Delta x\to 0}\sum_{i=1}^{n}\frac{b-a}{n}&\text{(Dado que $x_i^{*}\in\mathbb{Q}$)}\\
    &=\lim_{\Delta x\to 0}n\frac{b-a}{n}\\
    &=\lim_{\Delta x\to 0}b-a\\
    &=b-a
    \end{align*}
Por otro lado, si seleccionamos $x_i^{*}\notin\mathbb{Q}$, entonces
\begin{align*}
	\lim_{\Delta x\to 0}\sum_{i=1}^{n}f(x_i^{})\Delta x&=\lim_{\Delta x\to 0}\sum_{i=1}^{n}f(x_i^{})\frac{b-a}{n}\\
	&=\lim_{\Delta x\to 0}\sum_{i=1}^{n}0\times\frac{b-a}{n}&\text{(Dado que $x_i^{*}\notin\mathbb{Q}$)}\\
	&=\lim_{\Delta x\to 0}0\\
	&=0
\end{align*}
Lo que demuestra que la función no es Riemann integrable.\\
\vfill
Algunas propiedades de la integral de Riemann son
\begin{align}
	\int_{a}^{b}cdx&=c(b-a)\\
	\int_{a}^{a}f(x)dx&=0\\
	\int_{a}^{b}f(x)dx&=-\int_{b}^{a}f(x)dx\\
	\int_{a}^{b}f(x)dx&=\int_{a}^{c}f(x)dx+\int_{c}^{b}f(x)dx&\text{(Para cada $c \in\left[a,b\right]$)}
\end{align}
\newpage
Si $g(x)dx\leq f(x)$ para cada $x\in\left[a,b\right]$, entonces

\begin{align}
	\int_{a}^{b}g(x)dx\leq\int_{a}^{b}f(x)dx
\end{align}
Por último, si $f$ es una función continua en $\left[a,b\right]$, entonces se garantiza que existe $c \in\left[a,b\right]$ tal que
\begin{align}
	\int_{a}^{b}cdx&=f(c)(b-a)
\end{align}
En efecto; existen dos números reales $m$ y $M$ tal que para cada $x \in\left[a,b\right]$ se tiene:

$$m\leq f(x)\leq M$$

Por las propiedades (5) y (9)
$$m(b-a)\leq\int_{a}^{b}f(x)dx\leq M(b-a)$$

Dividiendo las desigualdades por $(b-a)$
$$m\leq\frac{\int_{a}^{b}f(x)dx}{b-a}\leq M$$

por el teorema del valor medio existe $c \in\left[a,b\right]$ tal que
$$f(c)=\frac{\int_{a}^{b}f(x)dx}{b-a}$$
de lo que se sigue lo que se quiere probar.\\

Para que una función real $f$ sea Riemann integrable en un intervalo $\left[a,b\right]$ es necesario y suficiente que la función $f$ tenga solo un número enumerable de discontinuidades en el intervalo $\left[a,b\right]$. De hecho si $f(x)$ es una función continua $\left[a,b\right]$, entonces la función
$$F(x)=\int_{a}^{x}f(t)dt$$
\newpage
es derivable en $(a,b)$ y $F'(x)=f(x)$ para cada $x\in (a,b)$. En efecto

\begin{align*}
	F'(x)&=\lim_{h\to 0}\frac{F(x+h)-F(x)}{h}\\
	&=\lim_{h\to 0}\frac{\int_{a}^{x+h}f(t)dt-\int_{a}^{x}f(t)dt}{h}\\
\intertext{Por la propiedad (7)}
&=\lim_{h\to 0}\frac{\int_{a}^{x+h}f(t)dt+\int_{x}^{a}f(t)dt}{h}\\
\intertext{Por la propiedad (8)}
&=\lim_{h\to 0}\frac{\int_{x}^{x+h}f(t)dt}{h}\\
\intertext{Por la propiedad (10)}
&=\lim_{h\to 0}\frac{(x+h-x)f(x_h)}{h}\\
&=\lim_{h\to 0}\frac{hf(x_h)}{h}\\
&=\lim_{h\to 0}f(x_h)
\intertext{Como $x_h\to x$, cuando $h\to 0$ y $f(x)$ es continua en $[a,b]$, entonces}
&=f(x).
\end{align*}

\end{document}
