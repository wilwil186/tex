\documentclass[12pt]{book}
\usepackage[utf8]{inputenc}
\usepackage[spanish]{babel}
\usepackage{hyperref}
\usepackage{graphicx}
\usepackage{geometry}
\usepackage{multicol}
\usepackage{amsmath}
\usepackage{multicol}
\usepackage{color}
\usepackage{bbding}



\pagestyle{myheadings}

\begin{document} 
%%%%%%%%%%%%%%%%%%%%%%%%%%%%%%%%%%%%%%%%%%%%%%%%%%%%%%%%%%%%%%%%%%%%%%%%%%%%%%%%%%%%%%%%%%%%%%%%%%%%%%%%%%%%%%%%%%%%%%%%%%%%%%%%%
\chapter*{Introducción}
%%%%%%%%%%%%%%%%%%%%%%%%%%%%%%%%%%%%%%%%%%%%%%%%%%%%%%%%%%%%%%%%%%%%%%%%%%%%%%%%%%%%%%%%%%%%%%%%%%%%%%%%%%%%%%%%%%%%%%%%%%%%%%%%%

\begin{multicols}{3}
{\scshape \normalsize
Una  \hspace*{0.5cm} {\bf introducción}
es la parte inicial  de un discurso \ en forma de síntesis para identificar 
el  tema que se va desarrollar, para despertar el interés del lector o de la audiencia y para
interiorizarlos en el tema. También puede denominarse {\bf apertura} o {\color{blue} prólogo.}\vspace{0.5cm}


La introducción se aplica tanto para un texto escrito como 
para un discurso oral. Una correcta introducción debe persuadir al lector o a la audiencia de modo 
que sientan interés en seguir leyendo o en escuchar sobre el tema en cuestión 
incluso, si no tienen conocimientos al respecto. \vspace{1cm}


La palabra introducción 
proviene del latín y significa “hacia el interior”, es decir que conduce 
hacia lo que será el meollo o el argumentde un discurso. 
Desde la retórica, la introducción es 
una forma de anticipar y tiene la propiedad de persuadir. \vspace{0.5cm}

Las principales características de una introducción consiste en: }

\end{multicols} 
%%%%%%%%%%%%%%%%%%%%%%%%%%%%%%%%%%%%%%%%%%%%%%%%%%%%%%%%%%%%%%%%%%%%%%%%%%%%%%%%%%%%%%%%%%%%%%%%%%%%%%%%%%%%%%%%%%%%%%%%%%%%%%%%%
\begin{enumerate}
    \item[\HandCuffRight] Ser breve y brindar los datos generales para entender qué tema se va a desarrollar.
    \item[\HandCuffRight] Plantear de manera clara y ordenada, el tema a desarrollar para que
    justifique su exposición.
    \item[\HandCuffRight] Contener un {\color{blue} \underline{lenguaje} }y términos de fácil decodificación, excepto para
    los disos científiccursos o especializados.
\end{enumerate}
%%%%%%%%%%%%%%%%%%%%%%%%%%%%%%%%%%%%%%%%%%%%%%%%%%%%%%%%%%%%%%%%%%%%%%%%%%%%%%%%%%%%%%%%%%%%%%%%%%%%%%%%%%%%%%%%%%%%%%%%%%%%%%%%%
Para elaborar una introducción es importante responder a las siguientes
preguntas:
%%%%%%%%%%%%%%%%%%%%%%%%%%%%%%%%%%%%%%%%%%%%%%%%%%%%%%%%%%%%%%%%%%%%%%%%%%%%%%%%%%%%%%%%%%%%%%%%%%%%%%%%%%%%%%%%%%%%%%%%%%%%%%%%%
\newpage
%%%%%%%%%%%%%%%%%%%%%%%%%%%%%%%%%%%%%%%%%%%%%%%%%%%%%%%%%%%%%%%%%%%%%%%%%%%%%%%%%%%%%%%%%%%%%%%%%%%%%%%%%%%%%%%%%%%%%%%%%%%%%%%%%
\markboth{Introducción}{}

\begin{enumerate}
    \item[\HandCuffRight]{\bf ¿Cuál es el tema a exponer?.} La respuesta debe evidenciar el argumento, con algunas de sus características y sus causas (sin explayar de manera literal la conclusión de la hipótesis o discurso).
    \item[\HandCuffRight]{\bf ¿Cuál fue el interés para exponer el tema?.} La respuesta debe definir el origen del interés, que puede ser diverso: profesional, académico, informativo o personal.
    \item[\HandCuffRight]{\bf¿Cuál fue la metodología o estrategia utilizada?.} La respuesta debe evidenciar el método elegido o requerido para llevar a cabo la hipótesis o desarrollo del argumento. Por ejemplo, a través de la investigación, teoría, instrumentos de observación, entre otros.
    \item[\HandCuffRight]{\bf ¿Cuál es la finalidad u objetivo del tema?.} La respuesta debe expresar si la intención del discurso es analizar, entretener, diferenciar o comprender un concepto nuevo sobre un tema ya conocido.
\end{enumerate}
%%%%%%%%%%%%%%%%%%%%%%%%%%%%%%%%%%%%%%%%%%%%%%%%%%%%%%%%%%%%%%%%%%%%%%%%%%%%%%%%%%%%%%%%%%%%%%%%%%%%%%%%%%%%%%%%%%%%%%%%%%%%%%%%%

\begin{multicols}{2}
{\ttfamily
Dependiendo del discurso a desarrollar, 
{\color{green}a veces resulta} \\
{\color{green} conveniente elaborar la in- \\troducción 
cuando el trabajo \\
esté finalizado}. Por ejemplo, \\
en los trabajos de investiga-\\ción científica, en los que \\
pueden surgir datos relevantes 
durante el desarrollo del análisis
de la información.
\\

La introducción es parte de una de estructura que aplica
en diferentes tipos de discursos y 
composiciones, como:}

\end{multicols}
%%%%%%%%%%%%%%%%%%%%%%%%%%%%%%%%%%%%%%%%%%%%%%%%%%%%%%%%%%%%%%%%%%%%%%%%%%%%%%%%%%%%%%%%%%%%%%%%%%%%%%%%%%%%%%%%%%%%%%%%%%%%%%%%%
\begin{enumerate}
    \item[i] {\color{red}La narración}. Consiste en un género que, mediante diferentes técnicas
    intelectuales, audiovisuales, etc., introduce al espectador los datos de una
    {\sc novela} o de una {\sc historia}
    \item[ii] {\color{red}La partitura musical}.Consiste en un documento que, mediante el len-
    guaje de símbolos musicales, establece una introducción para la canción
    y da pie al desarrollo de las distintas partes de la melodía.
    \item[iii] {\color{red}El artículo de divulgación científica}Consiste en un trabajo de investigación o comunicación científica,
    por lo que suele emplear un lenguaje técnico y especializado. Debe contener una introducción concisa y
    práctica que justifique la existencia del documento. \newpage \markboth{}{Introducción}
    \item[iv] {\color{red}El artículo periodístico.} Consiste en un documento que puede ser de
    opinión, informativo o crónica, y que proporciona datos detallados sobre
    un acontecimiento. \renewcommand{\thefootnote}{\fnsymbol{footnote}}
    \footnote{Fuente:\textbackslash \textbackslash https: www.caracteristicas.co/introduccion/}
\end{enumerate}

%notas de pie de pagina%

\markboth{Introducción}{}
%%%%%%%%%%%%%%%%%%%%%%%%%%%%%%%%%%%%%%%%%%%%%%%%%%%%%%%%%%%%%%%%%%%%%%%%%%%%%%%%%%%%%%%%%%%%%%%%%%%%%%%%%%%%%%%%%%%%%%%%%%%%%%%%%

\chapter{Tipos de documentos en \LaTeX}
\begin{multicols}{2}
\underline{En \LaTeX es pósible crear diferentes}
\underline{estilos de documentos,cada uno creado}
\underline{para diferentes tipos de necesidades, }
\underline{como:}
\begin{enumerate}
    \item[\HandCuffRight] Estilo book o español libro.
    \item[\HandCuffRight] Estilo article o en español artículo.
    \item[\HandCuffRight] Estilo report o en español reporte.
    \item[\HandCuffRight] Estilo beamer o presentación.
    \item[\HandCuffRight] Estilo slides o transparencia.
    \item[\HandCuffRight] Estilo letter o carta.
\end{enumerate}
A continuación se presentará una breve descripción de algunos estilos de documentos en \LaTeX.
\end{multicols}
\vspace*{12pt}

%linea%%%%%
\begin{center}
    \rule{15cm}{1pt}\par 
\end{center}


\vspace*{12pt}
\section{\bf Estilo \tt book}
En el estilo {\tt book} se dispone del comando \textbackslash chapter\{\}, para capítilos. 
La estructura básica de un documento en el estilo {\tt book} es la siguiente: 
\newpage \markboth{Tipos de documentos en \LaTeX}{}

\begin{center}
\fbox{\parbox[c][9cm]{7cm}{\textbackslash documentclass[opciones]\{book\}\\
\textbackslash title\{...\}\\
\textbackslash author\{...\}\\
\textbackslash date\{...\}\\
\textbackslash begin\{document\}\\
\textbackslash maketitle\\
.....................................\\
\textbackslash chapter\{...\}\\
.....................................\\
\textbackslash section\{...\}\\
..................................... \\
.....................................\\
\textbackslash subsection\{...\}\\
 .....................................\\
\textbackslash subsubsection\{...\}\\
.....................................\\
\textbackslash end\{document\}}}



\end{center}
{\bf
Observaciones generales sobre el estilo {\tt book}}
\begin{enumerate}
    
    \item[\HandCuffRight] {\bf El comando \textbackslash maketitle hace que se produzcan, en una página
    separada, las líneas para título, autor y fecha; al omitirlo, no
    se imprime la página del título.}
    \item[\HandCuffRight] {\bf \LaTeX imprime, en la página del título, la fecha del día actual
    (vigente en el computador) aún si no se usa el comando
    \textbackslash date\{ ... \}. Para eliminar completamente la fecha se debe
    escribir \textbackslash date\{\}.}
    \item[\HandCuffRight] {\bf Por defecto, cada capítulo comienza en una página de nume-
    ración impar (a mano derecha), a menos que se use la opción
    openany (véase la Tabla 2.2 de [1]), y se generan encabezados
    con los títulos de los capítulos (a mano izquierda) y de las
    secciones (a mano derecha). Tales encabezados se pueden suprimir o 
    modificar (véase la sección 2.10 de [1]). Los formatos
    para los títulos de secciones y capítulos también se pueden
    modificar; véase al respecto la sección 8.7 de [1].}
\end{enumerate}

\newpage \markboth{}{Tipos de documentos en \LaTeX}

\begin{enumerate}
    \item[\HandCuffRight] {\bf Los capítulos se numeran automáticamente pero es posible
    crear capítulos no numerados por medio del comando estrella
    \textbackslash chapter*\{ ... \} .}
    \item[\HandCuffRight] {\bf El entorno abstract no está disponible en el estilo book.}
    \item[\HandCuffRight] {\bf \LaTeX tiene herramientas especiales para manipular eficientemente 
    documentos grandes y mecanismos para la generación automática de tablas de contenido, índices 
    y bibliografía (véase el Capítulo 6 de [1]).}
\end{enumerate}
%linea%%%%%
\begin{center}
    \rule{15cm}{2pt}\par 
\end{center}

\section{\bf Estilo \tt article}
{\em La estructura básica de un documento en el estilo {\tt article es la siguiente:} }

\begin{center}
    \fbox{\parbox[c][9cm]{7cm}{\textbackslash documentclass[opciones]\{article\}\\
    \textbackslash title\{...\}\\
    \textbackslash author\{...\}\\
    \textbackslash date\{...\}\\
    \textbackslash begin\{document\}\\
    \textbackslash maketitle\\
    .....................................\\
    \textbackslash section\{...\}\\
    .....................................\\
    \textbackslash subsection\{...\}\\
    ..................................... \\
    .....................................\\
    \textbackslash subsubsection\{...\}\\

    .....................................\\
    \textbackslash end\{document\}}}
\end{center}

\newpage \markboth{Tipos de documentos en \LaTeX}{}
\noindent
Se presenta en la página 15 de [1] una reproducción del archivo ejem.dvi,
ejemplo concreto de un sencillo documento \LaTeX escrito (por dos prestigiosos
autores) con el estilo article. El documento fuente ejem. tex aparece en la
siguiente tabla.

\begin{flushright}
    \fbox{\parbox[c][11cm]{15cm}{\textbackslash documentclass[10pt]\{article\}\\
    \textbackslash usepackage\{utf8\}\{inputenc\} \\
    \textbackslash title\{Un artículo muy aburrido\}\\
    \textbackslash author\{Fernando Fernandez Consuegra \textbackslash thanks \{Con el patrocinio de \\ Colciencias.\}
    \textbackslash \textbackslash Domingo Dominguez Sinsuegra  \textbackslash thanks \{Sin el \\ patrocinio de Colciencias.\}
    \\
    \textbackslash date\{Enero 15 del 2000\} \\
    \textbackslash begin\{document\} \\
    \hspace{2pt}\textbackslash maketitle \\
    \hspace{2pt}\textbackslash begin\{abstract\} \\
    \hspace{2pt}Se presentan los resultados de una exhaustiva investigación \\ 
    \hspace{2pt}\textbackslash end\{abstract\}
    \hspace{2pt}Comenzemos por decir que no tenemos mucho que decir ...\\
    \hspace{2pt}\textbackslash section\{Primera Sección\} \\ 
    \hspace{2pt}Los temas tratados en esta sección pueden ser un tanto ... \\ 
    \hspace{2pt}\textbackslash subsection\{Primera subsección\} \\
    \hspace{2pt}El tema tratado aquí, dada su complejidad, amerita una ... \\ 
    \hspace{2pt}\textbackslash subsubsection\{Primer tópico de la subseción\} \\
    \hspace{2pt}Este es un tópico muy complicado y lo discutiremos en ... \\ 
    \hspace{2pt}\textbackslash section\{Segunda sección\}
    \hspace{2pt}esta es la segunda sección del presente artículo. Es ... \\ 
    \textbackslash end\{document\}
   }}
    
\end{flushright}
Observaciones generales sobre el estilo {\tt article}
\begin{multicols}{2}
\begin{enumerate}
    \item[\HandCuffRight] el comando {\bf \textbackslash maketitle} hace que se produzacan las líneas para título, author y fecha; al omitirlo, no aparece
    ninguna de ellas. Por lo tanto podemos crear documentos muy sencillos omitiendo la instrucción \textbackslash maketitle
\end{enumerate}
\end{multicols}

\newpage \markboth{}{Tipos de documentos en \LaTeX}
\begin{multicols}{2}
\begin{enumerate}
    \item[\HandCuffRight] Nótese \hspace{0.5cm}
    que \hspace{0.5cm} \textbackslash maketitle se
    coloca \hspace{0.5cm} después \hspace{2pt} de
    \textbackslash begin\{document\}, a diferencia de title, \textbackslash author y \textbackslash date,
    que aparecen en el preámbulo.
    \item[\HandCuffRight] \LaTeX separa los títulos largos en dos o más renglones, pero se puede usar \textbackslash \textbackslash dentro de
    \textbackslash title\{ ... \} para forzar separaciones en el título.
    \item[\HandCuffRight] Los nombres de dos o más autores se separan con \textbackslash and; para que aparezcan en renglones diferentes se separan con \textbackslash \textbackslash .
    \item[\HandCuffRight] Si puede utilizar el comando \textbackslash author\{...\} para escribir afiliaciones de los autores o instituciones, separando cada renglón con \textbackslash \textbackslash. 
    \item[\HandCuffRight] Si \hspace{2pt} se \hspace{2pt} omite el comando \textbackslash date\{...\} \LaTeX imprime de todas maneras la fecha del día actual (la fecha vigente en el computador). Para eliminar completamente la fecha se debe escribir \textbackslash date\{\}.
    \item[\HandCuffRight] La instrucción \textbackslash thanks \{ ...\} se puede utilizar en el argumento de los comandos \textbackslash author, \textbackslash title t¿y \textbackslash date para producir notas al pie de página con agradecimientos, \hspace{2pt} direcciones electrónicas u otro tipo de información sobre os autores o el artículo mismo.   \vspace{1cm} \item[\HandCuffRight] Por defecto, \LaTeX deja márgenes superior e izquierdo de una pulgada. Para otras opciones establecidas por defecto, véase
    la Tabla 2.2 de [1]. 
    \item[\HandCuffRight]  Por defecto, las páginas aparecen numeradas en la inferior y no tienen ningún encabezado en la parte superior. No obstante, el usuario puede incluir encabezados, con al numeración de páginas en la parte superior, por medio del comando \textbackslash pagestyle (véase la seción 2.10 de [1]). El usuario puede hacer otras modificaciones al formato de página preestablecido; esto se explica en la sección 2.13 de [1]. También se puede modificar el formato para los tulos de la secciones (tamaño, tipo de letra, justificación, etc): véase al respecto la sección 8.7 de [1].
    \item[\HandCuffRight] Las \hspace{2pt} o \hspace{2pt} divisiones más importantes en el estilo article son las secciones, subsecciones y subsubsecciones, creadas con los comandos \textbackslash section\{...\}, \textbackslash subsection\{...\} \hspace{1cm} y \textbackslash subsubsection\{...\}, respectivamente. Las dos primeras son numeradas automáticamente por \LaTeX, como se aprecia en el ejemplo de la página \newpage \markboth{Tipos de documentos en \LaTeX}{}
    15 de [1], pero podemos controlar esta numeración(véase la sección 6.2 de [1]). \LaTeX 
    también permite crear secciones y subsecciones no numeradas; por medio de los \textquotedblleft comandos estrella\textquotedblright  \hspace{1pt} \textbackslash section*\{ ... \} y \textbackslash subsection*\{ ... \}.

    \item[\HandCuffRight] El entorno \textquoteleft abstract\textquoteright \hspace{1pt} para el resumen del artículo tiene la sintaxis
    \fbox{\parbox[c][]{4cm}{\textbackslash\{abstract\}\\
    ...................\\
    \textbackslash end\{abstract\}}}
    
    y se debe colocar después de
    \textbackslash maketitle. El resumen aparece en letra más pequeña, inmediatamente antes del texto
    del artículo, o en una página
    separada si se usa la opción
    titlepage (Tabla 2.2 de \cite{Rodrigo de Castro Korgi}).
    Podemos hacer que \LaTeX imprima la expresión ’Resumen’, en vez de ’Abstract’, usando
    ya sea el paquete babel (sección
    2.9 de \cite{Rodrigo de Castro Korgi}) o instrucciones es-
    pecíficas (véase la sección 6.5 de
    [1])
\end{enumerate}
\end{multicols}
%linea%%%%%
\begin{center}
    \rule{15cm}{4pt}\par 
\end{center}

\section{\bf Estilo \tt report}
\begin{multicols}{3}
    El estilo report tiene la misma estructura del estilo book, con las siguientes diferencia (véase también la Tabla 2.2 de [1]):
\begin{enumerate}
\item[\HandCuffRight]  El estilo report está diseñado para impresión a una
sola cara (opción oneside). 
\item[\HandCuffRight]Los capítulos pueden comenzar en páginas de numeración par o impar
(opción openany).
\item[\HandCuffRight] Los números de
las páginas aparecen centrados en
la parte inferior y no hay encabezados en la parte superior, aunque éstos se pueden forzar usando la instrucción
\textbackslash pagestyle (sección 2.10 de [1]).
\item[\HandCuffRight] El entorno abstract sí está
disponible; el ’abstract’ o resumen se imprime en una página independiente, no numerada, adicional a la página del título.
\end{enumerate}

\end{multicols}
\begin{thebibliography}{}
    \bibitem{Rodrigo de Castro Korgi} Rodrigo de Castro Korgi, \textit{El universo de \LaTeX}, segunda Edición, Universidad  Nacional de Colombia, (2003)
\end{thebibliography}
    
\end{document}