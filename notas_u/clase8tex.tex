\documentclass[12pt]{book}

\usepackage[reqno]{amsmath} % Paquete para el manejo de expresiones matemáticas [reqno], [leqno] y [fleqn]
\usepackage{amsthm}
\usepackage{amssymb,amsmath,latexsym} %Paquete para llamar símbolos matemáticos
\usepackage[mathscr]{euscript}
\usepackage{graphicx} %paquete para el manejo de transformaciones geométricas de imagénes
\usepackage{color} %paquete para el manejo de color en textos.
%\usepackage[utf8]{inputenc} %paquete para el manejo de caracteres acentuados
\usepackage[french,spanish]{babel} %paquete que genera documentos en diferentes idiomas
\usepackage{enumerate}
\usepackage{multicol} % Paquete para modificar el número de columnas
\usepackage{layout} % Paquete  para revisar los valores de 
\usepackage{verbatim}
\pagestyle{myheadings}  % Estilo de página
%\pagenumbering{arabic} % Estilo de numéración
\hoffset1cm
\newcounter{Teorema}
\newcommand{\Teorema}{\stepcounter{Teorema}{\bf Teorema \theTeorema.} }


\DeclareMathOperator{\arcsec}{arcsec} % Creación de nuevos comandos en latex
\DeclareMathOperator{\Var}{Var}

\DeclareMathOperator*{\Hom}{Hom}

\setcounter{MaxMatrixCols}{15}

\allowdisplaybreaks % Control de cambios de página en alineaciones

\decimalpoint

%\renewcommand{\theequation}{\thesection.\arabic{equation}}
\numberwithin{equation}{section}
%\renewcommand{\theequation}{\theparentequation\arabic{equation}}


%% Nuevos teoremas

\theoremstyle{plain}  % Requiere el paquete amsthm

\newtheorem{thm}{Teorema}[section]
\newtheorem{Corol}[thm]{Colorario}
\newtheorem{Prop}[thm]{Proposición}
\newtheorem{axiom[thm]}{Axioma}
\newtheorem{conj}{Conjetura}
\newtheorem{Def}{Definición}[chapter]
\newtheorem{notacion}[Def]{Notación}
\newtheorem{nota}[Def]{Nota}

\renewcommand{\qedsymbol}{$\heartsuit$}

\begin{document}
\makeatletter	
\def\@roman#1{\romannumeral#1}
\makeatother
%\layout	% Muestra las medidas de las márgenes usadas en el documento.
\renewcommand{\baselinestretch}{1.5}% Controlar la distancia de espaciamiento entre renglones
\renewcommand{\thefootnote}{\arabic{footnote}} % Para hacer que Latex use diferentes simbolos para enumerar los píes de página.
\renewcommand{\refname}{Referencias bibliográficas}
\newcommand{\tto}{\longrightarrow}
\newcommand{\N}{\ensuremath{\mathbb{N}}}
\newcommand{\parcial}[2]{\frac{\partial#1}{\partial#2}}
\newcommand{\Norma}[1]{\Vert#1\Vert}
\newcommand{\upla}[2]{(#1_1,#1_2,$\ldots$,#1_{#2})}
	\newcommand{\uplamatrix}[3]{
	\begin{pmatrix}  
		#1_{11} &  #1_{12}  & \cdots  &  #1_{1 #3} \\   
		#1_{21} &  #1_{22} &  \cdots  & #1_{2 #3}  \\ 
		\vdots     &  \vdots      & \vdots  & \vdots \\
	#1_{#2 1} & #1_{#2 2} &  \cdots &  #1_{#2 #3}
\end{pmatrix}}
	\newcommand{\kupla}[3][k]{
	(#2_{#3},$\ldots$ #2_{#1})	
}




\noindent\textbf{Alineación con } \verb*|aligned| \textbf{y} \verb*|gathered|


\begin{verbatim}
	\begin{equation*}
	\left\{
	\begin{aligned}
		x_1+x_2+\lambda x_3&=\beta\\
		\lambda x_1+x_2+x_3&=1-\beta\\
		x_1+\lambda x_2+x_3&=2\beta
	\end{aligned}
	\right.
\end{equation*}
\end{verbatim}

	\begin{equation*}
	\left\{
	\begin{aligned}
		x_1+x_2+\lambda x_3&=\beta\\
		\lambda x_1+x_2+x_3&=1-\beta\\
		x_1+\lambda x_2+x_3&=2\beta
	\end{aligned}
	\right.
\end{equation*}


\begin{verbatim}
	\begin{equation*}
		\left.
		\begin{aligned}
			x_1+x_2+\lambda x_3&=\beta\\
			\lambda x_1+x_2+x_3&=1-\beta\\
			x_1+\lambda x_2+x_3&=2\beta
		\end{aligned}
		\right\}
	\end{equation*}
\end{verbatim}

	\begin{equation*}
	\left.
	\begin{aligned}
		x_1+x_2+\lambda x_3&=\beta\\
		\lambda x_1+x_2+x_3&=1-\beta\\
		x_1+\lambda x_2+x_3&=2\beta
	\end{aligned}
	\right\}
\end{equation*}


\begin{verbatim}
	\begin{equation*}
		\left\{
		\begin{aligned}
			x_1+x_2+\lambda x_3&=\beta\\
			\lambda x_1+x_2+x_3&=1-\beta\\
			x_1+\lambda x_2+x_3&=2\beta
		\end{aligned}
		\right\}
	\end{equation*}
\end{verbatim}


\begin{equation*}
	\left\{
	\begin{aligned}
		x_1+x_2+\lambda x_3&=\beta\\
		\lambda x_1+x_2+x_3&=1-\beta\\
		x_1+\lambda x_2+x_3&=2\beta
	\end{aligned}
	\right\}
\end{equation*}


\begin{verbatim}
	\begin{equation*}
			\begin{aligned}
			\frac{dx}{dt}+3\frac{dy}{dy}&=1\\
			\frac{dx}{dt}-\frac{dy}{dy}&=2\\
		\end{aligned}
		\qquad
		\begin{aligned}
			x(0)&=1\\
			y(0)&=2
		\end{aligned}
	\end{equation*}
\end{verbatim}


	\begin{equation*}
	\begin{aligned}
		\frac{dx}{dt}+3\frac{dy}{dy}&=1\\
		\frac{dx}{dt}-\frac{dy}{dy}&=2\\
	\end{aligned}
	\qquad
	\begin{aligned}
		x(0)&=1\\
		y(0)&=2
	\end{aligned}
\end{equation*}






\begin{verbatim}
	\begin{equation*}
		\left\{
		\begin{gathered}
			x_1+x_2+\lambda x_3=\beta\\
			\lambda x_1+x_2+x_3=1-\beta\\
			x_1+\lambda x_2+x_3=2\beta
		\end{gathered}
		\right.
	\end{equation*}
\end{verbatim}

\begin{equation*}
	\left\{
	\begin{gathered}
		x_1+x_2+\lambda x_3=\beta\\
		\lambda x_1+x_2+x_3=1-\beta\\
		x_1+\lambda x_2+x_3=2\beta
	\end{gathered}
	\right.
\end{equation*}


\begin{verbatim}
	\begin{equation*}
		\left.
		\begin{gahered}
			x_1+x_2+\lambda x_3=\beta\\
			\lambda x_1+x_2+x_3=1-\beta\\
			x_1+\lambda x_2+x_3=2\beta
		\end{gathered}
		\right\}
	\end{equation*}
\end{verbatim}

\begin{equation*}
	\left.
	\begin{gathered}
		x_1+x_2+\lambda x_3=\beta\\
		\lambda x_1+x_2+x_3=1-\beta\\
		x_1+\lambda x_2+x_3=2\beta
	\end{gathered}
	\right\}
\end{equation*}


\begin{verbatim}
	\begin{equation*}
		\left\{
		\begin{gathered}
			x_1+x_2+\lambda x_3=\beta\\
			\lambda x_1+x_2+x_3=1-\beta\\
			x_1+\lambda x_2+x_3=2\beta
		\end{gathered}
		\right\}
	\end{equation*}
\end{verbatim}


\begin{equation*}
	\left\{
	\begin{gathered}
		x_1+x_2+\lambda x_3=\beta\\
		\lambda x_1+x_2+x_3=1-\beta\\
		x_1+\lambda x_2+x_3=2\beta
	\end{gathered}
	\right\}
\end{equation*}


\begin{verbatim}
	\begin{equation*}
		\begin{gathered}
			\frac{dx}{dt}+3\frac{dy}{dy}=1\\
			\frac{dx}{dt}-\frac{dy}{dy}=2\\
		\end{gathered}
		\qquad
		\begin{gathered}
			x(0)=1\\
			y(0)=2
		\end{gathered}
	\end{equation*}
\end{verbatim}


\begin{equation*}
	\begin{gathered}[t]
		\frac{dx}{dt}+3\frac{dy}{dy}=1\\
		\frac{dx}{dt}-\frac{dy}{dy}=2\\
	\end{gathered}
	\qquad
	\begin{gathered}
		x(0)=1\\
		y(0)=2
	\end{gathered}
\end{equation*}

\vspace{1.5cm}

\noindent\textbf{Alineaciones con }\verb*|flalign|

\begin{verbatim}
	\begin{flalign*}
		x&=ay+b & y&=2x+1 & x=3x-1\\
		x&=cy+d & y&=3x-1 & x=5x-6
	\end{flalign*}
\end{verbatim}

	\begin{flalign*}
	x&=ay+b & y&=2x+1 & x&=3x-1\\
	x&=cy+d & y&=3x-1 & x&=5x-6
\end{flalign*}



\begin{verbatim}
	\begin{flalign}
		x&=ay+b & y&=2x+1 & x=3x-1\\
		x&=cy+d & y&=3x-1 & x=5x-6
	\end{flalign}
\end{verbatim}

\begin{flalign}
	x&=ay+b & y&=2x+1 & x&=3x-1\\
	x&=cy+d & y&=3x-1 & x&=5x-6
\end{flalign}


\vspace{1.5cm}

\noindent\textbf{ Alineación con el entorno } \verb*|eqnarray| \textbf{de} \LaTeX

\noindent Se deja como ejercicio para el estudiante leer este entorno propio de \LaTeX para realizar alineaciones.

\vspace{1.5cm}

\noindent\textbf{espaciamiento vertical en alineaciones}
\begin{verbatim}
	\begin{equation*}
			\begin{aligned}
			x_1+x_2+\lambda x_3&=\beta\\
			\lambda x_1+x_2+x_3&=1-\beta\\
			x_1+\lambda x_2+x_3&=2\beta
		\end{aligned}
	\qquad
				\begin{aligned}
		x_1+x_2+\lambda x_3&=\beta\\[2cm]
		\lambda x_1+x_2+x_3&=1-\beta\\[2cm]
		x_1+\lambda x_2+x_3&=2\beta
	\end{aligned}
	\end{equation*}
\end{verbatim}

	\begin{equation*}
	\begin{aligned}
		x_1+x_2+\lambda x_3&=\beta\\
		\lambda x_1+x_2+x_3&=1-\beta\\
		x_1+\lambda x_2+x_3&=2\beta
	\end{aligned}
	\qquad
	\begin{aligned}
		x_1+x_2+\lambda x_3&=\beta\\[2.0cm]
		\lambda x_1+x_2+x_3&=1-\beta\\[2.0cm]
		x_1+\lambda x_2+x_3&=2\beta
	\end{aligned}
\end{equation*}

	\begin{equation*}
	\begin{aligned}
		x_1+x_2+\lambda x_3&=\beta\\
		\lambda x_1+x_2+x_3&=1-\beta\\
		x_1+\lambda x_2+x_3&=2\beta
	\end{aligned}
	\qquad
	\begin{aligned}
		x_1+x_2+\lambda x_3&=\beta\\[0.25cm]
		\lambda x_1+x_2+x_3&=1-\beta\\[0.25cm]
		x_1+\lambda x_2+x_3&=2\beta
	\end{aligned}
\end{equation*}

\noindent\textbf{Control sobre cambios de página en alineaciones}


\begin{align*}
			x_1+x_2+\lambda x_3&=\beta\\
	\lambda x_1+x_2+x_3&=1-\beta\\
	x_1+\lambda x_2+x_3&=2\beta\\
				x_1+x_2+\lambda x_3&=\beta\\
	\lambda x_1+x_2+x_3&=1-\beta\\
	x_1+\lambda x_2+x_3&=2\beta\\
				x_1+x_2+\lambda x_3&=\beta\\
	\lambda x_1+x_2+x_3&=1-\beta\\
	x_1+\lambda x_2+x_3&=2\beta\\
				x_1+x_2+\lambda x_3&=\beta\\
	\lambda x_1+x_2+x_3&=1-\beta\\
	x_1+\lambda x_2+x_3&=2\beta\\			x_1+x_2+\lambda x_3&=\beta\\
	\lambda x_1+x_2+x_3&=1-\beta\\
	x_1+\lambda x_2+x_3&=2\beta\\			x_1+x_2+\lambda x_3&=\beta\\
	\lambda x_1+x_2+x_3&=1-\beta\\
	x_1+\lambda x_2+x_3&=2\beta\\			x_1+x_2+\lambda x_3&=\beta\\
	\lambda x_1+x_2+x_3&=1-\beta\\
	x_1+\lambda x_2+x_3&=2\beta\\			x_1+x_2+\lambda x_3&=\beta\\
	\lambda x_1+x_2+x_3&=1-\beta\\
	x_1+\lambda x_2+x_3&=2\beta\\			x_1+x_2+\lambda x_3&=\beta\\
	\lambda x_1+x_2+x_3&=1-\beta\\
	x_1+\lambda x_2+x_3&=2\beta\\			x_1+x_2+\lambda x_3&=\beta\\
	\lambda x_1+x_2+x_3&=1-\beta\\
	x_1+\lambda x_2+x_3&=2\beta\\
				x_1+x_2+\lambda x_3&=\beta \\
	\lambda x_1+x_2+x_3&=1-\beta\\
	x_1+\lambda x_2+x_3&=2\beta\\
				x_1+x_2+\lambda x_3&=\beta\\
	\lambda x_1+x_2+x_3&=1-\beta\\
	x_1+\lambda x_2+x_3&=2\beta\\
				x_1+x_2+\lambda x_3&=\beta\\
	\lambda x_1+x_2+x_3&=1-\beta\\
	x_1+\lambda x_2+x_3&=2\beta\\
				x_1+x_2+\lambda x_3&=\beta\\
	\lambda x_1+x_2+x_3&=1-\beta\\
	x_1+\lambda x_2+x_3&=2\beta\\
				x_1+x_2+\lambda x_3&=\beta\\
	\lambda x_1+x_2+x_3&=1-\beta\\
	x_1+\lambda x_2+x_3&=2\beta\\
				x_1+x_2+\lambda x_3&=\beta\\
	\lambda x_1+x_2+x_3&=1-\beta\\
	x_1+\lambda x_2+x_3&=2\beta
\end{align*}

\chapter{Opciones para la enumeración de fórmulas}

\section{Colocación y enumeración de fórmulas}

\begin{align}
				x_1+x_2+\lambda x_3&=\beta\\
	\lambda x_1+x_2+x_3&=1-\beta\\
	x_1+\lambda x_2+x_3&=2\beta
\end{align}


\section{Jerarquía de la enumeración}

\subsection{Prueba de enumeración}

\begin{align}
				x_1+x_2+\lambda x_3&=\beta\\
	\lambda x_1+x_2+x_3&=1-\beta\\
	x_1+\lambda x_2+x_3&=2\beta
\end{align}

\section{Numeración forzada}

\begin{align*}
				x_1+x_2+\lambda x_3&=\beta \tag*{(*)}\\
	\lambda x_1+x_2+x_3&=1-\beta \tag*{(**)}\\
	x_1+\lambda x_2+x_3&=2\beta  \tag*{(***)}
\end{align*}

\begin{align}
	x_1+x_2+\lambda x_3&=\beta \tag*{(*)}\\
	\lambda x_1+x_2+x_3&=1-\beta \tag*{(**)}\\
	x_1+\lambda x_2+x_3&=2\beta  \tag*{(***)}
\end{align}

\begin{align}
	x_1+x_2+\lambda x_3&=\beta \\
	\lambda x_1+x_2+x_3&=1-\beta \\
	x_1+\lambda x_2+x_3&=2\beta 
\end{align}

\section{Numeración subordinada}

\begin{subequations}\label{operaciones}
\begin{align}
A+B &:= \{x+y \mid x\in A,\ y\in B\} \label{suma}\\
AB &:= \{xy \mid x\in A,\ y\in B\} \label{producto}\\
-A &:= \{-x \mid x\in A\} \label{opuesto}\\
A~{-1} &:= \{a~{-1} \mid a\in A,\ a\ne O\} \label{inverso}
\end{align}
\end{subequations}
En (\ref{operaciones}) aparecen las definiciones de nuevos
conjuntos de números reales: (\ref{suma}) define la suma de
subconjuntos, (\ref{producto}) el producto, (\ref{opuesto}) el
opuesto y (\ref{inverso}) el inverso.

\section{Referencias cruzadas}

\begin{verbatim*}
	\begin{subequations}\label{operaciones1}
		\begin{align}
			A+B &:= \{x+y \mid x\in A,\ y\in B\} \label{suma1}\\
			AB &:= \{xy \mid x\in A,\ y\in B\} \label{producto1}\\
			-A &:= \{-x \mid x\in A\} \label{opuesto1}\\
			A~{-1} &:= \{a~{-1} \mid a\in A,\ a\ne O\} \label{inverso1}
		\end{align}
	\end{subequations}
	En \eqref{operaciones1} aparecen las definiciones de nuevos
	conjuntos de números reales: \eqref{suma1} define la suma de
	subconjuntos, \eqref{producto} el producto, \eqref{opuesto1} el
	opuesto y \eqref{inverso1} el inverso.
\end{verbatim*}

\begin{subequations}\label{operaciones1}
	\begin{align}
		A+B &:= \{x+y \mid x\in A,\ y\in B\} \label{suma1}\\
		AB &:= \{xy \mid x\in A,\ y\in B\} \label{producto1}\\
		-A &:= \{-x \mid x\in A\} \label{opuesto1}\\
		A~{-1} &:= \{a~{-1} \mid a\in A,\ a\ne O\} \label{inverso1}
	\end{align}
\end{subequations}
En \eqref{operaciones1} aparecen las definiciones de nuevos
conjuntos de números reales: \eqref{suma1} define la suma de
subconjuntos, \eqref{producto} el producto, \eqref{opuesto1} el
opuesto y \eqref{inverso1} el inverso.

\section{Ajustes en la posición de los números}

\begin{verbatim*}
	\begin{align}
		\parcial{f}{u}(\overrightarrow{x})&=\nabla f(\overrightarrow{x})\cdot u \\ 
		&=\parcial{f}{x_1}(x_1,\ldots,x_n)u_1+\parcial{f}{x_2}(x_1,\ldots,x_n)u_2+\cdots+\parcial{f}{x_n}(x_1,\ldots,x_n)u_n \raisetag{-0.1cm}
	\end{align}
	
	\begin{align}
		\parcial{f}{u}(\overrightarrow{x})&=\nabla f(\overrightarrow{x})\cdot u \\ 
		&=\parcial{f}{x_1}(x_1,\ldots,x_n)u_1+\parcial{f}{x_2}(x_1,\ldots,x_n)u_2+\cdots+\parcial{f}{x_n}(x_1,\ldots,x_n)u_n 
	\end{align}
	
\end{verbatim*}

\begin{align}
\parcial{f}{u}(\overrightarrow{x})&=\nabla f(\overrightarrow{x})\cdot u \\ 
&=\parcial{f}{x_1}(x_1,\ldots,x_n)u_1+\parcial{f}{x_2}(x_1,\ldots,x_n)u_2+\cdots+\parcial{f}{x_n}(x_1,\ldots,x_n)u_n \raisetag{-0.1cm}
\end{align}

\begin{align}
	\parcial{f}{u}(\overrightarrow{x})&=\nabla f(\overrightarrow{x})\cdot u \\ 
	&=\parcial{f}{x_1}(x_1,\ldots,x_n)u_1+\parcial{f}{x_2}(x_1,\ldots,x_n)u_2+\cdots+\parcial{f}{x_n}(x_1,\ldots,x_n)u_n 
\end{align}


\chapter{Teoremas y estructuras relacionadas}


\begin{verbatim}
	\newtheorem{thm}{Teorema}
	\newtheorem{Def}{Definición}
	\newtheorem{notacion}{Notación}
\end{verbatim}

\begin{Def}
	Una función $f$ definida y acotada en un intervalo cerrado $[a,b]$ se dice Riemann integrable en [a,b], si el límite de la sumas de Riemann
	\begin{equation}
		\lim\limits_{\Delta x\to 0}\sum\limits_{k=1}^{n}f(x_i^{*})\Delta x
	\end{equation}
existe y su valor es independiente tanto de las particiones del intervalo como de la selección de  los valores $x_i^{*}$.
\end{Def}


\begin{notacion}
El límite de las sumas de Riemann de  una función $f(x)$ Riemanna integrable en $[a,b]$ se llama la integral de Riemann de la función y se denota 
\begin{equation*}
	\int_{a}^{b}f(x)\, dx,
\end{equation*}
es decir,
\begin{equation*}
	\int_{a}^{b}f(x)\, dx=	\lim\limits_{\Delta x\to 0}\sum\limits_{k=1}^{n}f(x_i^{*})\Delta x
\end{equation*}
\end{notacion}

\begin{thm}\label{thm21}
	Si $f(x)$ es una función continua en el intervalo cerrado $[a,b]$, entonces la función definida por 
	\begin{equation*}
		F(x)=\int_{a}^{x}f(t)\, dt
	\end{equation*}
es deribable para todo $x\in(a,b)$, y 
\begin{equation*}
	\frac{dF}{dx}(x)=f(x)
\end{equation*}
para cada $x\in(a,b)$.
\end{thm}

Por el teorema  \eqref{thm21}...

\section{Enlazar numeración de estructuras definidas con el comando \texttt{\textbackslash newtheorem}}

\begin{verbatim}
	\newtheorem{thm}{Teorema}[chapter]
	\newtheorem{Corol}[thm]{Colorario}
	\newtheorem{Prop}[thm]{Proposición}
\end{verbatim}

\begin{thm}
	Teorema
\end{thm}
\begin{Corol}
	Colorario
\end{Corol}
\begin{Prop}
	Proposición
\end{Prop}

\begin{Def}
	Definición
\end{Def}

\begin{nota}
	Nota
\end{nota}

\section{El entorno \texttt{\textbackslash proof} del paquete \texttt{\textbackslash amsthm}}

\begin{verbatim*}
	\begin{proof}
	Sea ...
	\end{proof}
	
\end{verbatim*}

	\begin{proof}
	Sea ...
\end{proof}


\begin{verbatim*}
	\begin{proof}[Prueba.]
	Sea...
	\end{proof}
\end{verbatim*}

\begin{proof}[Prueba.]
	Sea...
\end{proof}


\end{document}