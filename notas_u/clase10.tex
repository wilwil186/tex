\documentclass[12pt]{book}

\usepackage[reqno]{amsmath} % Paquete para el manejo de expresiones matemáticas [reqno], [leqno] y [fleqn]
\usepackage{amsthm}
\usepackage{amssymb,amsmath,latexsym} %Paquete para llamar símbolos matemáticos
\usepackage[mathscr]{euscript}
\usepackage{graphicx} %paquete para el manejo de transformaciones geométricas de imagénes
\usepackage{color} %paquete para el manejo de color en textos.
%\usepackage[utf8]{inputenc} %paquete para el manejo de caracteres acentuados
\usepackage[french,spanish]{babel} %paquete que genera documentos en diferentes idiomas
\usepackage{enumerate}
\usepackage{pb-diagram}
\usepackage{multicol} % Paquete para modificar el número de columnas
\usepackage{multirow}
\usepackage{array}
\usepackage{layout} % Paquete  para revisar los valores de 
\usepackage{verbatim}
\pagestyle{myheadings}  % Estilo de página
\usepackage{color,graphics,graphicx}%\pagenumbering{arabic} % Estilo de numéración
\usepackage{graphpap}

\hoffset1cm
\newcounter{Teorema}
\newcommand{\Teorema}{\stepcounter{Teorema}{\bf Teorema \theTeorema.} }


\DeclareMathOperator{\arcsec}{arcsec} % Creación de nuevos comandos en latex
\DeclareMathOperator{\Var}{Var}

\DeclareMathOperator*{\Hom}{Hom}
\DeclareMathOperator{\Ker}{Ker}
\DeclareMathOperator{\Img}{Img}
\DeclareMathOperator{\Hf}{Hf}

\setcounter{MaxMatrixCols}{15}

\allowdisplaybreaks % Control de cambios de página en alineaciones

\decimalpoint

%\renewcommand{\theequation}{\thesection.\arabic{equation}}
\numberwithin{equation}{section}
%\renewcommand{\theequation}{\theparentequation\arabic{equation}}


%% Nuevos teoremas

\theoremstyle{plain}  % Requiere el paquete amsthm

\newtheorem{thm}{Teorema}[section]
\newtheorem{Corol}[thm]{Colorario}
\newtheorem{Prop}[thm]{Proposición}
\newtheorem{axiom[thm]}{Axioma}
\newtheorem{conj}{Conjetura}
\newtheorem{Def}{Definición}[chapter]
\newtheorem{notacion}[Def]{Notación}
\newtheorem{nota}[Def]{Nota}

\renewcommand{\qedsymbol}{$\heartsuit$}

\newcolumntype{M}{>{$}c<{$}}


\definecolor{verde1}{rgb}{0.8,1,0.8}
\definecolor{gris}{rgb}{0.5,0.5,0.5}


%\usepackage{draftwatermark}
%\SetWatermarkText{\includegraphics{guacamaya}} % por defecto Draft 
%\SetWatermarkScale{5} % para que cubra toda la página
%\SetWatermarkColor{verde1} % por defecto gris claro
%\SetWatermarkAngle{55}


\begin{document}
	\DeclareGraphicsExtensions{.pdf,.jpeg,png,.eps}
\makeatletter	
\def\@roman#1{\romannumeral#1}
\makeatother
%\layout	% Muestra las medidas de las márgenes usadas en el documento.
\renewcommand{\baselinestretch}{1.5}% Controlar la distancia de espaciamiento entre renglones
\renewcommand{\thefootnote}{\arabic{footnote}} % Para hacer que Latex use diferentes simbolos para enumerar los píes de página.
\renewcommand{\refname}{Referencias bibliográficas}
\newcommand{\tto}{\longrightarrow}
\newcommand{\N}{\ensuremath{\mathbb{N}}}
\newcommand{\parcial}[2]{\frac{\partial#1}{\partial#2}}
\newcommand{\Norma}[1]{\Vert#1\Vert}
\newcommand{\upla}[2]{(#1_1,#1_2,$\ldots$,#1_{#2})}
	\newcommand{\uplamatrix}[3]{
	\begin{pmatrix}  
		#1_{11} &  #1_{12}  & \cdots  &  #1_{1 #3} \\   
		#1_{21} &  #1_{22} &  \cdots  & #1_{2 #3}  \\ 
		\vdots     &  \vdots      & \vdots  & \vdots \\
	#1_{#2 1} & #1_{#2 2} &  \cdots &  #1_{#2 #3}
\end{pmatrix}}
	\newcommand{\kupla}[3][k]{
	(#2_{#3},$\ldots$ #2_{#1})	
}
\textbf{Tablas con columnas especiales}\\

\begin{verbatim}
	\begin{tabular}{|c|c|c|} \hline
		\multirow{3}{*}{Todas están unidas} & 1 & 2 \\  \cline{2-3}
		& 3 & 4\\  \cline{2-3}
		& 5 & 6 \\  \cline{1-3}
	\end{tabular}
	
\end{verbatim}


\begin{tabular}{|c|c|c|} \hline
\multirow{3}{*}{Todas están unidas} & 1 & 2 \\  \cline{2-3}
& 3 & 4\\  \cline{2-3}
& 5 & 6 \\  \cline{1-3}
\end{tabular}


\begin{verbatim}
	\begin{tabular}{|c|c|c|} \hline
		1 & 2  & \multirow{3}{*}{Todas están unidas} \\  \cline{1-2}
		3 & 4 & \\  \cline{1-2}
		5 & 6 & \\  \cline{1-3}
	\end{tabular}
\end{verbatim}

\begin{tabular}{|c|c|c|} \hline
	1 & 2  & \multirow{3}{*}{Todas están unidas} \\  \cline{1-2}
	 3 & 4 & \\  \cline{1-2}
	5 & 6 & \\  \cline{1-3}
\end{tabular}


\begin{verbatim}
	
	\begin{tabular}{|c|c|c|} \hline
		1 & \multirow{3}{*}{Todas están unidas} & 2 \\  \cline{1-1} \cline{3-3}
		3 & & 4  \\  \cline{1-1} \cline{3-3}
		5 & &  6\\  \cline{1-3}
	\end{tabular}
\end{verbatim}


\begin{tabular}{|c|c|c|} \hline
	1 & \multirow{3}{*}{Todas están unidas} & 2 \\  \cline{1-1} \cline{3-3}
	3 & & 4  \\  \cline{1-1} \cline{3-3}
	5 & &  6\\  \cline{1-3}
\end{tabular}

\vspace{1.5cm}

\noindent\textbf{@-expresiones}

\begin{tabular}{|@{\quad $\bullet$\quad}c|c|@{\quad$\triangle$\quad }c|} \hline
	\multicolumn{3}{|@{\quad $\star$}c|}{Fila especial} \\ \hline
	1 & \multirow{3}{*}{Columna especial} & 2 \\  
	3 & & 4  \\
	5 & &  6\\  \cline{1-3}
\end{tabular}


\vspace{1.5cm}
 
\noindent\textbf{Inserción y enumeración de tablas}

\begin{table}[htb]
	\centering
	\begin{tabular}{|@{\quad $\bullet$\quad}c|c|@{\quad$\triangle$\quad }c|} \hline
		\multicolumn{3}{|@{\quad $\star$}c|}{Fila especial} \\ \hline
		1 & \multirow{3}{*}{Columna especial} & 2 \\  
		3 & & 4  \\
		5 & &  6\\  \cline{1-3}
	\end{tabular}
\caption{Tabla explicativa}\label{tabla1}
\end{table}


\vspace{1.5cm}

\noindent\textbf{Tablas con el paquete } \verb*|array|



	\begin{tabular}{|m{3cm}|c|}  \hline
	\multicolumn{2}{|>{\bf}c|}{Criterio de la segunda derivada para campos escalalares} \\ \hline
	 Hessiana  $\Hf$ & Clasificación  \\ \hline
	 
	$Hf(\overrightarrow{a})$ es definida positiva  & Mínimo  local en $\overrightarrow{a}$ \\ \hline
	$Hf(\overrightarrow{a})$ es definida negativa & Máximo local en $\overrightarrow{a}$  \\  \hline
		$Hf(\overrightarrow{a})$ no es definida positiva ni negativa y $\det(\Hf(\overrightarrow{a}))\neq 0$ & Punto de silla en $\overrightarrow{a}$  \\ \hline
		$\det(\Hf(\overrightarrow{a}))=0$ & El criterio no decide \\ \hline
		
	\end{tabular}


\vspace{1.5cm}

\begin{table}[h!]
	\centering
	\begin{tabular}{|M|m{4cm}|M|} \hline
	\multicolumn{3}{|c|}{Soluciones de la ecuación cuadrática $ax^{2}+bx+c=0$} \\ \hline 
	\text{Discriminante} & número de soluciones & \text{Soluciones} \\ \hline
	b^{2}-4 ac=0 & única solución  & x=-\frac{b}{2a} \\ \hline
	b^{2}-4 ac>0 & dos  soluciones reales  & x=\frac{-b\pm\sqrt{	b^{2}-4 ac}}{2a} \\ \hline
	b^{2}-4 ac<0 & dos  soluciones complejas  & x=\frac{-b\pm\sqrt{4 ac-b^{2}}\mathbf{i}}{2a} \\ \hline
	\end{tabular}
\end{table}

\vspace{1.5cm}

\noindent\textbf{Tablas y textos circundante}


\vspace{0.5cm}

	Frase a la izquierda de la tabla \quad
	\begin{tabular}{|c|c|c|} \hline
Uno & Dos & Tres \\ \hline
 & & \\ \hline
  & & \\ \hline
	\end{tabular}


\vspace{2cm}
	Frase a la izquierda de la tabla \quad
	\begin{tabular}[t]{|c|c|c|} \hline
	Uno & Dos & Tres \\ \hline
	& & \\ \hline
	& & \\ \hline
\end{tabular}



\vspace{2cm}
	Frase a la izquierda de la tabla \quad
	\begin{tabular}[b]{|c|c|c|} \hline
		Uno & Dos & Tres \\ \hline
		& & \\ \hline
		& & \\ \hline
	\end{tabular}


\vspace{2cm}

	Frase a la izquierda de la tabla \quad
	\begin{tabular}[t]{|c|c|c|} \firsthline
		Uno & Dos & Tres \\ \hline
		& & \\ \hline
		& & \\ \hline
	\end{tabular}



\vspace{1.5cm}

\newpage
 
\noindent\textbf{Ambiente gráfico de \LaTeX}

\vspace{0.5cm}

\noindent\textbf{El paquete} \verb*|color|


 \vspace{0.5cm}

\textcolor{verde1}{Todos somos así}


 \vspace{0.5cm}

\pagecolor{green}

 \vspace{0.5cm}


\colorbox{red}{Color rojo}

 \vspace{0.5cm}


\fcolorbox{red}{yellow}{Color rojo}


 \vspace{0.5cm}


\setlength{\fboxrule}{3pt}
\fcolorbox{red}{yellow}{Pare}


\newpage

\pagecolor{white}

\noindent\textbf{Los paquetes } \verb*|graphics|  {\bf y} \verb*|graphicx|

\vspace{0.5cm}

\noindent\textbf{Aumento a escala de objetos}

\vspace{0.5cm}

\scalebox{0.5}{Fórmula cuadrática: $x=\frac{-b\pm\sqrt{b^{2}-4ac}}{2a}$}

\scalebox{1}{Fórmula cuadrática: $x=\frac{-b\pm\sqrt{b^{2}-4ac}}{2a}$}

\scalebox{1.5}{Fórmula cuadrática: $x=\frac{-b\pm\sqrt{b^{2}-4ac}}{2a}$}


\scalebox{1.5}[3]{Fórmula cuadrática: $x=\frac{-b\pm\sqrt{b^{2}-4ac}}{2a}$}


\resizebox{8cm}{1.5cm}{Fórmula cuadrática: $x=\frac{-b\pm\sqrt{b^{2}-4ac}}{2a}$}

\resizebox{8cm}{!}{Fórmula cuadrática: $x=\frac{-b\pm\sqrt{b^{2}-4ac}}{2a}$}


\resizebox{!}{!}{Fórmula cuadrática: $x=\frac{-b\pm\sqrt{b^{2}-4ac}}{2a}$}

\resizebox{2\width}{4\height}{Fórmula cuadrática: $x=\frac{-b\pm\sqrt{b^{2}-4ac}}{2a}$}


\noindent\textbf{Reflexión de objetos}

\vspace{0.5cm}

Fórmula cuadrática: $x=\frac{-b\pm\sqrt{b^{2}-4ac}}{2a}$\\

 \reflectbox{Fórmula cuadrática: $x=\frac{-b\pm\sqrt{b^{2}-4ac}}{2a}$}
 
 
 \vspace{0.5cm}
 
 \noindent\textbf{Rotación de objetos} \\
 
 \vspace{0.5cm}
 
 \setlength{\fboxrule}{1pt}
 
 
 \fbox{Mandarin}\quad 
 \rotatebox{45}{ \fbox{Mandarin}}\quad
 \rotatebox{90}{ \fbox{Mandarin}}\quad
  \rotatebox{135}{ \fbox{Mandarin}}\quad
   \rotatebox{180}{ \fbox{Mandarin}}\quad
    \rotatebox{225}{ \fbox{Mandarin}}\quad
     \rotatebox{270}{ \fbox{Mandarin}}\quad
     
     
     \vspace{3cm}
     
      \fbox{Mandarin}\quad 
 \rotatebox[origin=c]{45}{ \fbox{Mandarin}}\quad
  \rotatebox[origin=rc]{45}{ \fbox{Mandarin}}\quad
   \rotatebox[origin=rt]{45}{ \fbox{Mandarin}}\quad
    \rotatebox[origin=ct]{45}{ \fbox{Mandarin}}\quad
     \rotatebox[origin=lt]{45}{ \fbox{Mandarin}}\quad \\
     
      \rotatebox[origin=lc]{45}{ \fbox{Mandarin}}\quad 
           \rotatebox[origin=lb]{45}{ \fbox{Mandarin}}\quad 
                 \rotatebox[origin=cb]{45}{ \fbox{Mandarin}}\quad 
                       \rotatebox[origin=rb]{45}{ \fbox{Mandarin}}\quad 
                       
                       
        \vspace{2cm}
        
        \noindent\textbf{Inclusión de gráficas externas en documentos  \LaTeX}      
       
        \vspace{0.5cm}
%        \begin{center}
%        	       \includegraphics{guacamaya1}   
%        \end{center}
%     
%       
%\begin{center}
%	        \includegraphics{guacamaya}     
%\end{center}
%
%
%\begin{center}
%	\includegraphics{fractal}
%\end{center}
%
%\vspace{3cm}
%
%\noindent\textbf{Empleo de opciones de  }\verb*|\includegraphics{...}|
%
%\begin{center}
%	\fbox{\includegraphics{trex}}
%\end{center}
%
%\begin{center}
%	\includegraphics[draft]{trex}
%\end{center}
%
%\begin{center}
%	\includegraphics{trex} 
%	\includegraphics[scale=0.7]{trex}
%	\includegraphics[scale=0.4]{trex}
%\end{center}
%     
%   \begin{center}
%   	 \includegraphics[width=15cm,height=3.5cm]{trex}    
%   \end{center}
%
%\begin{center}
%	\includegraphics[angle=35,height=4.5cm]{trex}
%	\includegraphics[height=4.5cm,angle=35]{trex}
%\end{center}
%
%
%\begin{center}
%	\includegraphics{trex}
%	\includegraphics[viewport=0 50 120 150,clip]{trex}
%\end{center}
%     
%\begin{center}
%	\includegraphics{gato}
%	\includegraphics[width=10cm,height=5cm,angle=15]{gato}
%	\end{center}     
%     
%    \begin{center}
%    	\renewcommand{\arraystretch}{1.3}
%    	\begin{tabular}{|cb{8cm}|} \hline
%    		\multicolumn{2}{|c|}{\textbf{Tres razas de gatos icónicas}} \\\hline \hline
%    		\includegraphics[scale=0.5]{gatobombay} & \textbf{Bombay:} El bombay es un gato de constitución media y un pelaje negro azabache muy brillante que parece charol. El cráneo es ligeramente redondeado, tiene las orejas redondeadas y un hocico ancho. Sus preciosos ojos de color cobre o dorado están bastante separados y son intensos y expresivos. El cuerpo es firme y musculoso, con un lomo recto y fuerte. La trufa y el contorno del ojo son negros y las almohadillas de las patas son negras o de color marrón oscuro. \\
%    		\includegraphics[scale=0.5]{ragdoll} & \textbf{Ragdoll:} El Ragdoll es un gato grande de cuerpo alargado. Tiene huesos robustos, una cola larga y un pelaje afelpado. Parece aún más grande de lo que es.
%    		
%    		Aunque la cabeza es de tamaño medio, el pelo consigue que la cara parezca grande. Las orejas también son de tamaño medio y están colocadas en los laterales de la cabeza contribuyendo al aspecto triangular de la cara. Las patas son largas y fuertes. El mentón debe estar bien desarrollado, y los ojos ovalados deben ser azules.
%    		 \\ 
%    		
%    		\includegraphics[scale=0.5]{siames} & \textbf{Siames:} Es un gato alargado y elegante. Tiene cuerpo, cuello, patas y cola alargados.
%    		
%    		Esta raza es de tamaño medio, aunque con músculos proporcionados.
%    		
%    		El siamés es un gato de extremos. Su cabeza es un triángulo alargado. Las orejas altas están situadas sobre la cabeza a continuación de dicho triángulo. Su nariz es larga y recta y las patas son largas y esbeltas. Tiene la cola larga y rematada en punta. Los ojos, almendrados, son de un verde brillante. \\ \hline
%    	\end{tabular}
%    \end{center}
     
 \noindent\textbf{Gráficos y tablas como objetos flotantes}   
 
 
     \begin{verbatim}
	\begin{table}[posición]
		... comandos de la tabla
	\end{table}
\end{verbatim}     
 \begin{verbatim}
\begin{figure}[posición]
	... comandos de la tabla
\end{figure}
\end{verbatim}



 se deja como lectura al estudiante
 
% \begin{figure}[h!]
% 	\centering
% 	\includegraphics{gato}
% 	\caption{Figura de gatito}
% \end{figure}
%     
   
 \noindent\textbf{Entorno pinture de \LaTeXe}

  \begin{center}
  	\setlength{\unitlength}{1mm}
  	\begin{picture}(100,100)
    \graphpaper(0,0)(100,100)
  	\end{picture}
  \end{center}



  \begin{center}
	\setlength{\unitlength}{1mm}
	\begin{picture}(100,100)
		{\color{gris}\graphpaper(0,0)(100,100)}
	\end{picture}
\end{center}


  \begin{center}
	\setlength{\unitlength}{1mm}
	\begin{picture}(100,100)
		{\color{gris}\graphpaper(0,0)(100,100)}
		\put(50,50){Gauss}
		\put(40,60){Cauchy}
		\put(30,70){Newton}
	\end{picture}
\end{center}
  
  \vspace{1cm}   
     
  \noindent\textbf{Lineas rectas con } \verb*|\line|   
     
 \begin{center}
\setlength{\unitlength}{1mm}
\begin{picture}(100,60)\thicklines
\put(10,0){\line(2,3){40}} %segmento DA
\put(20,0){\line(1,2){30}} %segmento OB
\put(30,0){\line(1,3){20}} %segmento OC
\put(40,0){\line(1,6){10}} %segmento OD
\put(50,0){\line(0,1){60}} %segmento DE
\put(60,0){\line(-1,6){10}} %segmento OF
\put(70,0){\line(-1,3){20}} %segmento OG
\put(80,0){\line(-1,2){30}} %segmento OH
\put(90,0){\line(-2,3){40}} %segmento DI
\put(10,0){\line(1,0){80}} %segmento Al
\put(8,-5){$A$} \put(18,-5){$B$} \put(28,-5){$C$}
\put(38,-5){$D$} \put(48,-5){$E$} \put(58,-5){$F$}
\put(68,-5){$G$} \put(78,-5){$H$} \put(88,-5){$I$}
\put(49,61){$0$}
\end{picture}
\end{center}    

  \noindent\textbf{Flechas con } \verb*|\vector|   

  \begin{center}
  	\setlength{\unitlength}{1mm}
  	\begin{picture}(50,40)
  		{\color{gris}\graphpaper(0,0)(50,40)}
  		\thicklines
  		\put(5,30){\vector(0,-1){10}}
  		\put(0,10){\vector(1,0){50}}
  		\put(10,0){\vector(0,1){40}}
  		\put(15,20){\vector(3,1){25}}
  		\put(20,30){\vector(2,-3){20}}
  	\end{picture}   
  \end{center} 
  
  
 \noindent\textbf{Circulis con }\verb*|circle|
 
 \begin{center}
 	\setlength{\unitlength}{1mm}
 	\begin{picture}(80,80)
{\color{gris}\graphpaper(0,0)(80,80)}
\thicklines
\put(20,20){\circle{40}}
\put(20,60){\circle{20}}
\put(70,20){\color{gris}\circle*{15}}
\put(70,50){\circle*{10}}
\end{picture}
 \end{center}
 
 \noindent\textbf{El comando }\verb*|oval|
 
 \begin{center}
 	\setlength{\unitlength}{1mm}
\begin{picture}(150,70)
{\color{gris}%
\graphpaper(0,0)(150,70)}
\thicklines
\put(40,20){\oval(60,30)}
\put (40, 50){\oval(60 ,30) [lt]}
\put(100,40){\oval(20,50)}
\put(130,40){\oval(20,50) [b]}
\end{picture}
 \end{center}   
     
     \vspace{1cm}
     
 \noindent\textbf{Cajas}
 
 \begin{center}
\setlength{\unitlength}{1mm}
\begin{picture}(150,40)
{\color{gris}\graphpaper(0,0)(150,40)}
\thicklines
\put(10,10){\framebox(30,20){centro}}
\put(60,10){\framebox(30,20)[t]{arriba}}
\put(110,10){\dashbox{2}(30,20) [br]{extremo}}
\end{picture}
\end{center}

\vspace{1cm}

% \begin{center}
%	\setlength{\unitlength}{1mm}
%	\begin{picture}(150,40)
%		\graphpaper(0,0)(150,40)
%	  \put(5,10){\frame{\includegraphics[scale=0.25]{gato}}}
%	\end{picture}
%\end{center}


\vspace{1cm}

\noindent\textbf{El comando } \verb*|\shortstack|


\begin{center}
	\setlength{\unitlength}{1pt}
\begin{picture}(280,60)
\put(20,0){\shortstack{Se ponen\\palabras\\donde\\faltan\\
las ideas}}
\put(100,0){\shortstack{Se ponen palabras donde faltan las\\
i\\d\\e\\a\\s}}
\end{picture}
\end{center}
 
 \noindent\textbf{Curvas cuadráticas de Bézier}
 
 \begin{center}
\setlength{\unitlength}{1mm}
\begin{picture}(100,50)
{\color{gris}\graphpaper(0,0)(100,50)}
\thicklines
\qbezier(10,0)(40,40)(70,10)
\qbezier[55](0,20)(50,50)(100,10)
\end{picture}
\end{center}
 
  \begin{center}
 	\setlength{\unitlength}{1mm}
 	\begin{picture}(100,50)
 		{\color{gris}\graphpaper(0,0)(100,50)}
 		\thicklines
 		\qbezier(10,0)(40,40)(70,10)
 		\qbezier[55](0,20)(50,50)(100,10)
 	\end{picture}
 \end{center}
 
   \begin{center}
 	\setlength{\unitlength}{1mm}
 	\begin{picture}(100,50)
 		{\color{gris}\graphpaper(0,0)(100,50)}
 		\thicklines
 		{\color{red}\qbezier(0,25)(15,65)(31.416,25)}
 	 {\color{red}\qbezier(30,25)(47.123,-15),(62.832,25)}
 	\end{picture}
 \end{center}
 
 \vspace{1cm}
 
%  \begin{center}
% 	\setlength{\unitlength}{1mm}
% 	\begin{picture}(150,40)
% 		\graphpaper(0,0)(150,40)
% 		\multiput(0,10)(15,0){10}{\frame{\includegraphics[scale=0.15]{gato}}}
% 	\end{picture}
% \end{center}
 
 
 
\end{document}
