

\begin{document}
% 1. Realice el siguiente ejercicio en un documentoo tipo articulo, con letra a 12pt y con estilo de página
%vacío
% 2. Escriba el siguiente escrito:
Escribir textos es muy sencillo en \LaTeX 
hasta que llega alguna pabra que toca poner en \textbf{negrilla}
o en \texttt{MÁQUINA DE ESCRIBIR}, o que la palabra sea muy {\tiny grande} o muy  {\Huge pequeñas}. \\ 

\LaTeX nos permite realizar una graan variedad de textos. Podemos realizar desde un \underline{reporte}, 
un \textquotedblleft artículo \textquotedblright hasta un {\large Libro}. También nos permite poner 

\begin{center}
    El texto centrado
\end{center}
o
\begin{flushleft}
    \textbf{\huge El texto a la izquierda}
\end{flushleft}
o
\begin{flushright}
    \texttt{EL TEXTO CENTRADO A LA DERECHA}
\end{flushright}
También nos facilita la escritura de diferentes simbolos como:
\begin{center}
    \{\%,\$, $\mathsection$, $\dagger$, $\varcopyright$\}
\end{center}
y muchos más \dots \\ 
Pero cuando se escribe en \LaTeX nunca olvidar tener en cuenta 
\begin{itemize}
    \item Editar las \textbf{\Huge márgenes}
    \item Cargar todos los \textit{\tiny paquetes} necesarios.
    \item Recordar escribir su texto con buena \textbf{\texttt{ortografía}}
\end{itemize}
\end{document}