\documentclass[12pt]{book}

\usepackage[reqno]{amsmath} % Paquete para el manejo de expresiones matemáticas [reqno], [leqno] y [fleqn]
\usepackage{amsthm}
\usepackage{amssymb,amsmath,latexsym} %Paquete para llamar símbolos matemáticos
\usepackage{amsfonts}
\usepackage[mathscr]{euscript}
\usepackage{graphicx} %paquete para el manejo de transformaciones geométricas de imagénes
\usepackage{color} %paquete para el manejo de color en textos.
%\usepackage[utf8]{inputenc} %paquete para el manejo de caracteres acentuados
\usepackage[french,spanish]{babel} %paquete que genera documentos en diferentes idiomas
\usepackage{enumerate}
\usepackage{multicol} % Paquete para modificar el número de columnas
\usepackage{layout} % Paquete  para revisar los valores de 
\usepackage{verbatim}
\pagestyle{myheadings}  % Estilo de página
%\pagenumbering{arabic} % Estilo de numéración
\hoffset1cm
\newcounter{Teorema}
\newcommand{\Teorema}{\stepcounter{Teorema}{\bf Teorema \theTeorema.} }


\DeclareMathOperator{\arcsec}{arcsec} % Creación de nuevos comandos en latex
\DeclareMathOperator{\Var}{Var}

\DeclareMathOperator*{\Hom}{Hom}

\setcounter{MaxMatrixCols}{15}

\allowdisplaybreaks % Control de cambios de página en alineaciones

\decimalpoint

%\renewcommand{\theequation}{\thesection.\arabic{equation}}
\numberwithin{equation}{section}
%\renewcommand{\theequation}{\theparentequation\arabic{equation}}


%% Nuevos teoremas

\theoremstyle{plain}  % Requiere el paquete amsthm

\newtheorem{thm}{Teorema}[section]
\newtheorem{Corol}[thm]{Colorario}
\newtheorem{Prop}[thm]{Proposición}
\newtheorem{axiom[thm]}{Axioma}
\newtheorem{conj}{Conjetura}
\newtheorem{Def}{Definición}[chapter]
\newtheorem{Ej}{Ejemplo}[chapter]
\newtheorem{notacion}[Def]{Notación}
\newtheorem{nota}[Def]{Nota}

\renewcommand{\qedsymbol}{$\heartsuit$}
\providecommand{\abs}[1]{\lvert#1\rvert} %valor absoluto
\providecommand{\norm}[1]{\lVert#1\rVert} %norma

\begin{document}

\chapter{}

\noindent \textbf{{\huge CAMPOS VECTORIALES E\\
INTEGRALES DE LÍNEA}}

\section{Campos vectoriales}

\begin{Def}
{\it
Sea $D$ una región abierta en $\mathbb{R}^{n}$. Un \textbf{Campo vectorial} en $D$ es una apliclación F que a cada punto $p \in D$ le asigna un vector $F(p) \in \mathbb{R}^{n}$, con $m>1$. Si denotamos por $\vec{x}$ el vector posición de p, entonces podemos describir el campo vectorial por la función vectorial. Las funciones $f_{i}: D \to \mathbb{R}$ se llaman \textbf{componentes} del campo F. Si las componente $f_{i}$ son derivables decimos que el campo vectorial F es derivable. 

}
\end{Def}
{\it
\begin{Ej}

Sea $D \subseteq \mathbb{R}^{n}$ una región abierta y $f: D \to \mathbb{R}$ una función derivable. Entonces el campo  vectorial 


\begin{align*}
	F(\vec{x})&=\bigtriangledown f(x_{1},\cdots,x_{n}\\
	&=(\frac{\partial f}{\partial x_{1}}(x_{1},\cdots,x_{n}),\cdots,\frac{\partial f}{\partial x_{n}})
\end{align*}

\end{Ej}
\noindent se llama \textbf{Campo vectorial gradiente}. Los vectores del campo gradiente son ortogonales a las superficies de nivel de la función f.
}
\vfill
En muchos casos para entender un campo vectorial necesitamos dibujarlo, y esto no resulta una tarea no muy corta. Podemos usar para esta tarea ciertas líneas de campo, un concepto muy importante
\newpage


{\it
\begin{Def}
Una \textbf{línea de campo} de un campo vectorial $F(\vec{x})$ es una curva $\vec{r}(t)$, tal que
\begin{equation*}
    \frac{d\vec{r}}{dt} = F(\vec{r}(t)).
\end{equation*}
\end{Def}
Geométricamente significa que el campo vectroial F es tangente a sus líneas de campo en cada punto. 
Analicamente, las línea de campo de un campo vectorial $F(x_{1}, \cdots, x_{n}$ con componentes $f_{1},f_{2} \cdots f_{n}$ son las soluciones del sistema de ecuaciones diferenciales 
}
\begin{equation*}
\begin{cases}
\frac{dx_{1}}{dt}(t)=f_{1}(x_{1}(t), \cdots , x_{n}(t)) \\
\frac{dx_{2}}{dt}(t)=f_{2}(x_{1}(t), \cdots , x_{n}(t)) \\ 
\vdots \\ 
\frac{dx_{n}}{dt}(t) = f_{n}(x_{1}(t),\cdots , x_{n}(t)
\end{cases}
\end{equation*}

\section{Intregales de Línea}
{\it 
\begin{Def}
(Intregal de línea sobre un campo escalar). Sea $f: D \to \mathbb{R}$ una función continua, donde $D \subseteq \mathbb{R}^{n}$ es una región abierta. Y sea $\gamma$ una curva suave en $D \subseteq \mathbb{R}^{n}$ con una ecuación dada por una función vectorial $\vec{r}:[a,b] \to D$, $\vec{r} = \vec{r}(s)$, donde s es el parámetro de longitud de arco y b-a es la longitud de la curva $\gamma$. Entonces $f(\vec{r}(s))$ es una función real continua sobre el dominio $[a,b]$
\end{Def}
La \textbf{intregal de línea} de la función f a lo largo de la curva $\gamma$, donde $\gamma$ esta parámetrizada en términos del parámetro natura s (longitud de arco), $s\in [a,b]$, es 
}
\begin{equation*}
    \int_{\gamma}fds=\int_{a}^{b}f(\vec{r(s)}ds
\end{equation*}
{\it
Si tenemos una ecuación $\vec{r} = \vec{p}(t). t \in [c,d]$ de la curva dada $\gamma$, respecto a un parámetro arbitrario t al parámetro natural s aplicando la fórmula }
\begin{equation*}
    s = \int_{c}^{t} \norm{p'(u)}du \Rightarrow ds = \norm{p'(t)}dt
\end{equation*}
\newpage
entonces la intregal de línea a lo largo de $\gamma$, de cualquier parametrización $\vec{p}(t)$ de $\gamma$, es
\begin{equation*}
    \int_{\gamma}fds = \int_{c}^{d} f(\vec{p}(t)) \norm{\vec{p}'(t)}dt
\end{equation*}
\begin{Ej}
La intregral de línea de la función f(x,y)=xy a lo largo de la circunferencia con centro en el origen y radio $r>0$ es 
\end{Ej}

\begin{align*}
    \int_{\gamma} f ds &= \int_{0}^{2\pi} (r \cos{t})(r \sin{t})\norm{(-r \sin{t},r \cos{t})}dt \\
    &= \int_{0}^{2\pi} r^{3} \cos{t} \sin{t} dt \\
    &= 0
\end{align*}
\begin{Def}
(Intregal de línea sobre un campo vectorial). Sea $F: D \to \mathbb{R}^{n}$ un campo vectorial continuo, donde $D \subseteq \mathbb{R}^{n}$ una región abierta. Sea $\gamma$ una curva suave en $D$ con una ecuación dada por una ecuación vectorial $\vec{r}:[a,b] \to D$, $\vec{r}=\vec{r}(t)$. Entonces $F(\vec{r}(t))$ es una función vectorial continua sobre el dominio $[a,b]$.
\end{Def}
La \textbf{Intregal de línea} del campo vectorial F a lo largo de la curva $\gamma$ es 
\begin{equation*}
    \int_{\gamma} F \cdot d\vec{r} = \int_{a}^{b} F(\vec{r}(t)) \cdot \vec{r}'(t)dt
\end{equation*}
\begin{Ej}
la intregal de línea del campo vectorial $F(x,y)=(x+y,y)$ a lo largo de la curva con parametrización $\vec{r}(t)=(\cos{t},\sin{t}), t \in [0,2\pi]$ es 
\end{Ej}
\begin{align*}
\int_{\gamma} F \cdot d\vec{r} &= \int_{0}^{2\pi} (\cos{t} +\sin{t},\sin{t}) \cdot (- \sin{t}, \cos{t})dt \\
&= \int_{0}^{2 \pi} - \sin^{2} dt \\ 
&= \frac{1}{2} \int_{0}^{2 \pi} \cos{2t}-1dt \\ 
&= -\pi
\end{align*}
\end{document}
