\documentclass[12pt]{article}
\usepackage[utf8]{inputenc}
\usepackage[spanish]{babel}
\usepackage{hyperref}
\usepackage{graphicx}
\usepackage{geometry}
\usepackage{multicol}
\usepackage{amsmath}
\usepackage{amsfonts}

\begin{document}

%presentación
\pagestyle{empty}
%\afterpage{\blankpage}
\begin{center}
\begin{figure}[h]
\centering


\end{figure}
\Large
\hrule
\vspace{4mm}
\textbf{Trabajo opcional de Ecuaciones diferenciales}\\

\vspace{4mm}
\hrule
\large
\vfill
Autor\\

Wilson Eduardo Jerez Hernández \\
20181167034
\vfill
Profesora\\

Yudy Marcela Bolaños Rivera
\vfill
Universidad Distrital Francisco José de Caldas\\
Facultad de Ciencias y Educación\\
Curso de Ecuaciones diferenciales\\
\end{center}
\newpage
\section{Ejercicio}
Compruebe que la función \\ 
$x:(0, \infty) \to \mathbb{R}$ \\
tal que $t \to x = x(t) = \frac{1}{t}$, \\ 
es solución de la ecuación diferencial \\
\begin{center}
    $tx'+x=0$
\end{center}
\textbf{Desarrollo}\\ 
Primero que todo derivamos la función\\ 
\begin{equation*}
    x = \frac{1}{t} \to x'=(t^{-1})' \to x'=-t^{-2} \to x'=-\frac{1}{t^{2}}
\end{equation*}
Ahora remplazamos los valores de x' y de x en la ecuación diferencial 
\begin{equation}
    t(-\frac{1}{t^{2}})+\frac{1}{t} = 0 \to -\frac{1}{t}+\frac{1}{t}=0 \to 0=0
\end{equation}
por lo que concluimos que la función es solucion de la Ecuación diferencial.
\end{document}